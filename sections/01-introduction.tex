The evaluation of measurement, prediction, and forecasting methods become increasingly important and sophisticated as technologies and data availability enable their application in more and more areas. 
Conventionally, methods are evaluated using distance measures that take aggregated local absolute differences between predictions (or measurements) and target values into account. 
These measures fail to consider whether the right direction of change is predicted or measured correctly.
The information about whether a signal, be it a measurement device, forecast, or nowcast, can accurately predict an increase or decrease is crucial when making decisions based on the estimator's prediction results. 
In the following, we provide an overview over the characteristics and current evaluation schemes of the three application areas, measurement, forecasting, and nowcasting methods.
We highlight why trending evaluation is highly relevant in the respective fields.

Measurement aims to obtain accurate and reliable data about the current state of a system. 
A parameter or variable is can be measured regularly over a certain period to evaluate the systems development.
When introducing new measurement methods they are evaluated against current measurement techniques, called gold standard, by measuring the same quantity in different settings, for example, different individuals or environments.
In addition to using a paired t-test to determine systematic differences between two measurement series (\cite{watson2010method}), various indices such as the interclass correlation coefficient or Lin's concordance correlation have been proposed in practice to check the reproducibility of the measurement or to compare different measurement methods (\cite{lawrence1989concordance,koo2016guideline}). 
Furthermore, graphical tools were developed, such as the Bland-Altman diagram, visualizing the differences between the methods in relation to their mean value (\cite{bland1986statistical}). 
Trending considerations, that is the consistency of the test methods development with the gold standard, have already gained importance in literature (\cite{Saugel2015,saugel2018error,hiraishi2021concordance}). 
In this paper, we build upon the existing methods and extend them with new functionality.

The above-described trending idea is also of fundamental interest for evaluating and comparing forecasting methods. 
While measurement series are created by recording data using measuring instruments or direct observation, forecasting is concerned with predicting the future based on historical data, its patterns, and exogenous factors. 
The forecast is computed based on the current value of the quantity of interest and an estimate of the development until the target time is made.
A forecast's trending is perfectly consistent with the actual development of the target value if the actual change in the target value over this period matches the forecast change. 
In the current practice, a forecasting method is usually evaluated in terms of its prediction accuracy, measuring the difference between the prediction and the actual outcome of the target variable; different measures are, for example, reviewed in \cite{hyndman2006another}. 
Popular measures of this type are scale-dependent measures such as the \ac{rmse} or measures based on percentage errors such as the root mean squared percentage error. 
All of these typical measures are based on the absolute difference between the prediction and the true values, locally or globally. 
They are not capable of assessing the trending ability discussed above.

The evaluation of trending is also interesting for the assessment of nowcasts.
In contrast to conventional forecasting methods, nowcasting methods focus on predictions for the present, the immediate future, and the recent past (\cite{banbura2013now}) and are now widely used in fields such as economics and medicine (\cite{bok2018macroeconomic, Wolffram2023}).
Nowcasting has its origins in meteorology, and the methods were initially developed to describe the current state of the weather in detail and to predict the expected change on a time scale of a few hours (\cite{browning1989nowcasting,schmid2019nowcasting}). In economics, nowcasting is used, for example, to predict statistics such as the current economic situation (for example, GDP), which are collected with low frequency and are available with a considerable time delay (\cite{banbura2013now}).
Nowcasting methods use indicators that are related to the target variable and collected at a higher frequency.
Thus, the nowcasts can produce early and ongoing estimates for the target variable during the relevant period (\cite{castle2017forecasting}). 
In medicine, epidemic nowcasting assesses the current situation during an ongoing epidemic, considering the main pathogenic, epidemiological, clinical, and socio-behavioral factors (\cite{wu2021nowcasting}). 
For example, the nowcasting method can correct the daily COVID case numbers for events that have occurred but have not yet been reported before this data has been fully recorded at a central location (\cite{gunther2021nowcasting}). 
In this context, nowcasting is an estimation procedure in which the value of a target variable for a certain time, for example, today, is estimated based on current preliminary measurements, which are only finalized later. 
At the time of nowcasting, estimates of the target variable can be made for the current time and for a past time; the predicted change in the target variable is then the difference between the two nowcasts. A nowcast is consistent with the evolution of the target variable if the difference between the actual values at the current time and the actual values at a past time has the same sign as the predicted change. Like forecasts, nowcasts are often evaluated and compared in the literature based on performance measures such as the \ac{rmse} (\cite{gunther2021nowcasting}) or probabilistic measures \parencite{Wolffram2023}. The aspect of trending ability is not considered in the literature to the best of our knowledge. However, appropriate measures for assessing trending ability should be constructed because a nowcast with strong trending ability provides crucial information about the development of the target variable, for example, the number of cases of an epidemic, and can be the foundation of decisions on the introduction or cancellation of policy measures.

As outlined above, trending evaluation is crucial for measurements, predictions, and forecasts. 
However, regardless of the application area, most conventional method comparisons and assessments consider local absolute deviations without assessing whether the correct direction of change is being predicted or measured. In this paper, we want to complement the current evaluation measures with trending measurements and propose evaluation methods that assess the trending ability of a method.

The main contributions of this paper are summarized below
\begin{itemize}
\item We formalize trending and show how the predicted changes can be computed to assess trending for practical relevance. 
\item We present different variants of trending measures that consider either noiseless data or data with noise and small non-informative changes.
\item We introduce the conditional trending plot, a new graphical method for assessing local trending behavior.
\item We introduce bootstrap methods for calculating confidence intervals for the presented treding ratios.
\item We extend the concept of trending to probabilistic predictions.
\item We apply trending evaluation to three applications: Measurement, nowcasting, and forecasting data. 
\item We provide a ready-to-use code for trending evaluation. The code and the source code to replicate the results of our study are available in our public repository at ... .
\end{itemize}

The remainder of this paper is structured as follows. 
In Section \ref{sec:trending}, we formalize the concept of trending independently of the application domain and investigate quantification approaches for noise-free and noisy data. Noisy data refers to data where noise or unsystematic effects are present in the true values or predictions.
In particular, we introduce a new graphical method for evaluating trending, the conditional trending plot, which can represent the simultaneous evaluation of different intervals and asymmetries. 
Bootstrap methods for calculating confidence intervals for trending measures are also presented. 
In Section \ref{sec:application}, we show the results of applying our method to several practical examples from measurement, forecasting, and nowcasting. 
We conclude our paper in Section~\ref{sec:conclusion}.
