\unsure[inline]{Der Beginn ist jetzt sehr auf Measurement fixiert. Könnte man das vielleicht umdrehen: Analysen von Measurement/Nowcast/Forecast sind im Moment vor allem bezüglich Abstand (RMSE; MAE; aber auch probabilistische Analyse). Wichtige Info ist aber auch auch, wie gut ein Anstieg/Fall vorhergesagt wird. Dann drauf eingehen, wie das bisher in den Bereichen gemacht wird.}

Measurement, prediction, and forecasting methods are becoming increasingly important and sophisticated in a world of fast-developing technologies and increasing data availability. While introducing new medical measurement methods requires a comprehensive evaluation and comparison with established standard measurement methods to ensure their validity and reliability, it is generally essential for forecasts and nowcasting methods to assess the extent to which the predictions correspond to the development of reality. Conventionally, the methods are evaluated using distance measures that take aggregated local absolute differences between predictions (or measurements) and target values into account. For example, the continuously ranked probability score (CRPS) can be used to evaluate the accuracy of probabilistic predictions. It is also based on the integrated squared differences between the cumulative distribution function (CDF) of the prediction and the Heaviside jump function. These measures all fail to consider whether the right direction of change is being taken, which can often be highly relevant in practice. The information about whether an estimator, be it a measurement device, forecast, or nowcast, can accurately predict an increase or decrease is crucial when taking decisions based  on  the estimator's prediction results. In the following, we first provide an overview of the characteristics and evaluations of the three areas: Measurement, Prediction and Nowcasting and their relationship with Trending.
 
Measurement aims to obtain accurate and reliable data about the current state of a system. Often, a parameter or variable is measured at regular intervals over a certain period. In addition to using a paired t-test to determine systematic differences between two measurement series (\cite{watson2010method}), various indices such as the interclass correlation coefficient or Lin's concordance correlation have been proposed in practice to check the reproducibility of the measurement or to compare different measurement methods (\cite{lawrence1989concordance,koo2016guideline}). Furthermore, graphical tools were developed, such as the Bland-Altman diagram, visualizing the differences between the methods in relation to their mean value (\cite{bland1986statistical}). As already described above, one important aspect of testing the consistency of the new method with the gold standard is to adequately validate if the new method is measuring changes in the target variable in the right direction when the gold standard detects changes during the measurement process. 
Ideally, the signs of the changes between two consecutive points in time with a prescribed lag measured by the new and gold standard methods should always be the same. 
In this case, we call the new method trending perfect in line with the gold Standard method and refer to this type of agreement as trending ability. 
This type of consideration of the trending ability of a measurement method to the gold standard has also recently gained importance in the literature (\cite{saugel2015tracking, saugel2018error, hiraishi2021concordance}). 

%In (\cite{hiraishi2021concordance}), the authors proposed a new concordance rate for the four-quadrant diagram, which is based on the multivariate normal distribution and takes into account the correlation between the data of the individual subjects.

The above-described idea of trending is also of fundamental interest for evaluating and comparing forecasting and methods. 
While measurement series are created by recording data using measuring instruments or direct observation, forecasting is about predicting the future based on historical data, its patterns, and exogenous factors. 
When forecasting at a specific time, the forecast value at the target time corresponds to a prediction of the change in the target variable between the forecast origin and the target time. 
This predicted change results from the difference between the forecast value at the target time and the current actual value of the target variable at the time of prediction. 
A forecast is perfectly consistent with the actual development of the target value if the actual change in the target value over this period matches the forecast change. 
The current practice is that a forecasting method is usually evaluated in terms of its prediction accuracy, which is defined by a certain loss function and measures the difference between the prediction and the actual outcome of the target variable; an overview of the different measures can be found in \cite{hyndman2006another}. 
Popular measures of this type are scale-dependent measures such as the Root Mean Square Error (RMSE) or measures based on percentage errors such as the Root Mean Square Percentage Error (RMSPE). 
All of these typical measures are based on the absolute difference between the prediction and the true values, locally or globally. They are not capable of assessing the trending ability discussed above.

The evaluation of trending is also interesting for the assessment of nowcasts.
In contrast to conventional forecasting methods, nowcasting methods focus on predictions for the present, the immediate future, and the recent past (\cite{banbura2013now}) and are now widely used in fields such as economics and medicine (\cite{bok2018macroeconomic, wolffram2023collaborative}).
Nowcasting has its origins in meteorology, and the methods were initially developed to describe the current state of the weather in detail and to predict the expected change on a time scale of a few hours (\cite{browning1989nowcasting,schmid2019nowcasting}). In economics, nowcasting is used, for example, to predict statistics such as the current economic situation (for example, GDP), which are collected with low frequency and are available with a considerable time delay (\cite{banbura2013now}).
Nowcasting methods use indicators that are related to the target variable and collected at a higher frequency.
Thus, the nowcasts can produce early estimates for the target variable during the relevant period (\cite{castle2017forecasting}). For example, in forecasting GDP, we use economic indicators collected at a higher frequency from January to March 2024 to forecast GDP from the first quarter of 2024 before the official statistics are published.
In medicine, epidemic nowcasting is used to assess the current situation during an ongoing epidemic, considering the main pathogenic, epidemiological, clinical, and socio-behavioral factors (\cite{wu2021nowcasting}). 
For example, the nowcasting method can correct the daily COVID case numbers for events that have occurred but have not yet been reported before this data has been fully recorded at a central location (\cite{gunther2021nowcasting}). 
In this context, nowcasting is an estimation procedure in which the value of a target variable (for example, from today) is estimated based on current preliminary measurements, which are only finalized later. 
At the time of nowcasting, estimates of the target variable can be made for the current time and for a past time; the predicted change in the target variable is then the difference between the two nowcasts. A nowcast is consistent with the evolution of the target variable if the difference between the actual values at the current time and the actual values at a past time has the same sign as the predicted change. Like forecasts, nowcasts are often evaluated and compared in the literature based on performance measures such as the root mean absolute error (\cite{gunther2021nowcasting}) or probabilistic measures \parencite{Wolffram2023}. The aspect of trending ability is not considered in the literature to the best of our knowledge. However, appropriate measures for assessing trending ability should be constructed because a nowcast with strong trending ability provides crucial information about the development of the target variable, for example, the number of cases of an epidemic, and can be the foundation of decisions on the introduction or cancellation of policy measures.

As outlined above, trending evaluation is crucial for measurements, predictions, and forecasts. 
However, regardless of the application domain, most conventional method comparisons and evaluations consider local absolute differences in different ways without directly assessing whether the correct direction of change is output. 

\textbf{Contributions.} The main contributions of this paper are summarized below
\begin{itemize}

\item We formalise the concept of trending and show how the predicted changes can be meaningfully computed to assess trendability. 

\item We present different variants of trending measures that consider noiseless data as well as noisy and small non-informative changes.

\item We introduce the conditional trending plot, a new graphical method for assessing local trends.

\item We introduce bootstrap methods for calculating  confidence intervals for the presented treding ratios.

\item We extend the concept of trending to probabilistic predictions.

\item We apply trending evaluation to three different applications: Measurement, nowcasting and prediction data. 

\item We provide a ready-to-use code for trending detection. The code and the source code to replicate the results of our study are available in our public repository at ... .
\end{itemize}

The remainder of this paper is structured as follows. 
In Section \ref{sec:trending}, we formalize the concept of trending independently of the application domain and investigate quantification approaches for noise-free and noisy data. Noisy data refers to data where noise or unsystematic effects are present in the true values or predictions.
In particular, we introduce a new graphical method for evaluating trending, the conditional trending plot, which can represent the simultaneous evaluation of different intervals and asymmetries. 
Bootstrap methods for calculating confidence intervals for trending measures are also presented. 
In Section \ref{sec:application}, we show the results of applying our method to several practical examples from measurement, forecasting and nowcasting. 
We conclude our paper with the last section.
