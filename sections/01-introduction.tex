
\begin{itemize}
    \item Aktualität/Wichtigkeit herausstellen: 
    \begin{itemize}
        \item zwei Nowcasts mit gleichem MSE, aber unterschiedlichem Trending $\rightarrow$ Vergleich zwischen Nowcasts
    \end{itemize}
    \item Einordnung in Literatur
    \item Contributions
    \begin{itemize}
        \item Argue, why trending is important and why other measures do not \enquote{detect} it
        \item (Formalisation of Trending Ability)
        \item Evaluation and review of existing Approaches (both measures and graphical assessment)
        \item Development of new measure and graphical method to assess trending for measurements, nowcasts and forecasts
        \item Application to various data examples from practice
        \item Publish ready-to-use code
    \end{itemize}
    \item Überblick über Paper
\end{itemize}

The paper is organized as follows. In Section \ref{sec:notation}, we compare the three possible application areas for trending assessment: measurement, forecasting and nowcasting, and define the time series of change according to the application for a given time series. In Section \ref{sec:trending}, we formalise the concept of trending independently of the application domain and investigate quantification approaches. In particular, we introduce a new graphical method for evaluating trending, the conditional trending plot, which can represent the simultaneous evaluation of different intervals and asymmetries. Bootstrap methods for calculating confidence intervals for trending measures are also presented. In Section \ref{sec:application}, we show the results of applying our method to several practical examples from measurement, forecasting and nowcasting. We conclude our paper with the last section.