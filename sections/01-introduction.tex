\unsure[inline]{Der Beginn ist jetzt sehr auf Measurement fixiert. Könnte man das vielleicht umdrehen: Analysen von Measurement/Nowcast/Forecast sind im Moment vor allem bezüglich Abstand (RMSE; MAE; aber auch probabilistische Analyse). Wichtige Info ist aber auch auch, wie gut ein Anstieg/Fall vorhergesagt wird. Dann drauf eingehen, wie das bisher in den Bereichen gemacht wird.}

The introduction of new medical measurement methods requires comprehensive evaluation and comparison with established standard measurement methods in order to ensure their validity and reliability. 
To detect the systematic difference between one measurement methods with a gold standard method, a paired t-test can be performed on the pairwise difference of the two series of measurements to test the null hypothesis that the true mean difference is zero (\cite{watson2010method}).
In addition, various indices such as the interclass correlation coefficient or Lin's concordance correlation, which were originally introduced as a modification of the Pearson correlation, have been proposed in practice to check the reproducibility of the measurement or to compare different measurement methods (\cite{lawrence1989concordance,koo2016guideline,}). 
Furthermore, graphical tools were developed, such as the Bland-Altman diagram, which visualizes the differences between the methods in relation to their mean value (\cite{bland1986statistical}).

One important aspect of testing the consistency of the new method with the gold standard is to adequately validate if the new method is measuring changes in the target variable in the right direction, when the gold standard detects changes during the measurement process. 
Ideally, the signs of the changes between two consecutive points in time (with a defined time interval) measured by the new method and the gold standard method should always be the same. 
In this case, we call the new method trending perfect in line with the golden Stadnard method and refer to this type of agreement as trendability. 
This type of consideration of the trend capability of a measurement method to the gold standard has also recently gained importance in the literature. 
In (\cite{hiraishi2021concordance}), the authors proposed a new concordance rate for the four-quadrant diagram, which is based on the multivariate normal distribution and takes into account the correlation between the data of the individual subjects.

The above-described idea of trending is also of fundamental interest for the evaluation and comparison of forecasting and nowcasting methods. 
While measurement series are created by recording data using measuring instruments or direct observation, forecasting focuses on predicting the future based on historical data and its patterns. 
When forecasting at a specific point in time (forecast origin), the forecast value at the target time corresponds to a prediction of the change in the target variable between the forecast origin and the target time. 
This predicted change results from the difference between the forecast value at the forecast horizon and the current, actual value of the target variable at the forecast origin. 
A forecast is perfectly consistent with the actual development of the target value if the actual change in the target value over this period matches the forecast change. 
In practice, a forecasting method is usually evaluated in terms of its prediction accuracy, which is defined by a certain loss function and measures the difference between the prediction and the actual outcome of the target variable; an overview of the different measures can be found in \cite{hyndman2006another}. 
Prediction accuracy can be measured by scale-dependent measures such as the Root Mean Square Error (RMSE) or measures based on percentage errors such as the Root Mean Square Percentage Error (RMSPE). 
All of these typical measures are based on the absolute difference between the prediction and the true values, locally or globally, and are not capable of assessing the trendability discussed above without further adjustment.

Another area in which the detection of trends can be of interest is nowcasting.
In contrast to conventional forecasting methods, nowcasting focuses on predicting the current status.
Nowcasting has its origins in meteorology, and the methods were originally developed to describe the current state of the weather in detail and to predict the expected change on a time scale of a few hours (\cite{browning1989nowcasting,schmid2019nowcasting}).
Nowcasting methods are now widely used in other fields, such as economics and medicine, and focus on predictions for the present, the immediate future, and the recent past (\cite{banbura2013now, bok2018macroeconomic}).
Nowcasting can be used, for example, to predict important statistics such as the current economic situation (e.g. GDP), which are collected with low frequency and are available with a considerable time delay (\cite{banbura2013now}).
Nowcasting methods use indicators that are related to the target variable but can be collected at a higher frequency and published in a timely manner to produce and update early estimates for the target variable (e.g. GDP from the first quarter of 2024 ) during the relevant period (i.e. from January to March 2024) (\cite{castle2017forecasting}). 
In medicine, epidemic nowcasting is used during an ongoing epidemic to assess the current situation, taking into account the main pathogenic, epidemiological, clinical, and socio-behavioral factors (\cite{wu2021nowcasting}). 
The nowcasting method can be used, for example, to correct the daily Covid case numbers for events that have occurred but have not yet been reported before this data has been fully recorded at a central location (\cite{gunther2021nowcasting}). 
In this context, nowcasting is an estimation procedure in which the value of a target variable (e.g. from today) is estimated on the basis of current preliminary measurements, which is only finalised at a later point in time. 
At the time of the forecast, estimates of the target variable can be made for today and possibly for yesterday (if the value for yesterday has not yet been finalised); the predicted change in the target variable is then the difference between the two nowcast values. 
A nowcast is consistent with the development of the target variable if the difference in the actual value between today and yesterday, which is announced at a later date, has the same sign as the predicted change. 
It is obvious that a nowcast with strong trend capability provides crucial information on the development of the target variable (e.g. the number of cases of an epidemic) and can be a solid basis for decisions on the introduction or cancellation of policy measures. 
Unfortunately, like forecasts, nowcasts are often evaluated and compared in the literature on the basis of performance measures such as the root mean absolute error (\cite{gunther2021nowcasting}) and the aspect of trending ability is hardly considered in the literature. 

As outlined above, the ability to take into account the trendability of a method depending on the area of application, be it measurements, predictions and forecasts, is of high importance. However, regardless of the application domain, most conventional method comparisons and evaluations consider local absolute differences in different ways without directly assessing whether the correct direction of change is output. In this paper, we aim to formalize trending and present various methods for measuring and visualizing trending that can be applied to the evaluation of trending ability of methods in measurements, nowcasts and forecasts. The presented methods will be applied to a variety of real data examples. 

The paper is structured as follows. In section \ref{sec:trending} we formalize the concept of trending independently of the application domain and investigate quantification approaches for noise-free and noisy data. In particular, we introduce a new graphical method for evaluating trending, the conditional trending plot, which can represent the simultaneous evaluation of different intervals and asymmetries. Bootstrap methods for calculating confidence intervals for trending measures are also presented. In Section \ref{sec:application}, we show the results of applying our method to several practical examples from measurement, forecasting and nowcasting. We conclude our paper with the last section. The source code for the replication of the results of our study as well as a ready-to-use code for trendability detection are available in our public repository at ... available.