
The trending detection of time series, as motivated in the introduction, is fundamentally interesting for evaluating and comparing methods from three areas: measurement technology, forecasting, and nowcasting. In the following, the characteristics of these three areas and the respective evaluation and comparison approaches are presented one after the other.

Measurement aims to obtain accurate and reliable data about the current state of a system. Often a certain parameter or variable is measured at regular intervals over a certain period of time.
An important question is how to compare one method of measurement of a variable with another.
For this purpose, measured values can be recorded with the two measurement methods to be compared. One can be the gold standard, we note the measured values from the gold standard method with $(y_t)^T_{t=0}$) for the same variable in the same period of time. 
The measured values for a measurement method are then available in a time series form, can be noted with $(x_t)^T_{t=0}$. 
To detect the systematic difference between the two methods, a paired t-test can be performed on the pairwise difference of the two series of measurements to test the null hypothesis that the true mean difference is zero (\cite{watson2010method}).
In addition, various indices such as the interclass correlation coefficient or Lin's concordance correlation, which were originally introduced as a modification of the Pearson correlation, have been proposed in practice to check the reproducibility of the measurement or to compare different measurement methods (\cite{lawrence1989concordance,koo2016guideline,}). 
Furthermore, graphical tools were developed, such as the Bland-Altman diagram, which visualizes the differences between the methods in relation to their mean value (\cite{bland1986statistical}).

While measurement series are created by recording data using measuring instruments or direct observation, forecasting focuses on predicting the future based on historical data and its patterns. We note in the following the real realizations of the target value with $(y_t)^T_{t=0}$ and the prediction data for a target variable can be organized in such a way that $x_{\tau|t}$ the prediction of one predictor for time $\tau$ at time $t$ provided that $\tau$ is greater than $t$. The forecast is for the future point in time $\tau$ and is based on the information available at time $t$. 
In practice, a forecasting method is typically evaluated by forecast accuracy, which is defined by a certain loss function and evaluates the difference between the forecast and the actual outcome of the target variable.
A comparison of several forecasters can then be made, for example, by ranking the forecast accuracies of the forecasters.
In practice, prediction accuracy is measured by scale-dependent measures such as the Root Mean Square Error (RMSE) or measures based on percentage errors such as the Root Mean Square Percentage Error (RMSPE). All of these typical measures are based on the absolute difference between forecast and true values, locally or globally. For an overview of different measures, see \cite{hyndman2006another}. 
In contrast to conventional forecasting methods, nowcasting focuses on predicting the current status and is therefore sometimes also referred to as short-term forecasting. Nowcasting has its origins in meteorology, and the methods were originally developed to describe the current state of the weather in detail and to predict the expected change on a time scale of a few hours (\cite{browning1989nowcasting,schmid2019nowcasting}). One difference to the conventional forecast is that nowcasts can also use latent variables, which are often published with higher frequency in the current period, to forecast the target figure for the current point in time (\cite{castle2017forecasting}). 
Nowcasting focuses on predictions for the present, the immediate future, and the recent past, and is now being used in other fields such as economics and medicine. For example, Nowcasting can be used to predict important statistics about the current economic situation that is only available with a considerable time lag (\cite{banbura2013now, giannone2006nowcasting,fornaro2020nowcasting,bok2018macroeconomic}). In medicine, epidemic nowcasting is used during an ongoing epidemic to assess the current situation, taking into account the main pathogenic, epidemiological, clinical, and socio-behavioral factors (\cite{wu2021nowcasting}). For possible applications in this area, see \cite{johansson2014nowcasting,gunther2021nowcasting,birrell2021real}.
When predicting values in the present, it is usual that the true values of the recent past are not available, so only an estimate of these values is provided by the nowcast. For example, the current number of cases of an infectious disease may not be immediately available due to delays in data collection or reporting, or the situation may change rapidly so that the reported figures become inaccurate very quickly.
Using the notation for forecasting above, this means that a prediction value $x_{\tau|t}$ also exits for $\tau$ less than or equal to $t$.
Similar to forecasting, nowcasts are often evaluated and compared in the literature on the basis of performance measures such as the root mean absolute error (\cite{gunther2021nowcasting}).

All the measures discussed, regardless of the area, have in common that the comparison and evaluation of the methods take into account the local absolute differences in different ways but do not directly consider whether the correct direction of change is issued.
Because of the different ways in which data are generated and processed depending on the area of application, the data collected should be prepared in different ways for further processing in order to identify the trending ability (see Section~\ref{sec:trending}). For this purpose, a time series must be defined for a given time series (of measurements or predictions) that reflects the recording or prediction of changes (with a time delay $l$). For a series of measurements, the change between two discrete points in time is considered
\begin{equation}
    \diffxl = (x_{t+l} -x_{t})^T_{t=0}. \label{eq:diffxl_measure}
\end{equation}
For forecasts and nowcasts, we use $(y_t)^T_{t=0}$ to denote the true values of the quantity of interest. For forecasts, the current value is known at the time of the forecast, so a forecast for the time $t+l$ means the prediction of a change $x_{t+l|t} -y_{t}$ between $t$ and $t+l$. The time series of change can therefore be defined as 
\begin{equation}
    \diffxl = (x_{t+l|t} -y_{t})^T_{t=0}.\label{eq:diffxl_forecasting}
\end{equation}
For nowcasting, the value of the current point in time does not have to be known at the prediction time. Often, only an estimate of the current value is known. This can be taken into account by the following case distinction
\begin{equation}
\diffxl = 
\begin{cases} 
(x_{t+l|t} -x_{t|t})^T_{t=0} & \text{if } y_{t} \text{ is not known at time } t, \\
(x_{t+l|t} -y_{t})^T_{t=0}  & \text{else}.
\end{cases} \label{eq:diffxl_nowcasting}
\end{equation}



%To account for this, we note the nowcast value at time $t$ of the quantity of interest of time $\tau$ as $x^{t}_{\tau}$ (where $\tau$ can be less than $t$) 

%Data preparation is similar for measurements and predictions. Let $\mathbf{x} = (x_t)^T_{t=1}$ and $\mathbf{y}=(y_t)^T_{t=1}$ be two time series to be compared (e.g. $\mathbf{x}$, $\mathbf{y}$ are two measurements with two different measuring devices/ measurement methods at the same points in time or predicted values of two predictors in the same time period). 



%For the following analysis, the changes between points in time with the same time distance $l\in \mathbb{N}^+ $ in the time series of $\mathbf{x}$ are considered (the same applies for $\mathbf{y}$):

%\[\Delta^lx_t = x_{t+l} - x_{t} \text{ for } t+l\leq T. \]\\
%, then the serie of changes $\mathfrak{X}$, which is later used to assess trendability, is defined as:

%\[\Delta^l \mathfrak{X} _t = x^{t}_{t+l} - x^t_{t} \text{ for } t+l\leq T.\]
%If the true value $y_t$ were known at time $t$, then the following applies: $x^t_t = y_t$.

%\begin{itemize}
%    \item Was ist nowcasting?
%    \item Aus welchem Anwendungsbereichen kommt es? 
%    \item In welchen Bereichen in der Medizin spielt es eine Rolle?
%    \item Warum ist Nowcasting anders als Forecasting/Measurement?
%    \item Wo spielt Trending in der Literatur schon eine Rolle?
%\end{itemize}

