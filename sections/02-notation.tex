
The trending detection of time series, as motivated in the introduction, is fundamentally interesting for evaluating and comparing methods from three areas: measurement technology, forecasting, and nowcasting. In the following, the characteristics of these three areas and the respective evaluation and comparison approaches are presented one after the other.
This paper relates to numerical data. \todo{Hängt irgendwie in der Luft} \unsure{Müsste es nicht auch ausschließlich um wiederholte Messungen gehen?}

Measurement aims to obtain accurate and reliable data about the current state of a system.
Often, the question is how to compare one method of measurement of a variable with another.
For this purpose, measured values can be recorded with the two measurement methods to be compared (one can be the gold standard, we note the measured values from the gold standard method with $(y_t)^T_{t=0}$) for the same variable in the same period of time.
The measured values for a measurement method are then available in a time series form, e.g., can be noted with $(x_t)^T_{t=0}$.
To detect the systematic difference between the two methods, a paired t-test can be performed on the pairwise difference of the two series of measurements to test the null hypothesis that the true mean difference is zero (\cite{watson2010method}).
In addition, various indices such as the interclass correlation coefficient or Lin's concordance correlation, which were originally introduced as a modification of the Pearson correlation, have been proposed in practice to check the reproducibility of the measurement or to compare different measurement methods (\cite{lawrence1989concordance,koo2016guideline,}). 
Furthermore, graphical tools were developed, such as the Bland-Altman diagram, which visualizes the differences between the methods in relation to their mean value (\cite{bland1986statistical}).

While measurement series are created by recording data using measuring instruments or direct observation, forecasting focuses on predicting the future based on historical data and its patterns. We note in the following the real realizations of the target value with $(\mathbf{y})^T_{t=0}$ and the prediction data for a target variable can be organized in such a way that $x_{t,\tau}$ the prediction of one predictor for time $\tau$ at time $t$ provided that $\tau$ is greater than $t$. \todo{Motivieren, warum beide Indizes nötig}
In practice, a forecasting method is typically evaluated by forecast accuracy, which is defined by a certain loss function and evaluates the difference between the forecast and the actual outcome of the target variable.
A comparison of several forecasters can then be made, for example, by ranking the forecast accuracies of the forecasters.
In practice, prediction accuracy is measured by scale-dependent measures such as the Root Mean Square Error (RMSE) or measures based on percentage errors such as the Root Mean Square Percentage Error (RMSPE). All of these typical measures are based on the absolute difference between forecast and true values, locally or globally. For an overview of different measures see \cite{hyndman2006another}. Compared to conventional forecasting methods, nowcasting focuses on predicting the current status and is therefore also referred to as short-term forecasting. Nowcasting  has its origins in meteorology, and the methods were originally developed to describe the current state of the weather in detail and to predict the expected change on a time scale of a few hours (\cite{browning1989nowcasting,schmid2019nowcasting}).
Nowcasting focuses on predictions for the present, the immediate future, and the recent past and is now being used in other fields, such as the economics, where important statistics on the current economic situation are available only after a considerable time lag (\cite{banbura2013now}), and in medicine, where epidemic nowcasting is used during an ongoing epidemic outbreak to assess the current situation, taking into account the main pathogenic, epidemiological, clinical and socio-behavioral factors (\cite{wu2021nowcasting}). \todo{sehr langer Satz}
Applications of nowcasting methods for econometrics can be found here in \cite{giannone2006nowcasting,fornaro2020nowcasting,bok2018macroeconomic} or for predictions of epidemic development here in \cite{johansson2014nowcasting,gunther2021nowcasting,birrell2021real}. \todo{Satz klingt noch nicht so ganz rund}
When predicting values in the present, it is usual that the true values of the recent past are not available, so only an estimate of these values is provided by the nowcast.\todo{Vielleicht noch etwas konkreter machen}
Using the notation for forecasting above, this means that a prediction value $x_{t,\tau}$ also exiertits \todo{word?} for $\tau$ less than or equal to $t$.
Similar to forecasting, nowcasters are often evaluated and compared in the literature on the basis of performance measures such as the root mean absolute error (\cite{gunther2021nowcasting}).

All the measures discussed, regardless of the area, have in common that the comparison and evaluation of the methods take into account the local absolute differences in different ways but do not directly consider whether the correct direction of change is issued
To our best knowledge, there is no work in the literature that addresses the trending capability of measurement series or predictions with continuous outcomes.
Because of the different ways in which data are generated and processed depending on the area of application, the data collected should be prepared in different ways for further processing in order to identify the trending ability (see Section~\ref{sec:trending}). \todo{Vielleicht nochmal Praxisbezug herausstellen, warum es sinnvoll ist, die Differenzen so zu formulieren?}
In particular, the following shows how the time series of change (with lag $l$) should be defined, depending on whether measurement, forecasting, or nowcasting methods are considered. For measurement:
\begin{equation}
    \diffxl = (x_{t+l} -x_{t})^T_{t=0} \label{eq:diffxl_measure}
\end{equation}
For forecasting and nowcasting bezeichnen wir mit y die wahren werte der quantity of interesst. For Forcasting ist der Werzum Zeitpu
\begin{equation}
    \diffxl = (x_{t,t+l} -y_{t})^T_{t=0} \label{eq:diffxl_forecasting}
\end{equation}
For nowcasting:
\begin{equation}
\diffxl = 
\begin{cases} 
(x_{t,t+l} -x_{t,t})^T_{t=0} & \text{if } y_{t} \text{ is not known at time } t, \\
(x_{t,t+l} -y_{t})^T_{t=0}  & \text{else}.
\end{cases} \label{eq:diffxl_nowcasting}
\end{equation}

\todo[inline]{Formeln als Tabelle oder in den Text einbinden.}
\todo[inline]{$\mathbf{y}$ definieren.}

%To account for this, we note the nowcast value at time $t$ of the quantity of interest of time $\tau$ as $x^{t}_{\tau}$ (where $\tau$ can be less than $t$) 

%Data preparation is similar for measurements and predictions. Let $\mathbf{x} = (x_t)^T_{t=1}$ and $\mathbf{y}=(y_t)^T_{t=1}$ be two time series to be compared (e.g. $\mathbf{x}$, $\mathbf{y}$ are two measurements with two different measuring devices/ measurement methods at the same points in time or predicted values of two predictors in the same time period). 



%For the following analysis, the changes between points in time with the same time distance $l\in \mathbb{N}^+ $ in the time series of $\mathbf{x}$ are considered (the same applies for $\mathbf{y}$):

%\[\Delta^lx_t = x_{t+l} - x_{t} \text{ for } t+l\leq T. \]\\
%, then the serie of changes $\mathfrak{X}$, which is later used to assess trendability, is defined as:

%\[\Delta^l \mathfrak{X} _t = x^{t}_{t+l} - x^t_{t} \text{ for } t+l\leq T.\]
%If the true value $y_t$ were known at time $t$, then the following applies: $x^t_t = y_t$.

%\begin{itemize}
%    \item Was ist nowcasting?
%    \item Aus welchem Anwendungsbereichen kommt es? 
%    \item In welchen Bereichen in der Medizin spielt es eine Rolle?
%    \item Warum ist Nowcasting anders als Forecasting/Measurement?
%    \item Wo spielt Trending in der Literatur schon eine Rolle?
%\end{itemize}

