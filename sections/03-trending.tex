
\begin{itemize}
    \item Notation kurz wiederholen
    \item Weiterer Aufbau Kapitel klären: 
\end{itemize}

\subsection{Basics of trending and 4Q plots}

\begin{itemize}
    \item Trending mathematisch $P(x y) > 0$ / Statistical consistency of $sgn(x)$ mit $sgn(y)$
    \item 4Q plots
\end{itemize}

\subsection{Trending accuracy and other measures}

\begin{itemize}
    \item Existierende Ansätze aus Literatur: Accuracy, Agreementmaße, Kreuztabellen, Dichotome Vorhersagen
    \item Trending als rolling estimate über die Zeit
    \item Was ist mit missing values?
\end{itemize}

\subsection{Accounting for noise}

\begin{itemize}
    \item Kleine Beobachtungen interessieren uns weniger als große; verrauschen vor allem Ergebnis
    \item Daten mit Rauschen: Man kann kleine Beobachtungen (zu klären, was das genau heißt) ausschließen: man schätzt $P(XY > 0 | \abs (X) > \epsilon$ $\rightarrow$ kleiner Exkurs, warum Gewichtung schwierig zu interpretieren ist?
    
\end{itemize}

\subsection{Conditional trending plot}

\begin{itemize}
    \item Grafische Evaluation von $P(XY > 0 | X)$ mit bedingter KDE
    \begin{itemize}
        \item Vorteil: Im Gegensatz zum 4Q-Plot kann man mehrere Nowcasts in einem Schaubild vergleichen
        \item Vorteil: Ermöglicht Analyse von beliebigen Bereichen separat, im Gegensatz zu festgeschriebenen Maßen
        \item Vorteil: man macht Asymmetrien und lokales Verhalten sichtbar 
        \item Begründen, warum andere "klassische" Maße nicht funktionieren
    \end{itemize}
\end{itemize}


Zu diskutieren:
\begin{itemize}
    \item Konfidenzintervalle
    \item Weitere Maße aus dem Kreuztabellen/dichotomes Forecasting: sensitivity, specitivity, ...
\end{itemize}

\subsection{Beispiele (simple, illustrative Beispiele)}