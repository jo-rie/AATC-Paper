
In this paper, we examine various methods to analyze trending, the statistical consistency of observed and predicted change direction.
While the computation of predicted change varies between the application areas measurement (see Section~\ref{subsec:notation}), nowcasting and forecasting, the assessment of trending using the computed predicted and observed change is similar.
The evaluation of trending can accompany other evaluation techniques such as measures of deviation or probabilistic scoring rules.

Four-quadrant plots facilitate the visual inspection of the trending ability for a signal (see Section~\ref{subsec:trending-four-quadrant-plot}).
The trending ratio, that is, the ratio of change directions predicted correctly over all changes, numerically evaluates trending.
Visually, it is the ratio of concordant points over all points in a four-quadrant plot (see Section~\ref{subsec:trending-measures}).
The positive and negative trending ratio analyze the trending ratio given the predicted change is positive or negative, respectively.
Thus, they quantify the credibility of the respective predictions. 
The applications of Section~\ref{sec:application_measurement} show, that models, in general, indeed have different positive and negative trending abilities and that they add valuable information to the trending ratio.
In the applications, the bootstrap condfidence intervals of Section~\ref{subsec:trending-measures} are used to quantify the estimation uncertainty of the trending measures.
The width of the confidence intervals is around 0.1 for around 100 samples, while it is around 0.01 for 8000 samples.
For reasonably well trending models, 100 samples are thus sufficient to differentiate from random guessing or models with high trending differences, 1000 samples enable a more precise differentiation.

A conditional trending plot visualizes the probability of correct trending over the predicted change of the signal (see Section\ref{subsec:trending-cond-prob}).
It is based on a multivariate \acf{kde} of predicted and observed change.
In the application, it gives reasonable insights into local effects of the trending ability.
Section~\ref{subsec:trending-probabilistic} adapts measures of probabilistic forecast evaluation to the trending evaluation of probabilistic forecasts and nowcasts.
The \acf{bs} as numerical assessment of probabilistic trending is introduced and reliability diagrams are used to visualize the local trending ability of probabilistic forecasts.

The methods of trending assessment are applied to COVID-19-nowcasting, emergency department arrival forecasting, and invasive and non-invasive blood pressure measurements in Section~\ref{sec:application}.
While trending evaluation is not the only aspect of evaluation, it is a valuable addition to the evaluation of nowcasts, forecasts, and measurements.
For models with very different accuracies, these are usually replicated in trending evaluation, but the trending evaluation can differentiate between models with similar accuracies.
As in the application in Section~\ref{sec:application-covid}, models with average point forecast evaluation measures can have the most meaningful positive trending ability.

Throughout this paper, we did not expand on two modeling aspects, which we leave for further research.
In the estimation, we did not consider sequential correlation.
The computation of differences is a standard procedure in time series analysis to remove sequential dependence, but, in general, the trending measures could be adapted to consider sequential correlation.
Similarly, the bootstrap confidence intervals could be adapted to consider sequential correlation by using time series bootstrap methods~\parencite{Hardle2003,Kreiss2012}.

The estimators of Section~\ref{subsec:trending-measures} could encourage predictors to exploit imbalances in the number of observed positive and negative changes.
A large difference in the number of observed positive and negative changes is unlikely in the trending setting, as $\diffy$ is obtained from differencing time series data.
In addition, we do not advise assessing signals solely on their trending ability and, thus, provide an incentive for the exploitation of estimator characteristics.
However, if the number of positive and negative observed changes differs widely, the use of unbalanced-data-aware measures should be considered.
There are various adapted measures for unbalanced outcomes, for example, Cohen's $\kappa$ \parencite{Cohen1960} or those listed in \textcite[Table 3.3]{Jolliffe2012}.
