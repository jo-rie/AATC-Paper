\section{Additional material on Section~\ref{sec:application-covid}}\label{sec:appendix-application-covid}


%\begin{figure}
%    \centering
%    \begin{subfigure}[t]{.48\textwidth}
%        \includegraphics{plots/covid_nowcast/20_kde_lag_1}
%        \caption{\ac{kde} for true values and nowcasts of lag 1.}
%    \end{subfigure}\hspace{0.01\textwidth}
%    \begin{subfigure}[t]{.48\textwidth}
%        \includegraphics{plots/covid_nowcast/20_kde_lag_7}
%        \caption{\ac{kde} for true values and nowcasts of lag 7.}
%    \end{subfigure}\hspace{0.01\textwidth}
%    \begin{subfigure}[t]{.48\textwidth}
%        \includegraphics{plots/covid_nowcast/20_kde_lag_14}
%        \caption{\ac{kde} for true values and nowcasts of lag 14.}
%    \end{subfigure}
%    \caption{\Ac{kde} for true values and nowcasts of lags 1, 7, and 14 to assess the distribution of values. \hl{AUSWERTUNG}. Exclusion areas based on the 10\%  quantile of absolute values are listed in Table~\ref{tab:app-covid-marginals}.}
%    \label{fig:app-covid-kde}
%\end{figure}


\begin{table}
    \centering
    \begin{tabular}{l l}
        \toprule
        Abbreviation & Nowcasting hub key \\
        \midrule
        EPI & Epiforecasts-independent \\
        ILM & ILM-prop \\
        KIT & KIT-simple\_nowcast \\
        LMU & LMU\_StaBLab-GAM\_nowcast \\
        RIVM & RIVM-KEW \\
        RKI & RKI-weekly\_report \\
        SU & SU-hier\_bayes \\
        SZ & SZ-hosp\_nowcast\\
        ENS-MEAN & NowcastHub-MeanEnsemble\\
        ENS-MED & NowcastHub-MedianEnsemble\\
        \bottomrule
    \end{tabular}
    \caption{Matching the abbreviation to the key in the nowcasting hub.
    Information on the models and references is listed in \citet[][Table 1]{Wolffram2023}.}
    \label{tab:app-covid-models}
\end{table}


\begin{table}
    \centering
    \tiny
    \begin{tabular}{llllllllll}
\toprule
 & $\widebar{x^{\Delta, 1}}$ & $\sigma_{x^{\Delta, 1}}$ & $q_{0.1} (x^{\Delta, 1})$ & $\widebar{x^{\Delta, 7}}$ & $\sigma_{x^{\Delta, 7}}$ & $q_{0.1} (x^{\Delta, 7})$ & $\widebar{x^{\Delta, 14}}$ & $\sigma_{x^{\Delta, 14}}$ & $q_{0.1} (x^{\Delta, 14})$ \\
\midrule
EPI & 72 & 520 & 45 & 37 & 1,411 & 78 & -62 & 1,976 & 145 \\
ILM & 40 & 281 & 26 & 144 & 1,457 & 103 & 147 & 2,357 & 140 \\
KIT & 24 & 355 & 50 & 112 & 1,306 & 171 & 92 & 1,965 & 265 \\
LMU & -48 & 285 & 27 & -6 & 1,180 & 124 & -109 & 1,947 & 168 \\
ENS-MEAN & 21 & 267 & 23 & 56 & 1,214 & 98 & -3 & 1,956 & 235 \\
ENS-MED & 20 & 259 & 24 & 28 & 1,207 & 101 & -52 & 1,955 & 186 \\
RIVM & -8 & 242 & 32 & -50 & 1,264 & 123 & -104 & 2,034 & 191 \\
RKI & 109 & 363 & 34 & 367 & 1,194 & 146 & 419 & 1,833 & 326 \\
SU & 43 & 376 & 47 & 23 & 1,391 & 181 & -68 & 2,127 & 264 \\
SZ & 24 & 201 & 27 & 123 & 1,155 & 185 & 105 & 1,889 & 242 \\
True & -21 & 263 & 27 & -86 & 1,238 & 127 & -109 & 2,194 & 284 \\
\bottomrule
\end{tabular}

    \caption{Analysis of the nowcast and true differences for the lags 1, 7, and 14 days.
    The column (1), $l=l$ shows the number of values greater than zero for lag $l$, $\sigma_{x^{\Delta, l}}$ the standard deviation, and $q_{0.1} (x^{\Delta, l})$ the 10\% quantile of the differences' absolute values.}
    \label{tab:app-covid-marginals}
\end{table}

\begin{table}
    \centering
    \begin{subtable}[t]{\textwidth}
        \begin{tabular}{l p{0.11\textwidth} p{0.11\textwidth} p{0.11\textwidth} p{0.11\textwidth} p{0.11\textwidth} p{0.11\textwidth}}
\toprule
 & $\mu^1$ & $\mu^{+, 1}$ & $\mu^{-, 1}$ & $\mu^1_{q_{0.1}}$ & $\mu^{+, 1}_{q_{0.1}}$ & $\mu^{-, 1}_{q_{0.1}}$ \\
\midrule
EPI & {0.68\newline(0.62, 0.74)} & {0.64\newline(0.55, 0.72)} & {0.73\newline(0.63, 0.81)} & {0.69\newline(0.63, 0.75)} & {0.64\newline(0.55, 0.73)} & {0.75\newline(0.65, 0.82)} \\
ILM & {0.73\newline(0.67, 0.79)} & {0.67\newline(0.59, 0.76)} & {0.82\newline(0.73, 0.88)} & {0.74\newline(0.68, 0.79)} & {0.68\newline(0.60, 0.77)} & {0.82\newline(0.72, 0.88)} \\
KIT & {0.62\newline(0.55, 0.68)} & {0.58\newline(0.49, 0.67)} & {0.65\newline(0.56, 0.73)} & {0.62\newline(0.56, 0.69)} & {0.59\newline(0.51, 0.67)} & {0.66\newline(0.57, 0.74)} \\
LMU & {0.66\newline(0.60, 0.72)} & {0.66\newline(0.57, 0.75)} & {0.66\newline(0.57, 0.73)} & {0.66\newline(0.59, 0.71)} & {0.66\newline(0.56, 0.75)} & {0.66\newline(0.57, 0.73)} \\
ENS-MEAN & {0.81\newline(0.75, 0.86)} & {0.76\newline(0.68, 0.84)} & {0.88\newline(0.81, 0.93)} & {0.81\newline(0.75, 0.86)} & {0.76\newline(0.68, 0.83)} & {0.88\newline(0.81, 0.94)} \\
ENS-MED & {0.75\newline(0.68, 0.80)} & {0.69\newline(0.61, 0.77)} & {0.81\newline(0.73, 0.89)} & {0.75\newline(0.69, 0.81)} & {0.69\newline(0.60, 0.77)} & {0.83\newline(0.74, 0.90)} \\
RIVM & {0.77\newline(0.72, 0.82)} & {0.75\newline(0.66, 0.83)} & {0.79\newline(0.71, 0.85)} & {0.78\newline(0.72, 0.83)} & {0.75\newline(0.66, 0.83)} & {0.81\newline(0.73, 0.87)} \\
RKI & {0.74\newline(0.68, 0.79)} & {0.67\newline(0.59, 0.74)} & {0.88\newline(0.79, 0.93)} & {0.74\newline(0.68, 0.79)} & {0.66\newline(0.58, 0.73)} & {0.87\newline(0.78, 0.93)} \\
SU & {0.71\newline(0.65, 0.77)} & {0.66\newline(0.57, 0.74)} & {0.78\newline(0.69, 0.85)} & {0.72\newline(0.66, 0.78)} & {0.67\newline(0.58, 0.75)} & {0.79\newline(0.70, 0.87)} \\
SZ & {0.74\newline(0.68, 0.80)} & {0.68\newline(0.60, 0.76)} & {0.82\newline(0.73, 0.90)} & {0.74\newline(0.68, 0.80)} & {0.68\newline(0.60, 0.76)} & {0.82\newline(0.73, 0.90)} \\
\bottomrule
\end{tabular}

    \caption{1 day.}
    \end{subtable}
    \begin{subtable}[t]{\textwidth}
        \begin{tabular}{l p{0.11\textwidth} p{0.11\textwidth} p{0.11\textwidth} p{0.11\textwidth} p{0.11\textwidth} p{0.11\textwidth}}
\toprule
 & $\mu^14$ & $\mu^{+, 14}$ & $\mu^{-, 14}$ & $\mu^14_{q_{0.1}}$ & $\mu^{+, 14}_{q_{0.1}}$ & $\mu^{-, 14}_{q_{0.1}}$ \\
\midrule
EPI & {0.83\newline(0.78, 0.87)} & {0.79\newline(0.70, 0.85)} & {0.87\newline(0.80, 0.92)} & {0.85\newline(0.80, 0.89)} & {0.81\newline(0.73, 0.87)} & {0.90\newline(0.83, 0.95)} \\
ILM & {0.86\newline(0.81, 0.90)} & {0.78\newline(0.70, 0.85)} & {0.96\newline(0.90, 0.99)} & {0.87\newline(0.82, 0.91)} & {0.80\newline(0.71, 0.86)} & {0.96\newline(0.90, 0.99)} \\
KIT & {0.81\newline(0.75, 0.86)} & {0.76\newline(0.67, 0.83)} & {0.87\newline(0.79, 0.92)} & {0.82\newline(0.76, 0.86)} & {0.76\newline(0.68, 0.84)} & {0.88\newline(0.81, 0.93)} \\
LMU & {0.88\newline(0.83, 0.92)} & {0.85\newline(0.77, 0.91)} & {0.91\newline(0.85, 0.95)} & {0.89\newline(0.85, 0.93)} & {0.87\newline(0.79, 0.92)} & {0.91\newline(0.85, 0.95)} \\
ENS-MEAN & {0.83\newline(0.77, 0.87)} & {0.77\newline(0.69, 0.84)} & {0.89\newline(0.83, 0.95)} & {0.84\newline(0.79, 0.88)} & {0.78\newline(0.70, 0.84)} & {0.91\newline(0.84, 0.95)} \\
ENS-MED & {0.84\newline(0.79, 0.89)} & {0.79\newline(0.70, 0.86)} & {0.90\newline(0.83, 0.95)} & {0.85\newline(0.80, 0.90)} & {0.80\newline(0.71, 0.86)} & {0.91\newline(0.84, 0.95)} \\
RIVM & {0.85\newline(0.80, 0.89)} & {0.82\newline(0.74, 0.88)} & {0.88\newline(0.80, 0.93)} & {0.85\newline(0.80, 0.90)} & {0.83\newline(0.75, 0.89)} & {0.88\newline(0.80, 0.93)} \\
RKI & {0.81\newline(0.75, 0.86)} & {0.71\newline(0.63, 0.77)} & {0.98\newline(0.93, 1.00)} & {0.81\newline(0.75, 0.86)} & {0.71\newline(0.63, 0.78)} & {1.00\newline(nan, nan)} \\
SU & {0.88\newline(0.83, 0.92)} & {0.84\newline(0.76, 0.90)} & {0.92\newline(0.86, 0.96)} & {0.89\newline(0.85, 0.93)} & {0.85\newline(0.77, 0.91)} & {0.94\newline(0.88, 0.97)} \\
SZ & {0.82\newline(0.77, 0.87)} & {0.76\newline(0.68, 0.83)} & {0.90\newline(0.83, 0.94)} & {0.83\newline(0.78, 0.88)} & {0.78\newline(0.69, 0.85)} & {0.90\newline(0.83, 0.94)} \\
\bottomrule
\end{tabular}

        \caption{14 days.}
    \end{subtable}
    \caption{Trending ratio $\acc$, positive trending ratio $\accp$, and negative trending ratio $\accm$ for the models with and without exclusion areas for the lag 1 and 14 days. The exclusion areas are rectangles centered on the zero points with a width and height of 10\% of the quantile of the absolute values of nowcast and true values. }
    \label{tab:app-covid-trending-ratios-lag-1-14}
\end{table}



\begin{figure}
\centering
\includegraphics{plots/covid_nowcast/30_4q_plots}
\caption{Four-quadrant plots for the models ILM, RIVM, RKI, and ENS-MEAN and the lags of one, seven, and 14 days. The spread in both directions increases with the lag. }
\label{fig:app-covid-4q}
\end{figure}


\begin{figure}
    \centering
%    \includegraphics{}
    \begin{subfigure}[t]{.48\textwidth}
    \includegraphics{plots/covid_nowcast/40_cond_prob_lag_1}
    \caption{Conditional trending plot for lag 1.}\label{fig:app-covid-cond-prob-1}
    \end{subfigure}\hfill
    \begin{subfigure}[t]{.48\textwidth}
    \includegraphics{plots/covid_nowcast/40_cond_prob_lag_14}
    \caption{Conditional trending plot for lag 1.}\label{fig:app-covid-cond-prob-14}
    \end{subfigure}
    \begin{subfigure}[t]{.48\textwidth}
    \includegraphics{plots/covid_nowcast/40_acc_eps_lag_1}
    \caption{Trending ratio over exclusion area size in $\diffx$ for lag 1.}\label{fig:app-covid-trending-ratio-1}
    \end{subfigure}\hfill
    \begin{subfigure}[t]{.48\textwidth}
    \includegraphics{plots/covid_nowcast/40_acc_eps_lag_14}
    \caption{Trending ratio over exclusion area size in $\diffx$ for lag 1.}\label{fig:app-covid-trending-ratio-14}
    \end{subfigure}
    \caption{Conditional trending plot and trending ratio over exclusion area for the nowcasts of the seven-day hospitalization rate ILM, RKI, RIVM, and ENS-MED for the lag seven days.}
    \label{fig:app-covid-cond-prob-trending-ratio-1-14}
\end{figure}


\begin{figure}
    \centering
    \begin{subfigure}[t]{.48\textwidth}
        \includegraphics{plots/covid_nowcast/60_reliability_diagram_lag_7}
        \caption{Reliability diagram for horizon seven days.} \label{fig:app-covid-reliability-7}
    \end{subfigure}\hfill
    \begin{subfigure}[t]{.48\textwidth}
        \includegraphics{plots/covid_nowcast/60_reliability_diagram_lag_14}
        \caption{Reliability diagram for horizon 14 days.} \label{fig:app-covid-reliability-14}
    \end{subfigure}
    \begin{subfigure}{\textwidth}
        \includegraphics{plots/covid_nowcast/70_prob_hist}
        \caption{Count histogram of the predicted probabilities for the horizon one, seven, and 14 days.} \label{fig:app-covid-prob-hist}
    \end{subfigure}
    \caption{The reliability diagram for the models ILM, RIVM, RKI, and ENS-MED for the horizon seven and 14 days.
    Additionally, the count of predicted probabilities for the horizons is shown.
        The reliability diagram bins are chosen according to the empirical quantiles of the predicted probabilities.
    As the models issue small or large probabilities of increase for the higher horizons, little information on the accuracy of moderate probability predictions is available.}
    \label{fig:}
\end{figure}
