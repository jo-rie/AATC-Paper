\documentclass[oneside]{article}

% Language setting
% Replace `english' with e.g. `spanish' to change the document language
\usepackage[english]{babel}
\usepackage[utf8]{inputenc} % für Umlaute
\usepackage{fontenc}

% Set page size and margins
% Replace `letterpaper' with`a4paper' for UK/EU standard size
\usepackage[a4paper,top=2cm,bottom=2cm,left=3cm,right=3cm,marginparwidth=1.75cm]{geometry}

%%%% ---- Useful packages

%%% --- Colors
\usepackage[dvipsnames]{xcolor}

%%% --- Figures and Graphics
\usepackage{graphicx}
\usepackage{subcaption}
\usepackage{tikz}

%%% --- Tabulars
\usepackage{booktabs}
\usepackage{multirow} % cells over several rows
\usepackage{tabularx}
%\usepackage{tabularray}
\usepackage{etoolbox}
\AtBeginEnvironment{tabular}{\small}

%%% --- Mathematics
\usepackage{amsmath,amsfonts,amssymb,amsthm,mathtools, mathabx}
\usepackage{bm} % bold mathematics
\usepackage{bbm} % for blackboard 1
\usepackage[linesnumbered,lined,algo2e,figure,boxed]{algorithm2e}
\usepackage[short]{optidef} % for aligning linear programs

%%% --- Other
\usepackage[colorlinks=true, allcolors=blue]{hyperref}
%\usepackage[hidelinks]{hyperref}
\usepackage[style=authoryear,backend=biber]{biblatex}
\addbibresource{library.bib}
\addbibresource{library-jonas.bib}
\usepackage{csquotes}
\usepackage[shortcuts]{extdash} % For Hyphens in English
%\usepackage{acro}
% \DeclareAcronym{pdf}{short=PDF, long={probability distribution function}}

%%% --- Commenting, Todonotes
\usepackage[colorinlistoftodos,prependcaption]{todonotes}
\usepackage{comment, soul}
\newcommand{\ar}{$\rightarrow$}
\presetkeys{todonotes}{backgroundcolor=yellow!30, bordercolor=yellow!50, linecolor=yellow!50, figwidth=\textwidth}{}
%\newcommand{\hl}[1]{\textcolor{Aquamarine}{#1}}

\usepackage{xargs,ifthen} % Use optional arguements in commands
\newcommandx{\unsure}[2][1=]{\todo[linecolor=blue,backgroundcolor=blue!25,bordercolor=blue,#1]{#2}}
\newcommandx{\discuss}[2][1=]{\todo[linecolor=orange,backgroundcolor=orange!25,bordercolor=orange,#1]{#2}}
\newcommandx{\missing}[2][1=]{\todo[linecolor=red,backgroundcolor=red!25,bordercolor=red,#1]{#2}}
\newcommandx{\change}[2][1=]{\todo[linecolor=red,backgroundcolor=red!25,bordercolor=red,#1]{#2}}
\newcommandx{\info}[2][1=]{\todo[linecolor=OliveGreen,backgroundcolor=OliveGreen!25,bordercolor=OliveGreen,#1]{#2}}
\newcommandx{\improvement}[2][1=]{\todo[linecolor=Plum,backgroundcolor=Plum!25,bordercolor=Plum,#1]{#2}}
\newcommandx{\done}[2][1=]{\todo[linecolor=gray,backgroundcolor=gray!25,bordercolor=gray,#1]{#2}}
\newcommandx{\thiswillnotshow}[2][1=]{\todo[disable,#1]{#2}}
\usepackage{marginnote}
\let\marginpar\marginnote

%%% --- Acronyms
\usepackage{acro}
\DeclareAcronym{kde}{short=KDE, long=kernel density estimation}
\DeclareAcronym{bca}{short=BCa, long=bias-corrected and accelerated}
\DeclareAcronym{cdf}{short=CDF, long=cumulative distribution function}


%%%% ---- Own Commands
%%% --- Theorem Settings
\theoremstyle{plain}% Theorem-like structures provided by amsthm.sty
\newtheorem{theorem}{Theorem}
\newtheorem{exa}{Example}
\newtheorem{rem}{Remark}
\newtheorem{proposition}{Proposition}
\newtheorem{lemma}{Lemma}
\newtheorem{corollary}{Corollary}

\theoremstyle{definition}
\newtheorem{definition}{Definition}
\newtheorem{remark}{Remark}
\newtheorem{example}{Example}

%%% --- Math Commands

\DeclarePairedDelimiter{\abs}\lvert\rvert
\DeclareMathOperator{\sign}{sign}
\newcommand{\tmax}{\bar{t}}
\newcommand{\ind}[1]{\mathbbm{1}\{#1\}}
\newcommand{\Prob}[1]{P(#1)}
\newcommand{\cond}{\:\lvert\:}
\newcommand{\card}[1]{\abs{#1}}

\newcommand{\R}{\mathbb{R}}
\newcommand{\SBer}{\text{SBer}}

\newcommand{\diffxl}{\mathbf{x}^{\Delta,l}}
\newcommand{\diffyl}{\mathbf{y}^{\Delta,l}}
\newcommand{\diffx}{\mathbf{x}^{\Delta}}
\newcommand{\diffy}{\mathbf{y}^{\Delta}}
\newcommand{\diffxrv}{X^{\Delta}}
\newcommand{\diffyrv}{Y^{\Delta}}
\newcommand{\diffxlt}{\mathbf{x}^{\Delta,l}_t}
\newcommand{\diffylt}{\mathbf{y}^{\Delta,l}_t}
\newcommand{\diffxt}{x^{\Delta}_t}
\newcommand{\diffyt}{y^{\Delta}_t}

\newcommand{\acc}{\mu}
\newcommand{\accp}{\acc^+}
\newcommand{\accm}{\acc^-}
%\newcommand{\acceps}[1][]{\ifthenelse { \equal {#1} {} }{\acc_{\eps}}{\acc_{#1}}} %https://stackoverflow.com/a/7314951
\newcommand{\acceps}[1][\varepsilon]{\acc_{#1}} %https://stackoverflow.com/a/7314951
\newcommand{\accpeps}[1][\varepsilon]{\acceps[#1]^+}
\newcommand{\accmeps}[1][\varepsilon]{\acceps[#1]^-}

%%% --- Other Commands
\renewcommand{\arraystretch}{1.2}



%%% --- ToDo List
%\usepackage{enumitem}
%\newlist{todolist}{itemize}{2}
%\setlist[todolist]{label=$\square$}
%\usepackage{pifont}
%\newcommand{\cmark}{\ding{51}}%
%\newcommand{\xmark}{\ding{55}}%
%\newcommand{\done}{\rlap{$\square$}{\raisebox{2pt}{\large\hspace{1pt}\cmark}}%
%\hspace{-2.5pt}}
%\newcommand{\wontfix}{\rlap{$\square$}{\large\hspace{1pt}\xmark}}

\title{Trending assessment for nowcasts, measurements, and forecasts}
\author{Oliver Grothe, Bolin Liu, Jonas Rieger}

\begin{document}
\maketitle

\begin{abstract}
Existing and new measures for the capability of capturing the trend of nowcasts are presented, evaluated, and compared on synthetic and real-world data.
\end{abstract}

\listoftodos


\begin{itemize}
    \item Aufteilung
    \begin{itemize}
        \item Introduction (1.5 pages): Bolin fängt an, Conclusion (0.5 pages), Abstract
        \item Application areas (1 pages): Bolin
        \item Trending (5-6 pages): Jonas fängt an
        \item Application (5-6 pages): Jonas
    \end{itemize}
    \item Datenquellen
\end{itemize}

\begin{itemize}    
    \item Journal/Anwendungsgebiet klären: Angewandtes Medizin Statistikjournal: z.B. BMC Medical Research Methodology
\end{itemize}

\section{Introduction}\label{sec:introduction}

\begin{itemize}
    \item Aktualität/Wichtigkeit herausstellen: 
    \begin{itemize}
        \item zwei Nowcasts mit gleichem MSE, aber unterschiedlichem Trending $\rightarrow$ Vergleich zwischen Nowcasts
    \end{itemize}
    \item Einordnung in Literatur
    \item Contributions
    \begin{itemize}
        \item Argue, why trending is important and why other measures do not \enquote{detect} it
        \item (Formalisation of Trending Ability)
        \item Evaluation and review of existing Approaches (both measures and graphical assessment)
        \item Development of new measure and graphical method to assess trending for measurements, nowcasts and forecasts
        \item Application to various data examples from practice
        \item Publish ready-to-use code
    \end{itemize}
    \item Überblick über Paper
\end{itemize}


\section{Application areas and Notation} \label{sec:notation}

The trending detection of time series, as motivated in the introduction, is fundamentally interesting for evaluating and comparing methods from three areas: measurement technology, forecasting, and nowcasting. In the following, the characteristics of these three areas and the respective evaluation and comparison approaches are presented one after the other.

Measurement aims to obtain accurate and reliable data about the current state of a system. Often a certain parameter or variable is measured at regular intervals over a certain period of time.
An important question is how to compare one method of measurement of a variable with another.
For this purpose, measured values can be recorded with the two measurement methods to be compared. One can be the gold standard, we note the measured values from the gold standard method with $(y_t)^T_{t=0}$) for the same variable in the same period of time. 
The measured values for a measurement method are then available in a time series form, can be noted with $(x_t)^T_{t=0}$. 
To detect the systematic difference between the two methods, a paired t-test can be performed on the pairwise difference of the two series of measurements to test the null hypothesis that the true mean difference is zero (\cite{watson2010method}).
In addition, various indices such as the interclass correlation coefficient or Lin's concordance correlation, which were originally introduced as a modification of the Pearson correlation, have been proposed in practice to check the reproducibility of the measurement or to compare different measurement methods (\cite{lawrence1989concordance,koo2016guideline,}). 
Furthermore, graphical tools were developed, such as the Bland-Altman diagram, which visualizes the differences between the methods in relation to their mean value (\cite{bland1986statistical}).

While measurement series are created by recording data using measuring instruments or direct observation, forecasting focuses on predicting the future based on historical data and its patterns. We note in the following the real realizations of the target value with $(y_t)^T_{t=0}$ and the prediction data for a target variable can be organized in such a way that $x_{\tau|t}$ the prediction of one predictor for time $\tau$ at time $t$ provided that $\tau$ is greater than $t$. The forecast is for the future point in time $\tau$ and is based on the information available at time $t$. 
In practice, a forecasting method is typically evaluated by forecast accuracy, which is defined by a certain loss function and evaluates the difference between the forecast and the actual outcome of the target variable.
A comparison of several forecasters can then be made, for example, by ranking the forecast accuracies of the forecasters.
In practice, prediction accuracy is measured by scale-dependent measures such as the Root Mean Square Error (RMSE) or measures based on percentage errors such as the Root Mean Square Percentage Error (RMSPE). All of these typical measures are based on the absolute difference between forecast and true values, locally or globally. For an overview of different measures, see \cite{hyndman2006another}. 
In contrast to conventional forecasting methods, nowcasting focuses on predicting the current status and is therefore sometimes also referred to as short-term forecasting. \unsure[noinline]{Ich finde da wird der Unterschied noch nicht so richtig klar} Nowcasting has its origins in meteorology, and the methods were originally developed to describe the current state of the weather in detail and to predict the expected change on a time scale of a few hours (\cite{browning1989nowcasting,schmid2019nowcasting}).
Nowcasting focuses on predictions for the present, the immediate future, and the recent past, and is now being used in other fields such as economics and medicine. For example, Nowcasting can be used to predict important statistics about the current economic situation that is only available with a considerable time lag (\cite{banbura2013now, giannone2006nowcasting,fornaro2020nowcasting,bok2018macroeconomic}). In medicine, epidemic nowcasting is used during an ongoing epidemic to assess the current situation, taking into account the main pathogenic, epidemiological, clinical, and socio-behavioral factors (\cite{wu2021nowcasting}). For possible applications in this area, see \cite{johansson2014nowcasting,gunther2021nowcasting,birrell2021real}.
When predicting values in the present, it is usual that the true values of the recent past are not available, so only an estimate of these values is provided by the nowcast. For example, the current number of cases of an infectious disease may not be immediately available due to delays in data collection or reporting, or the situation may change rapidly so that the reported figures become inaccurate very quickly.
Using the notation for forecasting above, this means that a prediction value $x_{\tau|t}$ also exits for $\tau$ less than or equal to $t$.
Similar to forecasting, nowcasts are often evaluated and compared in the literature on the basis of performance measures such as the root mean absolute error (\cite{gunther2021nowcasting}).

All the measures discussed, regardless of the area, have in common that the comparison and evaluation of the methods take into account the local absolute differences in different ways but do not directly consider whether the correct direction of change is issued.
Because of the different ways in which data are generated and processed depending on the area of application, the data collected should be prepared in different ways for further processing in order to identify the trending ability (see Section~\ref{sec:trending}). For this purpose, a time series must be defined for a given time series (of measurements or predictions) that reflects the recording or prediction of changes (with a time delay $l$). For a series of measurements, the change between two discrete points in time is considered
\begin{equation}
    \diffxl = (x_{t+l} -x_{t})^T_{t=0}. \label{eq:diffxl_measure}
\end{equation}
For forecasts and nowcasts, we use $(y_t)^T_{t=0}$ to denote the true values of the quantity of interest. For forecasts, the current value is known at the time of the forecast, so a forecast for the time $t+l$ means the prediction of a change $x_{t+l|t} -y_{t}$ between $t$ and $t+l$. The time series of change can therefore be defined as 
\begin{equation}
    \diffxl = (x_{t+l|t} -y_{t})^T_{t=0}.\label{eq:diffxl_forecasting}
\end{equation}
For nowcasting, the value of the current point in time does not have to be known at the prediction time. Often, only an estimate of the current value is known. This can be taken into account by the following case distinction
\begin{equation}
\diffxl = 
\begin{cases} 
(x_{t+l|t} -x_{t|t})^T_{t=0} & \text{if } y_{t} \text{ is not known at time } t, \\
(x_{t+l|t} -y_{t})^T_{t=0}  & \text{else}.
\end{cases} \label{eq:diffxl_nowcasting}
\end{equation}



%To account for this, we note the nowcast value at time $t$ of the quantity of interest of time $\tau$ as $x^{t}_{\tau}$ (where $\tau$ can be less than $t$) 

%Data preparation is similar for measurements and predictions. Let $\mathbf{x} = (x_t)^T_{t=1}$ and $\mathbf{y}=(y_t)^T_{t=1}$ be two time series to be compared (e.g. $\mathbf{x}$, $\mathbf{y}$ are two measurements with two different measuring devices/ measurement methods at the same points in time or predicted values of two predictors in the same time period). 



%For the following analysis, the changes between points in time with the same time distance $l\in \mathbb{N}^+ $ in the time series of $\mathbf{x}$ are considered (the same applies for $\mathbf{y}$):

%\[\Delta^lx_t = x_{t+l} - x_{t} \text{ for } t+l\leq T. \]\\
%, then the serie of changes $\mathfrak{X}$, which is later used to assess trendability, is defined as:

%\[\Delta^l \mathfrak{X} _t = x^{t}_{t+l} - x^t_{t} \text{ for } t+l\leq T.\]
%If the true value $y_t$ were known at time $t$, then the following applies: $x^t_t = y_t$.

%\begin{itemize}
%    \item Was ist nowcasting?
%    \item Aus welchem Anwendungsbereichen kommt es? 
%    \item In welchen Bereichen in der Medizin spielt es eine Rolle?
%    \item Warum ist Nowcasting anders als Forecasting/Measurement?
%    \item Wo spielt Trending in der Literatur schon eine Rolle?
%\end{itemize}



\section{Trending assessment}\label{sec:trending}

As outlined in Section~\ref{sec:introduction}, we refer by trending to the consistency of an observed change with the change issued by another measurement, a nowcast, or a forecast.
In this section, we want to specify trending and lay out different methods of measuring and visualizing trending. 
In general, trending can be assessed for different lags $l$ separately, and the relevance of considered lags has to be determined based on the question at hand.
In Section~\ref{sec:notation}, the time series of differences $\diffx$ and $\diffy$ were introduced.
Whereas the computation of $\diffx$ differs in the different contexts, the trending assessment is similar.
In the following, we omit the difference's lag $l$ for the ease of notation; $\diffx$ and $\diffy$ refer to $\diffxl$ and $\diffyl$ for a common lag $l$.
Similarly, we call a measurement, nowcast, or forecast a \textit{prediction} for the quantity of interest $y$. \discuss{Is prediction a good word?}
The remainder of the section is structured as follows.
Section~\ref{subsec:trending-basics} introduces trending basics and visualizes trending in a synthetical example.
Section~\ref{subsec:trending-measures} reviews conveyed quantification approaches from other fields.
The approaches are extended to the trending setting in Section~\ref{subsec:trending-noise}.
Section~\ref{subsec:trending-cond-prob} introduces a new graphical method of trending assessment.
Section~\ref{subsec:trending-bootstrap} gives bootstrap methods for computing confidence intervals for the measures and conducts a simulation study on their properties.


\subsection{Basics of trending and four-quadrant plots}\label{subsec:trending-basics}

As noted above, a prediction is trending perfectly if all occurred directions of change are predicted correctly, that is, the sign of all elements of $\diffx$ and $\diffy$ coincide.
Perfect trending is hardly the case in practice, but a weaker requirement is needed.
Accordingly, when assessing the trending, we examine the statistical consistency of $\sign(\diffx)$ and $\sign(\diffy)$.
A simple yet insightful method is the four-quadrant plot.
For example, it is known in cardiac output measurement analysis~\parencite{Saugel2015,perrino1998intraoperative}. 
Thereby, the occurred changes are plotted together with the predicted changes, that is, $(\diffy_t, \diffx_t)$ for $t = 1, \dots, T$.
Thus, the x-axis of a four-quadrant plot shows the true value differences, whereas the y-axis displays prediction data differences.
Points in the green upper right and lower left quadrants reflect a correct trending for the respective time step, whereas points in the remaining red quadrants show incorrect predicted changes.
Figure~\ref{fig:trending_basic_4q} displays a basic four-quadrant plot.
Points 2, 3, 5, and 6 show trending, whereas points 1, 4, and 7 count for anti-trending behavior.
Figure~\ref{fig:trending_basic_4q_sample} shows a four-quadrant plot for a more extensive simulated data set with $T=1461$, for example, four years of daily data.
The data generation is described in Appendix~\ref{sec:app-trending-data-generation} and will carry through this section.
The four-quadrant plot resembles a butterfly-like shape, with more data in the upper right and lower left quadrants than in the remaining quadrants.
The four-quadrant plot is intuitive to interpret, and the magnitude and direction of change are shown simultaneously.
It can be extended by including information on the date of difference in the point color.
In Figure~\ref{fig:trending_basic_4q_sample_color}, blue points refer to small time indices and turn green for larger $t$.
However, four-quadrant plots become crowded for larger datasets, and the sequential information on the differences is complex to assess thoroughly.
There are other visualization techniques, such as polar plots or the Bland-Altmann analysis.
However, they lack the clearness and intuition of the four-quadrant plot while adding no more information on the trending~\parencite{Saugel2015}.

\begin{figure}
\centering
\begin{subfigure}[t]{.24\textwidth}
\includegraphics{plots/illustrative_examples/4Q_without_excl}
\caption{Basic four-quadrant plot.} \label{fig:trending_basic_4q}
\end{subfigure}\hspace{0.01\textwidth}%
\begin{subfigure}[t]{.24\textwidth}
\includegraphics{plots/illustrative_examples/4q_excl_box}
\caption{Four-quadrant plot with rectangle exclusion area.}\label{fig:trending_basic_4q_excl_box}
\end{subfigure}\hspace{0.01\textwidth}%
\begin{subfigure}[t]{.24\textwidth}
\includegraphics{plots/illustrative_examples/4q_excl_axis}
\caption{Four-quadrant plot with horizontal exclusion area.} \label{fig:trending_basic_4q_excl_axis}
\end{subfigure}\hspace{0.01\textwidth}%
\begin{subfigure}[t]{.24\textwidth}
\includegraphics{plots/illustrative_examples/4q_excl_cross}
\caption{Four-quadrant plot with crossed exclusion area.}\label{fig:trending_basic_4q_excl_cross}
\end{subfigure}%
\caption{Illustrations of the four-quadrant plot with sample points and with and without exclusion areas. }
\label{fig:trending_4q}
\end{figure}

\subsection{Trending ratio and other measures}\label{subsec:trending-measures}

Analyzing the number of points in the green quadrants versus the red quadrants boils down to assessing the probability of trending $P(\diffxrv \diffyrv > 0)$, where $\diffyrv$ and $\diffxrv$ denote random variables for new elements of $\diffy$ and $\diffx$, respectively. \discuss{Überlegung: sollte genauer beschrieben sein, wofür die beiden ZV stehen?}
Note that $z_1 z_2 > 0$ if and only if $\sign(z_1) = \sign(z_2)$ ($z_1, z_2 \in \R \setminus \{ 0 \}$).
To estimate $P(\diffxrv \diffyrv > 0)$ it lies at hand to use
\begin{equation}
    \acc (\diffx, \diffy) \coloneqq \frac{\sum_{t \in \mathcal{T}} \ind{\diffx \diffy > 0}}{\card{T}}.\label{eq:acc}
\end{equation}
We refer to this estimator as the trending ratio of the prediction and set $\mathcal{T} = \{1, \dots, T\}$.
Visually, the measure computes the fraction of points in the upper right or lower left quadrant.
Such a $2 \times 2$ table of counts is often called a contingency table. It is often used in other scientific areas for evaluation, for example, in dichotomous forecasting or as a confusion matrix in classification analysis~\parencites(see, e.g., the introductions in)()[Ch. 4]{James2021}[Ch. 3]{Jolliffe2012}.
There, a wide range of other methods is usually used to analyze further characteristics of contingency tables.
Two simple measures focusing on specific areas of interest are the positive and negative trending ratio $\accp$ and $\accm$, respectively.
They are defined as
\begin{align}
    \accp (\diffx, \diffy) &\coloneqq \frac{\sum_{t \in \mathcal{T}} \ind{\diffxt \diffyt > 0} \ind{\diffxt > 0}}{\sum_{t \in \mathcal{T}} \ind{\diffxt > 0}} \label{eq:accp}\\
    \accm (\diffx, \diffy) &\coloneqq \frac{\sum_{t \in \mathcal{T}} \ind{\diffxt \diffyt > 0} \ind{\diffxt < 0}}{\sum_{t \in \mathcal{T}} \ind{\diffxt < 0}}\label{eq:accm}
\end{align}
In the classification context, these measures are known as positive or negative predictive value and hit rate or detection failure ratio in forecasting.
The two give estimates of $P(\diffxrv \diffyrv > 0 | \diffxrv > 0)$ and $P(\diffxrv \diffyrv > 0 | \diffxrv < 0)$, respectively.
There are various adapted measures for unbalanced outcomes, for example, Cohen's $\kappa$ \parencite{Cohen1960}.
Cohen's $\kappa$ is usually used to measure inter-rater agreement and takes into account the ratio of occurred agreement and the probability of agreement by chance.
Cohen's $\kappa$ reduces to rescaling the trending ratio in the case of a $2\times2$ table and balanced outcomes.
Balanced outcome hereby refers to approximately equal occurrences of values smaller and greater zero for both $\diffx$ and $\diffy$.
Unbalanced outcomes of the differences are unlikely in our setting as they are obtained from differencing time series data.
Nevertheless, suppose in a four-quadrant plot, the positive and negative $\diffy$ are observed to differ widely. In that case, one should consider another as measures such as Cohen's $\kappa$ or those listed in \textcite[Table 3.3]{Jolliffe2012}.
All measures can also be evaluated as a rolling estimate to detect changes in performance over time.
Figure~\ref{fig:trending_ratio_time_series} depicts a rolling window estimate of the trending ratio for the simulated data of Figures~\ref{fig:trending_basic_4q_sample} and \ref{fig:trending_basic_4q_sample_color}.
Thus, the yearly course of the prediction's trending ratio can be detected.
This seasonal behavior cannot be observed in the two four-quadrant plot as the high number of points overlays it.

\begin{figure}
    \centering
    \begin{subfigure}[t]{.24\textwidth}
\includegraphics{plots/illustrative_examples/4Q_sample_without_time}
\caption{Four-quadrant plot with simulated data.}\label{fig:trending_basic_4q_sample}
\end{subfigure}\hspace{0.01\textwidth}
\begin{subfigure}[t]{.24\textwidth}
\includegraphics{plots/illustrative_examples/4Q_sample_with_time}
\caption{The data is colored according to the time index $t$, the greener, the later.}\label{fig:trending_basic_4q_sample_color}
\end{subfigure}\hspace{0.01\textwidth}
\begin{subfigure}[t]{.48\textwidth}
    \includegraphics{plots/illustrative_examples/trending_ratio_time_series.pdf}
    \caption{Rolling estimate of the trending ratio over time with window length 91. }\label{fig:trending_ratio_time_series}
    \end{subfigure}%
    \caption{Visualizations for data with a time-varying trending ratio. We defer information on the data generation process of the data in \ref{fig:trending_basic_4q_sample} and \ref{fig:trending_basic_4q_sample_color} to the appendix (see \ref{sec:app-trending-data-generation}). The trending ratios on the whole data set are $\mu = 0.7577$ and with exclusion area $\diffx < \varepsilon$, $\mu_{1.0} = 0.7712$, respectively. Strong yearly seasonality of the trending ratio becomes visible in Figure~\ref{fig:trending_ratio_time_series}. The green curve $k_t$ shows the probability of $\diffxt$ having the same sign as $\diffyt$ for each time step. The rolling estimates hang back as the windows are backward-looking.}
\end{figure}

Missing values are a common problem, in particular in nowcasting and forecasting.
This will, for example, become evident in the data study in Section~\ref{sec:application-covid}.
Data pairs with missing values can be excluded from the computation of the measures.
Nevertheless, missing values can considerably influence the results, mainly if they appear systematically.
Predicting only in those times of certainty and omitting prediction in other times would improve evaluation measures.
So, although the estimates can be computed, the time spans with missing values should be inspected visually.


\subsection{Accounting for noise}\label{subsec:trending-noise}

The above measures use only the information on a point's quadrant and neglect further details on its location within the quadrant, as displayed in a four-quadrant plot.
However, points close to the zero point have intuitively less explanatory power.
If there is noise or non-systematic effects present in true value or predictions, those points are likely to be driven by noise to a large amount. 
Using an exclusion area around the zero point is a straightforward and highly interpretable extension of the above methods.
Points within that zone are neither plotted in the four-quadrant plot nor included in the calculation of the measures.
The exclusion area's shape can be chosen according to the noise characteristics in predictions and true values.
In general, it is likely that the models of predictions have a noise component and should thus be part of the exclusion area.
We denote the measures accounting for an exclusion area by
\begin{align}
    \acceps (\diffx, \diffy) &\coloneqq \frac{\sum_{t \in \mathcal{T}} \ind{\diffx \diffy > 0} \ind{\abs{\diffxt} > \varepsilon}}{\sum_{t \in \mathcal{T}} \ind{\abs{\diffxt} > \varepsilon}}\label{eq:acceps}\\
    \accpeps (\diffx, \diffy) &\coloneqq \frac{\sum_{t \in \mathcal{T}} \ind{\diffxt \diffyt > 0} \ind{\diffxt > \varepsilon}}{\sum_{t \in \mathcal{T}} \ind{\diffxt > \varepsilon}} \label{eq:accpeps}\\
    \accmeps (\diffx, \diffy) &\coloneqq \frac{\sum_{t \in \mathcal{T}} \ind{\diffxt \diffyt > 0} \ind{\diffxt < \varepsilon}}{\sum_{t \in \mathcal{T}} \ind{\diffxt < \varepsilon }}\label{eq:accmeps}
\end{align}
The estimators can easily be adapted for other shapes of the exclusion area.
Figure~\ref{fig:trending_4q} visualizes different shapes of the exclusion area.
A box-shaped exclusion area leaves points out that are small in both components and are thus likely to be driven away from the zero point by noise.
Thus, only point 1 is excluded.
An exclusion area along one axis removes points in which one of the components could change the sign by a small amount of noise.
This particularly suits prediction models, where small amounts of noise are inevitable.
A cross-shaped exclusion area along both axes accounts for sign reversal in both components.
For those two methods, points 1 and 7 or 1, 6, and 7 are excluded, respectively.
\todo{Berechnung $\mu$ für verschiedene Varianten für Beispiel im Anhang?}
In addition to these straight shapes, exclusion areas could also be determined based on an error model of the components.
For example, one could exclude points likely to be in another quadrant or within a specific quantile range of a distribution around the zero point, for example, exclude the points within the 5 \% quantile and 95 \% quantile of a standard normal distribution. 
These shapes are theoretically appealing, but the interpretation of the resulting estimators gets ambiguous.
Thus, we use the simple exclusion areas above.

In most applications, it is advisable to determine the exact shape and size of the exclusion area based on domain knowledge or expert opinions.
In addition, the size can also be calculated as a proportion of the total variance or the total range of the data.
A third approach is to visualize the trending ratio for different values of $\varepsilon$.
In Section~\ref{sec:application}, there are various examples of such a plot.
As a base approach, we advocate using no exclusion area in the four-quadrant plot as the points do not complicate its interpretation and a $\acceps$-over-$\varepsilon$ plot to visualize the course of the trending ratio over different exclusion area sizes.


\subsection{Conditional trending plot}\label{subsec:trending-cond-prob}
\discuss{Allgemeine Frage: können wir hier von Wahrscheinlichkeit sprechen? }
The above-defined estimators give information on the probabilites $P(\diffxrv \diffyrv > 0)$, $P(\diffxrv \diffyrv > 0 | \diffxrv > \varepsilon)$ and $P(\diffxrv \diffyrv > 0 | \diffxrv < \varepsilon)$.
To assess the trending ability of a prediction, an evaluation on a finder grid is of interest.
Whereas it would be possible to build analog measures on other intervals, assessing $P(\diffxrv \diffyrv > 0 | \diffx = x)$ graphically eases the simultaneous evaluation of various intervals.
In addition, the conditional trending plot facilitates the comparison of various methods in a single plot, and asymmetries of $P(\diffxrv \diffyrv > 0 | \diffx = x)$ with respect to $x$ in the trending ability can be detected.

The probability $P(\diffxrv \diffyrv > 0 | \diffxrv = x)$ cannot be computed by the counting approach, as for almost all $x \in \R$, no data is available.
Instead, the smoothed approach of a multivariate \acf{kde} facilitates a continuous estimation of $P(\diffxrv \diffyrv > 0 | \diffxrv = x)$.
A comprehensive introduction into multivariate \ac{kde} can be found in \textcite{Gramacki2018}, and implementations are available in many programming languages~\parencite[e.g., for  Python in][]{Seabold2010}.
The \ac{kde} yields estimates for $P(\diffxrv \diffyrv > 0 | \diffxrv = x)$ for all values of $x \in \R$.
For kernels with infinite support, for example, the Gaussian kernel, this can convey a false impression of validity for values far beyond the values in $\diffx$.
Thus, the plot should be limited to the core values of $\diffx$ without outliers.
Besides the kernel, a bandwidth selector has to be chosen for a multivariate \ac{kde}.
Various methods are available with different strengths and weaknesses.
Figure~\ref{fig:trending-cond-prob-bw} shows the resulting conditional trending plots for the three well-known selectors, rule-of-thumb, cross-validation maximum likelihood, and cross-validation least squares using the \verb|statsmodels| python package~\parencite{Seabold2010}.
The dashed line shows the theoretical $P(\diffyrv \diffxrv > 0 | \diffxrv = x)$.
The second method, cross-validation least squares, needs long computation times while yielding small or no bandwidth results, even for the two relatively small datasets.
The rule-of-thumb and cross-validation maximum likelihood both yield reasonable results at moderate computation times. 
Further examples, including comparisons between methods concerning their trending ability, are available for the applications in Section~\ref{sec:application}.

\begin{figure}
    \centering
    \begin{subfigure}{.48\textwidth}
        \includegraphics{plots/illustrative_examples/cond_prob_plot_bw_asym_butterfly}
        \caption{First dataset with asymmetric dependence.}
    \end{subfigure}
    \begin{subfigure}{.48\textwidth}
        \includegraphics{plots/illustrative_examples/cond_prob_plot_bw_normal}
        \caption{Second dataset. }
    \end{subfigure}
    \caption{Resulting conditional trending plot for different bandwidth selection processes. Cross-validation least squares takes a considerably larger computation time. It does not converge for the first data set and yields a too small bandwidth for the second data set. The rule of thumb is the fastest method but tends to oversmooth. The cross-validation maximum likelihood method yields a more reasonable bandwidth with moderate computation time. }\label{fig:trending-cond-prob-bw}
\end{figure}


\subsection{Confidence intervals based on bootstrapping}\label{subsec:trending-bootstrap}

Confidence intervals can be issued to account for the estimation uncertainty of the measures above.
Bootstrapping confidence intervals are a nonparametric technique based on resampling~\parencite[for introductions see][]{Hesterberg2011,Bittmann2021}.
Whereas classical confidence intervals are computed from parametric assumptions on the underlying data, new samples are drawn with replacement from the dataset in bootstrapping.
The confidence interval is then computed based on the bootstrap samples.
We examine three methods for bootstrapping here: the intuitive percentile and the more sophisticated basic and \ac{bca} method.
In the \textit{percentile} approach, the confidence interval for level $\alpha$ is built directly from the bootstrap estimators' empirical distribution.
The \textit{basic} approach computes the confidence interval based on the non-bootstrapped estimate using the bootstrapped quantile deviations~\parencite{Davison1997}.
The \ac{bca} method modifies the quantiles of the empirical bootstrap distribution by a bias and an acceleration parameter~\parencite{Efron1987}.
Typically, the percentile approach needs larger datasets and provides an easy and fast estimate, while the \ac{bca} is computationally expensive but needs smaller datasets for reasonable confidence intervals.
The basic approach balances those two aims.
We compare the approaches in a small synthetic data study concerning their small-dataset behavior and computation time.
We vary the number of available time steps $T$ to be a typical time-series value, such as 30 for daily data in a month, 52 for weekly data, 168, 365, 720, and 1024.
The considered datasets are again outlined in Appendix~\ref{sec:app-trending-data-generation}, the first dataset with asymmetric dependence.
In the calculations, the \verb|scipy| package's implementation of bootstrap confidence intervals is used~\parencite{Virtanen2020}.
The prescribed confidence level is 90 \%, and the number of bootstrap samples is $10,000$.
The share of confidence intervals covering the true values per method and $T$ are shown in Table~\ref{tab:trending_bootstrap}.
The true values of the accuracy are computed based on a dataset of size $10^8$, yielding 0.7501 and 0.7700 for the two datasets.
The computation times per method and dataset are shown in Figure~\ref{fig:trending_bootstrap_time}.
For the small sample sizes up to $T = 168$, only the \ac{bca} method keeps the confidence interval size and yields slightly wider confidence intervals. 
The method's results do not differ for the larger sample sizes. 
The computation time for the \ac{bca} method is slightly larger than for the other methods, but all methods have a moderate computation time.

\begin{table}
    \centering
    \begin{subtable}{.48\textwidth}
        \begin{tabular}{llll}
\toprule
 & percentile & basic & bca \\
\midrule
30 & 0.84 (0.249) & 0.86 (0.250) & 0.91 (nan) \\
52 & 0.89 (0.194) & 0.89 (0.193) & 0.89 (0.198) \\
168 & 0.91 (0.109) & 0.90 (0.109) & 0.90 (0.110) \\
365 & 0.90 (0.074) & 0.90 (0.074) & 0.90 (0.074) \\
720 & 0.90 (0.053) & 0.90 (0.053) & 0.90 (0.053) \\
1024 & 0.90 (0.044) & 0.90 (0.044) & 0.89 (0.044) \\
\bottomrule
\end{tabular}

        \caption{First dataset}
    \end{subtable}\hspace{0.02\textwidth}
    \begin{subtable}{.48\textwidth}
        \begin{tabular}{llll}
\toprule
 & percentile & basic & BCa \\
\midrule
30 & 0.87 (0.243) & 0.88 (0.242) & 0.92 (0.249) \\
52 & 0.87 (0.188) & 0.89 (0.188) & 0.90 (0.192) \\
168 & 0.89 (0.106) & 0.90 (0.106) & 0.90 (0.107) \\
365 & 0.90 (0.072) & 0.90 (0.072) & 0.90 (0.072) \\
720 & 0.90 (0.052) & 0.90 (0.052) & 0.90 (0.052) \\
1024 & 0.89 (0.043) & 0.90 (0.043) & 0.90 (0.043) \\
\bottomrule
\end{tabular}

        \caption{Second dataset}
    \end{subtable}
    \caption{Proportion of bootstrapping confidence intervals covering the true value of trending ratio per method and sample size $T$. The average width of the confidence interval is listed in brackets.}
    \label{tab:trending_bootstrap}
\end{table}

\begin{table}
    \centering
    \begin{subfigure}{0.48\textwidth}
        \includegraphics{plots/illustrative_examples/boxplot_comp_time_butterfly}
        \caption{First dataset}
    \end{subfigure}
    \begin{subfigure}{0.48\textwidth}
        \includegraphics{plots/illustrative_examples/boxplot_comp_time_normal}
        \caption{Second dataset}
    \end{subfigure}
    \caption{Boxplot of the computation time of the different bootstrapping method and data set sizes $T$. The computation time refers to bootstrapping one confidence interval based upon $10,000$ values. Each boxplot reflects $10,000$ samples. The \ac{bca} method takes slightly longer than the other two, but the difference is negligible.}
    \label{fig:trending_bootstrap_time}
\end{table}



In standard bootstrapping, the sample is assumed to be independent and identically distributed.
Thus, before applying bootstrapping methods, the strength of sequential dependence should be inspected, for example, by analyzing the autocorrelation and partial autocorrelation.
The simple bootstrapping methods above do not account for serial dependence. 
Bootstrapping focusing on time series data is covered in~\textcite{Hardle2003,Kreiss2012}, for example.


%\section{Simulation Studies (?)}
%
%Eher nicht, weil angewandtes Paper

\section{Application to real-world data} \label{sec:application}
In this section, we evalute the trending in three different applications.
In the first application in Section~\ref{sec:application-covid}, we consider the nowcasting of the seven-day hospitalization rate for COVID-19 in Germany.
Section~\ref{sec:application-eda} applies the trending assessment to forecast methods for the number of arrivals in a large emergency department.
In the last application in Section~\ref{sec:application_measurement}, we consider the trending assessment of non-invasive blood pressure measurements compared to invasive blood pressure measurements.

\subsection{Covid nowcasting} \label{sec:application-covid}

Amid the COVID-19 pandemic, the importance of accurate and prompt nowcasts of the pandemic's progression has become evident.
Various indicators measured the pandemic's spread and severity. 
In Germany, the seven-day hospitalization rate was established as a central steering measure for COVID-19 measures in November 2021, and the imposition of severe public restrictions was based on it~\citep{RobertKochInstitute2021}. 
The Robert Koch Institute (RKI) provided preliminary daily reports on the seven-day hospitalization rate.
However, these reports were subject to severe delays and revisions in two sources.
The first source is technical delays in the reporting process, for example, due to different authorities passing the data to the RKI~\citep{RobertKochInstitute2024}.
The second, more systematic source is the structure of the seven-day hospitalization rate.
To a given date, all the cases are allocated whose first positive test result was on that date and who were hospitalized in relation to the disease in the following.
The seven-day hospitalization rate is the average number of those cases per $100,000$ inhabitants on the given date and six days before.
Thus, the final seven-day hospitalization rate can only be reported with a significant delay of more than 70 days, as the hospitalization of infected inhabitants can occur much later than the first positive test.
Nevertheless, as the seven-day hospitalization rate was considered a major indicator of the pandemic's development, many organizations and institutions started to issue nowcasts, including research teams and newspapers.
To collect nowcasts of the seven-day hospitalization rate by different nowcast groups, the COVID19-Nowcasting-Hub was established~\citep{ChairOfEconometricsAndStatisticsAtKarlsruheInstituteOfTechnology2024}.
The nowcasts contain the seven-day hospitalization rate's predictive mean, median, and other quantiles.

\begin{figure}
    \centering
    \begin{subfigure}[t]{0.48\textwidth}
    \includegraphics{plots/covid_nowcast/00_true_data.pdf}
    \caption{Realisations.}
    \label{fig:app-covid-true}
    \end{subfigure}\hfill
    \begin{subfigure}[t]{0.48\textwidth}
    \includegraphics{plots/covid_nowcast/00_nowcast_data.pdf}
    \caption{Same-day nowcasts.}
    \label{fig:app-covid-nowcast}
        \end{subfigure}
    \caption{True and nowcast data of the seven-day-hospitalization in Germany from November 22, 2021, to April 29, 2022 \citep{ChairOfEconometricsAndStatisticsAtKarlsruheInstituteOfTechnology2024}.
    The outliers in the RKI model of values above $10^8$ are removed before the following analysis.}
    \label{fig:app-covid-true-nowcast}
\end{figure}

The data contains eight nowcasts from scientific and public institutions, nowcast communities, and a newspaper, using different input variables, calendar data, and length of training data.
The model structures are diverse, including Bayesian models, generalized additive models, and parametric bootstrapping.
Table~\ref{tab:app-covid-models} in the appendix lists the different abbreviations and the respective names in the COVID19-Nowcasting-Hub.
For information on the model design, we refer to~\citet{Wolffram2023}.
Using the nowcasts, two ensemble methods are constructed using the ensembles' mean or median.
We denote them by ENS-MEAN and ENS-MED.
\citet{Wolffram2023} describes the design of the nowcast tasks and the data submission guidelines for the teams stated in a preregistered study. 
In line with the initial study design, we consider the period from November 22, 2021, to April 29, 2022, as the evaluation period.
For the true values, we use the data from February 8, 2024.
The nowcasts are resolved with regard to all inhabitants and specific geographical and age breakdowns.
We do not expand on the models' performance on specific regions or age groups in Germany and the probabilistic nowcast assessment.
Figure~\ref{fig:app-covid-true-nowcast} displays the true and nowcast data for the evaluation period.
The time comprises the fourth wave's end in December 2021 and nearly the entire fifth wave of the pandemic in Germany, lasting until May 28, 2022~\citep{Tolksdorf2022}.

Table~\ref{tab:app-covid-rmse} summarizes the point evaluation measures for the issued mean of the different models.
The models issue same-day nowcasts for nearly all 159 days of the evaluation period~\citep[for explanations of the missing values, see][Tables A2, A3, and A4]{Wolffram2023}.
The best-performing models in terms of RMSE and MAE are the ILM and RKI models.
The ensemble methods perform worse than the best models regarding the mean location.
The performance of the models is diverse, with more than twice as high RMSE values for the worst models compared to the best models.
Note that the high values for the EPI model could be driven by an exceptionally far-off value at the end of the evaluation period.

In addition to close inspection of the point evaluation measures, assessing the trending of the nowcasts is crucial.
To assess the impact of taken measures and the direction of the curve, it is essential to distinguish between rising and falling hospitalization rates.
If hospitalization rates rise, measures should be tightened, while falling rates might allow for loosening measures.
Especially, asymmetries are of interest to assess whether some models are better at recognizing a fall than a rise or vice versa.

\begin{table}[]
    \centering
    \begin{tabular}{llllr}
\toprule
 & rmse & mae & mse & count \\
model &  &  &  &  \\
\midrule
ILM-prop & 614.5 & 410.3 & 377,570.5 & 530 \\
RIVM-KEW & 630.3 & 418.5 & 397,230.8 & 817 \\
NowcastHub-MeanEnsemble & 671.1 & 466.9 & 450,392.0 & 610 \\
NowcastHub-MedianEnsemble & 681.0 & 467.2 & 463,694.3 & 610 \\
LMU\_StaBLab-GAM\_nowcast & 759.9 & 540.3 & 577,518.7 & 574 \\
KIT-simple\_nowcast & 919.0 & 639.3 & 844,614.2 & 861 \\
SZ-hosp\_nowcast & 955.2 & 688.3 & 912,327.0 & 535 \\
Epiforecasts-independent & 1,455.8 & 860.4 & 2,119,446.4 & 275 \\
SU-hier\_bayes & 1,364,823.3 & 83,367.7 & 1,862,742,601,687.6 & 426 \\
RKI-weekly\_report & 84,339,155.2 & 7,624,993.0 & 7,113,093,105,644,206.0 & 338 \\
\bottomrule
\end{tabular}

    \caption{Point evaluation measures for the issued mean of the different models. The evaluation period comprises 159 days. }
    \label{tab:app-covid-rmse}
\end{table}

\subsubsection*{Results}

In the following, we apply the trending assessment of Section~\ref{sec:trending} to the nowcasts of the seven-day hospitalization rate.
We report the trending for the horizons 1, 7, and 14 days.
While horizon one assesses the short-term trending, horizons seven and 14 consider the medium-term trending.
The horizons seven and 14 are particularly interesting, reflecting a usual period until new policy changes are taken.

Before stating the results, we provide background information on the marginal distributions of the true values and nowcasts for the different horizons.
Table~\ref{tab:app-covid-marginals} in Appendix~\ref{sec:appendix-application-covid} presents marginal statistics such as standard deviation and quantiles of the nowcasts and true values.
Overall, the variability and general level of differences grow with the horizon.
The standard deviation increases from roughly 300 for horizon one to 1,200 for horizon seven and 2,000 for horizon 14 days.
Similarly, the 10\%-quantile of differences increases.
The 10\%-quantile is used for the exclusion areas in the trending assessment.
The exclusion area is rectangular; a point falls within it if both $\diffy$ and $\diffx$ are below the respective 10\%-quantile of the absolute differences.
Thus, points are still included in the trending assessment if they are large in one dimension but not in the other, thus ensuring that substantial changes in, for example, $\diffy$ are to be recognized by the nowcast and vice versa.

Table~\ref{tab:app-covid-trending-ratios-lag-7} lists the trending ratios for all models with and without exclusion areas for the horizon of seven days.
The trending ratios without exclusion area range from 0.72 to 0.85 for the horizon of seven days.
The negative trending ratios are higher than the positive trending ratios for all models.
The confidence intervals for the positive and negative trending ratios do not overlap for all models, indicating that the trending ratios are indeed different.
The 10\%-quantile exclusion areas have, at most, an influence of 0.03 on the ratios.
The model with the highest trending ratio is the ILM model, and the model with the lowest is the RKI model.
The confidence intervals between all models overlap.
The positive trending ratio implies a similar ranking of the models, while the negative ratio is second best for the RKI model.
For the horizons of one and 14 days, we refer to Table~\ref{tab:app-covid-trending-ratios-lag-1-14} in Appendix~\ref{sec:appendix-application-covid}.

Figure~\ref{fig:app-covid-cond-prob-trending-ratio-7} shows the conditional trending plots and the trending ratio over the exclusion area for the horizon seven days; the respective plots for the horizons one day and 14 days are shown in Figure~\ref{fig:app-covid-cond-prob-trending-ratio-1-14}.
Here, only the best models in point evaluation measures, ILM, RKI, RIVM, and ENS-MED, are shown to keep the plots easily readable.
If RKI or ILM issues a fall in the hospitalization rate, the probability of a fall is higher than if RIVM or ENS-MED issues a fall.
The opposite is the case for a nowcasted hospitalization rate increase, and the difference between the models' performance is higher.
Similar observations can be made for the horizon of 14 days in Figure~\ref{fig:app-covid-cond-prob-14}.
For a horizon of one day, the models' conditional trending ability difference is less pronounced (see Figure~\ref{fig:app-covid-cond-prob-1}).
The RKI model is still less conclusive when issuing an increase in the hospitalization rate, while RIVM is most informative in that case.
The curves cross for an issued fall, with ENS-MED being on top for issued falls above 250.

The trending ratios for various exclusion areas are shown in Figure~\ref{fig:app-covid-trending-ratio-7}.
In general, the trending ratio increases with larger exclusion areas.
While the RIVM and ENS-MED trending ratios evolve similarly, the RKI and ILM trending ratios get closer.
For the horizon of one day, the RKI trending ratio decreases with increasing exclusion area size while the other models rise (see Figure~\ref{fig:app-covid-trending-ratio-1}).
For the horizon of 14 days, all trending ratio curves increase with the exclusion area size (see Figure~\ref{fig:app-covid-trending-ratio-14}).

\begin{table}
    \centering
    \tiny
    \begin{tabular}{l p{0.11\textwidth} p{0.11\textwidth} p{0.11\textwidth} p{0.11\textwidth} p{0.11\textwidth} p{0.11\textwidth}}
\toprule
 & $\mu^7$ & $\mu^{+, 7}$ & $\mu^{-, 7}$ & $\mu^7_{q_{0.1}}$ & $\mu^{+, 7}_{q_{0.1}}$ & $\mu^{-, 7}_{q_{0.1}}$ \\
\midrule
EPI & {0.77\newline(0.70, 0.82)} & {0.67\newline(0.58, 0.76)} & {0.87\newline(0.79, 0.92)} & {0.78\newline(0.72, 0.83)} & {0.68\newline(0.59, 0.77)} & {0.88\newline(0.80, 0.93)} \\
ILM & {0.85\newline(0.80, 0.90)} & {0.73\newline(0.64, 0.80)} & {0.99\newline(0.94, 1.00)} & {0.85\newline(0.80, 0.89)} & {0.74\newline(0.65, 0.81)} & {0.99\newline(0.94, 1.00)} \\
KIT & {0.74\newline(0.68, 0.80)} & {0.64\newline(0.55, 0.72)} & {0.87\newline(0.79, 0.93)} & {0.75\newline(0.69, 0.80)} & {0.64\newline(0.55, 0.72)} & {0.88\newline(0.81, 0.94)} \\
LMU & {0.80\newline(0.74, 0.85)} & {0.70\newline(0.62, 0.79)} & {0.91\newline(0.84, 0.95)} & {0.81\newline(0.75, 0.86)} & {0.72\newline(0.63, 0.79)} & {0.92\newline(0.85, 0.96)} \\
ENS-MEAN & {0.82\newline(0.76, 0.86)} & {0.71\newline(0.62, 0.78)} & {0.94\newline(0.89, 0.99)} & {0.82\newline(0.77, 0.87)} & {0.71\newline(0.63, 0.79)} & {0.96\newline(0.90, 0.99)} \\
ENS-MED & {0.82\newline(0.76, 0.86)} & {0.70\newline(0.62, 0.78)} & {0.96\newline(0.90, 0.99)} & {0.83\newline(0.77, 0.87)} & {0.72\newline(0.64, 0.79)} & {0.96\newline(0.90, 0.99)} \\
RIVM & {0.83\newline(0.77, 0.87)} & {0.74\newline(0.65, 0.81)} & {0.92\newline(0.86, 0.96)} & {0.83\newline(0.78, 0.88)} & {0.74\newline(0.65, 0.81)} & {0.93\newline(0.87, 0.97)} \\
RKI & {0.72\newline(0.65, 0.77)} & {0.60\newline(0.51, 0.68)} & {0.98\newline(0.92, 1.00)} & {0.73\newline(0.66, 0.78)} & {0.61\newline(0.53, 0.68)} & {0.98\newline(0.92, 1.00)} \\
SU & {0.81\newline(0.75, 0.86)} & {0.71\newline(0.62, 0.78)} & {0.92\newline(0.85, 0.96)} & {0.81\newline(0.75, 0.85)} & {0.71\newline(0.63, 0.79)} & {0.92\newline(0.85, 0.96)} \\
SZ & {0.78\newline(0.72, 0.83)} & {0.67\newline(0.58, 0.75)} & {0.91\newline(0.84, 0.96)} & {0.78\newline(0.72, 0.83)} & {0.67\newline(0.58, 0.75)} & {0.92\newline(0.85, 0.97)} \\
\bottomrule
\end{tabular}

    \caption{The trending ratio $\accl[7]$, positive trending ratio $\accpl[7]$, and negative trending ratio $\accml[7]$ for the models with and without exclusion areas for the horizon seven days. The exclusion areas are rectangles centered on the zero points with a width and height of twice the 10\%-quantile of the absolute values of nowcast and true values. }
    \label{tab:app-covid-trending-ratios-lag-7}
\end{table}

\begin{figure}
    \centering
%    \includegraphics{}
    \begin{subfigure}[t]{.48\textwidth}
    \includegraphics{plots/covid_nowcast/40_cond_prob_lag_7}
    \caption{Conditional trending plot.}\label{fig:app-covid-cond-prob-7}
    \end{subfigure}\hfill
    \begin{subfigure}[t]{.48\textwidth}
    \includegraphics{plots/covid_nowcast/40_acc_eps_lag_7}
    \caption{Trending ratio over exclusion area size in $\diffx$.}\label{fig:app-covid-trending-ratio-7}
    \end{subfigure}
    \caption{Conditional trending plot and trending ratio over exclusion area for the nowcasts of the seven-day hospitalization rate ILM, RKI, RIVM, and ENS-MED for the horizon seven days.}
    \label{fig:app-covid-cond-prob-trending-ratio-7}
\end{figure}



\subsubsection*{Discussion}

For all horizons, the influence of the exclusion area on the 10\%-quantile level is negligible.
For example, the trending ratio changes at most by 0.03 for the EPI model with $\accml[14]$.
The exclusion areas are thus not crucial for the trending assessment in the case of the nowcasts of the seven-day hospitalization rate.
The lower bound of confidence intervals is at least 0.68 for all models, indicating that they perform better than random guessing the trend.

Trending assessment evaluates the models differently from point evaluation measures.
RKI is among the best in point evaluation measures but performs worse in trending assessment.
The assessment of asymmetry in the conditional trending plots is crucial for interpreting the trending ratios, with the RKI model being the most prominent example.

Figure~\ref{fig:app-covid-trending-ratio-7} shows that the trending ratio increases with larger exclusion areas. 
This indicates that if the model predicts a large change, the direction is indeed better than when a small change is predicted.

A more extensive training size would be beneficial for assessing the models' performance.
For the evaluation period of 159 days, the trending ratio confidence intervals overlap; thus, no conclusions can be drawn from the trending evaluation.

\subsection{Forecasting emergency department arrivals}\label{sec:application-eda}

In a second example, we consider forecasting the hourly number of arrivals in a large emergency department.
Good forecasts are crucial for planning staff and resources.
Several models are used for predicting hourly outcomes in a study by \citet{Rostami-Tabar2023}.
Every 12 hours, the models issue hourly forecasts for the next 48 hours.
Thus, the management can take measures according to the expected number of arrivals, for example, through redeploying staff and reconfiguring units.

\citet{Rostami-Tabar2023} publish means and probabilistic quantile forecasts, which are evaluated through RMSE, pinball loss, pinball skill scores, and PIT-histograms.
The models are trained on data from April 1, 2014, to February 28, 2018, and evaluated on data from March 1, 2018, to February 28, 2019.
For further notes on the models and the evaluation, we refer to \citet{Rostami-Tabar2023}.
From the issued forecasts, we use the mean as a point forecast for the trending assessment and evaluate the probabilistic trending subsequently.
We use the forecasts issued at the first time point for every target time.
Thus, the forecasts are issued 36 to 48 hours ahead of the target time, and the emergency department management has time to adjust the measures according to the expected number of arrivals.
Considering only the forecasts of at least 36 hours ahead, we restrict the evaluation period to March 2, 2018, at noon, to February 28, 2019, at 23:00, comprising 8,724 hours.

In this setup, trending assessment is a simple and intuitive way to assess the models' performance.
The trending perspective is easy for the management to understand and implement, as simple comparisons of the expected workload to a recent shift can be made.
If, for example, the staff was near capacity in the last shift and an increase in the number of arrivals is expected, the management can take measures to adjust the workload.

The number of arrivals has a strong weekly and daily pattern.
Thus, we consider the horizons of 72 hours, the last already observed shift of the same hour of day, and seven days, the previous shift of the same hour and day.
Table~\ref{tab:app-eda-point-evaluation} lists the point evaluation measures and the count of available forecasts.
The best-performing models regarding RMSE and MAE are the NBI-2 and Poisson-2 models.
More than 8,600 forecasts are available for all models, with differences in the number due to missing values on four afternoons in 2018.
Note that the reported values for the RMSE differ from those in \citet{Rostami-Tabar2023}.
In contrast to their work, we use only the forecast data at least 36 hours ahead and not the entire forecast data for evaluation.

\begin{table}
\centering
\begin{tabular}{l r r r}
\toprule
Model & RMSE & MAE & Count \\
\midrule
NBI-2 & 8.883 & 3.200 & 8688 \\
Poisson-2 & 8.884 & 3.200 & 8688 \\
Poisson-1 & 9.164 & 3.238 & 8688 \\
Benchmark-2 & 9.246 & 3.236 & 8688 \\
Ttr-2 & 9.394 & 3.266 & 8688 \\
NOtr-1 & 9.413 & 3.276 & 8688 \\
NOtr-2 & 9.413 & 3.276 & 8688 \\
Poisson-2-I & 9.458 & 3.276 & 8688 \\
Benchmark-1 & 10.065 & 3.331 & 8688 \\
GBM-2 & 11.663 & 3.542 & 8688 \\
tbats & 12.905 & 3.912 & 8724 \\
Prophet & 13.078 & 3.877 & 8724 \\
qreg-1 & 13.337 & 3.758 & 8688 \\
Regression-Poisson & 21.162 & 4.818 & 8724 \\
ADAM-iETSX & 28.000 & 5.561 & 8724 \\
ETS & 29.358 & 5.742 & 8724 \\
\bottomrule
\end{tabular}

\caption{Point evaluation measures for the models. The smaller count for some models stems from missing forecasts scattered throughout the evaluation period.}\label{tab:app-eda-point-evaluation}
\end{table}


\subsubsection*{Results}

Table~\ref{tab:app-eda-marginals} analyzes the differences in marginal distributions for the forecasts and true values for the horizons of three and seven days.
Note that the difference definition aligns with Section~\ref{subsec:notation}, defined as the difference between the forecasted mean and true value of three and seven days before, as the true value is available when issuing the forecast.
The fraction of positive differences varies between 0.39 and 0.63 for the horizon of three days and between 0.37 and 0.63 for the horizon of seven days.
The variability of differences decreases for the larger horizon for most models; only for the ETS model does it increase.
The 10\%-quantile of the differences is between zero and one for all models and horizons.
Thus, we exclude only differences smaller than one from the trending assessment.
The resulting fraction of included values in the computation is also listed in Table~\ref{tab:app-eda-marginals} and is at least 79\% of the values.

\begin{table}
    \centering
    \begin{tabular}{lllllllll}
\toprule
 & (1), l=3 & $\sigma_{x^{\Delta, 3}}$ & $q_{0.1} (x^{\Delta, 3})$ & (2), l=3 & (1), l=7 & $\sigma_{x^{\Delta, 7}}$ & $q_{0.1} (x^{\Delta, 7})$ & (2), l=7 \\
\midrule
Benchmark-1 & 0.45 & 5.05 & 0.50 & 0.80 & 0.44 & 4.43 & 0.47 & 0.78 \\
Benchmark-2 & 0.51 & 5.11 & 0.52 & 0.80 & 0.50 & 4.29 & 0.45 & 0.78 \\
Poisson-1 & 0.53 & 5.04 & 0.51 & 0.81 & 0.52 & 4.38 & 0.48 & 0.79 \\
Poisson-2 & 0.53 & 5.05 & 0.52 & 0.80 & 0.53 & 4.42 & 0.48 & 0.78 \\
NOtr-1 & 0.52 & 5.03 & 0.51 & 0.81 & 0.51 & 4.41 & 0.49 & 0.79 \\
NOtr-2 & 0.52 & 5.03 & 0.51 & 0.81 & 0.51 & 4.41 & 0.49 & 0.79 \\
GBM-2 & 0.39 & 4.93 & 0.51 & 0.80 & 0.37 & 4.61 & 0.49 & 0.79 \\
Ttr-2 & 0.51 & 5.03 & 0.50 & 0.81 & 0.50 & 4.41 & 0.49 & 0.79 \\
NBI-2 & 0.53 & 5.04 & 0.52 & 0.81 & 0.53 & 4.41 & 0.48 & 0.79 \\
qreg-1 & 0.39 & 5.01 & 0.49 & 0.81 & 0.39 & 4.84 & 0.51 & 0.80 \\
Poisson-2-I & 0.51 & 5.03 & 0.51 & 0.81 & 0.50 & 4.42 & 0.49 & 0.79 \\
tbats & 0.63 & 5.35 & 1.00 & 0.92 & 0.63 & 5.04 & 1.00 & 0.92 \\
ADAM-iETSX & 0.57 & 7.76 & 0.83 & 0.88 & 0.57 & 7.49 & 0.78 & 0.87 \\
ETS & 0.58 & 7.49 & 0.78 & 0.87 & 0.58 & 7.68 & 0.84 & 0.88 \\
Regression-Poisson & 0.51 & 6.65 & 0.67 & 0.85 & 0.51 & 6.49 & 0.67 & 0.85 \\
Prophet & 0.62 & 5.27 & 1.00 & 0.91 & 0.62 & 5.15 & 1.00 & 0.91 \\
True & 0.54 & 6.62 & 1.00 & 0.93 & 0.55 & 5.89 & 1.00 & 0.92 \\
\bottomrule
\end{tabular}

    \caption{Marginal analysis of the nowcast and true differences. The column (1) shows the fraction of values greater than zero for horizon $l$, $\sigma_{x^{\Delta, l}}$ the standard deviation, and $q_{0.1} (x^{\Delta, l})$ the 10\% quantile of the differences' absolute values.}
    \label{tab:app-eda-marginals}
\end{table}

Table~\ref{tab:app-eda-trending-ratios} lists the trending ratios for all models for three and seven-day horizons.
The trending ratios range from 0.68 to 0.84 for a horizon of three days and from 0.68 to 0.82 for seven days.
The negative and positive trending ratios differ for all models and horizons.
For some models, for example, the GBM-2 model, the positive trending ratio is higher; for some models, for example, the tbats model, the negative trending ratio is higher.
The confidence interval width is at most 0.02 for the trending ratios and at most 0.03 for the positive and negative trending ratios.
The models GBM-2, qreg-1, and Benchmark-1 have the highest positive trending ratio for three and seven days horizon, while Poisson-2 and NBI-2 have the highest negative trending ratio.

Figure~\ref{fig:app-eda-cond-prob} shows the conditional trending plots for the models Benchmark-1, GBM-2, NBI-2, Poisson-2, and qreg-1 for the horizons three and seven days and thus inspects the local trending ability of the models with highest positive and negative trending ratio.
The conditional trending plots show similar courses for the two horizons, while the curves are shifted downwards for the horizon of seven days.
The model's relative trending ability evolves consistently for the two horizons, with the NBI-2 and Poisson-2 models being indistinguishable.
The GBM-2 model outperforms the qreg-1 model for all $x$.
The models NBI-2 and Poisson-2 have the highest trending ability for all negative values of $x$ and the lowest trending ability for all positive values of $x$.
Benchmark-1 lies between the other models for all $x$.

Figure~\ref{fig:app-eda-prob} visualizes the probabilistic trending evaluation for the same subset of models.
The \acfp{bs} are shown in Figure~\ref{fig:app-eda-prob-brier}, and the reliability diagrams for the horizons three and seven days are shown in Figures~\ref{fig:app-eda-prob-rel-3} and \ref{fig:app-eda-prob-rel-7}.
The \acp{bs} are smallest for NBI-2 and Poisson-2 for both horizons, while the \acp{bs} for the other models are larger and differ more.
The qreg-1 model has the highest \ac{bs} for both horizons.
The reliability diagrams of GBM-2 and NBI-2 are also close and show a too-small fraction of increases for the predicted probability overall.
For the other models, the reliability diagrams show a fraction of increases that are too large for the corresponding predicted probability.

\begin{table}
    \centering
    \begin{tabular}{l p{0.11\textwidth} p{0.11\textwidth} p{0.11\textwidth} p{0.11\textwidth} p{0.11\textwidth} p{0.11\textwidth}}
\toprule
 & $\mu^{3}_+$ & $\mu^{+, 3}_+$ & $\mu^{-, 3}_+$ & $\mu^{7}_+$ & $\mu^{+, 7}_+$ & $\mu^{-, 7}_+$ \\
\midrule
ADAM-iETSX & {0.70\newline(0.69, 0.71)} & {0.68\newline(0.67, 0.69)} & {0.72\newline(0.71, 0.73)} & {0.68\newline(0.67, 0.69)} & {0.67\newline(0.66, 0.69)} & {0.69\newline(0.67, 0.70)} \\
Benchmark-1 & {0.83\newline(0.82, 0.84)} & {0.86\newline(0.85, 0.87)} & {0.81\newline(0.79, 0.82)} & {0.81\newline(0.80, 0.82)} & {0.86\newline(0.85, 0.87)} & {0.78\newline(0.76, 0.79)} \\
Benchmark-2 & {0.84\newline(0.83, 0.84)} & {0.83\newline(0.82, 0.85)} & {0.84\newline(0.83, 0.85)} & {0.82\newline(0.81, 0.83)} & {0.83\newline(0.82, 0.84)} & {0.80\newline(0.79, 0.82)} \\
ETS & {0.68\newline(0.67, 0.69)} & {0.66\newline(0.65, 0.67)} & {0.70\newline(0.69, 0.72)} & {0.67\newline(0.66, 0.68)} & {0.66\newline(0.64, 0.67)} & {0.68\newline(0.66, 0.69)} \\
GBM-2 & {0.82\newline(0.81, 0.82)} & {0.90\newline(0.89, 0.91)} & {0.77\newline(0.76, 0.78)} & {0.78\newline(0.77, 0.79)} & {0.88\newline(0.87, 0.90)} & {0.73\newline(0.72, 0.74)} \\
NBI-2 & {0.84\newline(0.83, 0.85)} & {0.83\newline(0.82, 0.84)} & {0.85\newline(0.84, 0.86)} & {0.82\newline(0.81, 0.83)} & {0.82\newline(0.81, 0.83)} & {0.82\newline(0.81, 0.83)} \\
NOtr-1 & {0.83\newline(0.83, 0.84)} & {0.83\newline(0.82, 0.84)} & {0.84\newline(0.82, 0.85)} & {0.81\newline(0.80, 0.82)} & {0.82\newline(0.81, 0.83)} & {0.80\newline(0.79, 0.81)} \\
NOtr-2 & {0.83\newline(0.83, 0.84)} & {0.83\newline(0.82, 0.84)} & {0.84\newline(0.82, 0.85)} & {0.81\newline(0.80, 0.82)} & {0.82\newline(0.81, 0.83)} & {0.80\newline(0.79, 0.81)} \\
Poisson-1 & {0.84\newline(0.83, 0.84)} & {0.82\newline(0.81, 0.83)} & {0.85\newline(0.84, 0.86)} & {0.82\newline(0.81, 0.82)} & {0.82\newline(0.81, 0.83)} & {0.81\newline(0.80, 0.82)} \\
Poisson-2 & {0.84\newline(0.83, 0.85)} & {0.83\newline(0.82, 0.84)} & {0.85\newline(0.84, 0.86)} & {0.82\newline(0.81, 0.82)} & {0.82\newline(0.81, 0.83)} & {0.82\newline(0.80, 0.83)} \\
Poisson-2-I & {0.83\newline(0.83, 0.84)} & {0.84\newline(0.83, 0.85)} & {0.83\newline(0.82, 0.84)} & {0.81\newline(0.80, 0.82)} & {0.83\newline(0.81, 0.84)} & {0.80\newline(0.79, 0.81)} \\
Prophet & {0.75\newline(0.74, 0.76)} & {0.72\newline(0.71, 0.73)} & {0.79\newline(0.77, 0.80)} & {0.74\newline(0.73, 0.74)} & {0.72\newline(0.70, 0.73)} & {0.76\newline(0.75, 0.77)} \\
Regression-Poisson & {0.72\newline(0.71, 0.73)} & {0.73\newline(0.71, 0.74)} & {0.72\newline(0.70, 0.73)} & {0.70\newline(0.69, 0.71)} & {0.71\newline(0.70, 0.73)} & {0.69\newline(0.67, 0.70)} \\
Ttr-2 & {0.84\newline(0.83, 0.84)} & {0.84\newline(0.83, 0.85)} & {0.83\newline(0.82, 0.85)} & {0.81\newline(0.80, 0.82)} & {0.83\newline(0.82, 0.84)} & {0.80\newline(0.79, 0.81)} \\
qreg-1 & {0.80\newline(0.79, 0.80)} & {0.88\newline(0.87, 0.89)} & {0.75\newline(0.74, 0.76)} & {0.77\newline(0.76, 0.78)} & {0.86\newline(0.85, 0.88)} & {0.71\newline(0.70, 0.72)} \\
tbats & {0.75\newline(0.74, 0.76)} & {0.72\newline(0.71, 0.73)} & {0.80\newline(0.78, 0.81)} & {0.73\newline(0.72, 0.74)} & {0.71\newline(0.69, 0.72)} & {0.76\newline(0.74, 0.77)} \\
\bottomrule
\end{tabular}

    \caption{Trending ratio $\acc$, positive trending ratio $\accp$, and negative trending ratio $\accm$ for the models with the exclusion of zero-containing points for the horizons 72 hours and seven days.}
    \label{tab:app-eda-trending-ratios}
\end{table}

\begin{figure}
    \centering
    \begin{subfigure}[t]{0.48\textwidth}
    \includegraphics{plots/ed_arrival/50_Cond_Prob_lag_3}
    \caption{Horizon three days}
    \end{subfigure}\hfill
    \begin{subfigure}[t]{0.48\textwidth}
    \includegraphics{plots/ed_arrival/50_Cond_Prob_lag_7}
    \caption{Horizon seven days}
    \end{subfigure}
    \caption{Conditional trending plots for the horizons three and seven days and the models with the best positive or negative trending ability. The plots of NBI-2 and Poisson-2 are indistinguishable.}
    \label{fig:app-eda-cond-prob}
\end{figure}

\begin{figure}
    \begin{subfigure}{0.32\textwidth}
    \tiny
    \begin{tabular}{lll}
\toprule
 & lag 3 d & lag 7 d \\
\midrule
Benchmark-1 & 0.1586 & 0.1761 \\
GBM-2 & 0.1590 & 0.1759 \\
NBI-2 & 0.1549 & 0.1717 \\
Poisson-2 & 0.1549 & 0.1714 \\
qreg-1 & 0.1679 & 0.1843 \\
\bottomrule
\end{tabular}

    \caption{Brier Scores for the different models and horizons.}\label{fig:app-eda-prob-brier}
    \end{subfigure}\hspace{0.01\textwidth}%
    \begin{subfigure}[t]{0.32\textwidth}
    \includegraphics{plots/ed_arrival/60_reliability_diagram_lag_3}
    \caption{Reliability diagram for horizon three days.}\label{fig:app-eda-prob-rel-3}
    \end{subfigure}\hspace{0.01\textwidth}%
    \begin{subfigure}[t]{0.32\textwidth}
    \includegraphics{plots/ed_arrival/60_reliability_diagram_lag_7}
    \caption{Reliability diagram for horizon seven days.}\label{fig:app-eda-prob-rel-7}
    \end{subfigure}
    \caption{Probabilistic trending evaluation for the models Benchmark-1, GBM-2, NBI-2, Poisson-2, and qreg-1 for the horizons three and seven days.}
    \label{fig:app-eda-prob}
\end{figure}

\subsubsection*{Discussion}

The trending ability is consistent for the two horizons, with the models' relative trending ability evolving similarly for the two horizons.
The models' trending ability is generally higher for the smaller horizon, but the differences are minor, and confidence intervals overlap.

The trending differs for all models for positive and negative predicted change directions.
While some models, such as GBM-2 and qreg-1, have the highest positive trending ratio, others, such as Poisson-2 and NBI-2, have the highest negative trending ratio.
Thus, the uncertainty of the model's predicted change has to be assessed differently based on the direction.

The results of the probabilistic trending evaluation endorse the point trending assessment and assign the best scores to NBI-2 and Poisson-2. 
The reliability diagrams show that they underestimate the fraction of increases slightly. 

The example shows that trending assessment is detached from standard point evaluation measures.
While the models with the lowest RMSE, NBI-2 and Poisson-2, also have a high trending ability, three models with below-average point evaluation measures, Benchmark-1, GBM-2, and qreg-1, have a high positive trending ability.


\subsection{Invasive and non-invasive blood pressure monitoring} \label{sec:application_measurement}

In the last briefer example, we consider the trending assessment of measurement data.
The data is from the MIMIC-III database, including various information on patients in critical care units of the Beth Israel Deaconess Medical Center in Boston (Massachusetts, USA, \cite{Johnson2016}).
The data is publicly available and also includes numerical measurement data such as heart rate, blood pressure, and oxygen saturation in a waveform database \citetext{\citealp{Moody2017}; available through \citealp{Goldberger2000}}.

For some patients, the data includes \ac{abp} and \ac{nbp} measurements.
While non-invasive blood pressure measurement methods are relatively gentle, they are less accurate than invasive methods.
For an overview of blood pressure measurement methods, see \citet{Saugel2014}.
For critical patients, changes in blood pressure can be crucial for the treatment.
Thus, trending assessment can be performed in addition to standard accuracy analysis~\citep[see, for example, ][]{Mostafa2020}.
Thus, we assess the trending ability of the non-invasive blood pressure measurements compared to the invasive blood pressure measurements.
The database contains 64,168 numerical data records.
One data record includes all numerical measurements for one patient.
The measured signals vary in length, frequency, and type of measurement.
Thus, only a subset of the data contains measurements of \ac{abp} and \ac{nbp} simultaneously.
2,548 include at least one measurement of systolic \ac{abp} and \ac{nbp} and 1,327 include at least one measurement of systolic \ac{abp} and \ac{nbp} at the same time; for the mean \ac{abp} and \ac{nbp}, the numbers are 2,605 and 1,516, respectively.

We consider the horizons of one minute, five minutes, and 15 minutes for the trending assessment, as those are typical intervals of NBP measurements.

\subsubsection*{Results}

Again, we exclude the smallest 10\% of absolute differences in trending assessment.
The resulting four-quadrant plots of the mean and systolic blood pressure measurements for the different horizons are shown in Figure~\ref{fig:app-mimic-4q}.
The number of points in the four-quadrant plot is smaller due to the restriction to data records with measurements of mean or systolic \ac{abp} and \ac{nbp} simultaneously for two consecutive times with the specified horizons.
Thus, we use the \ac{nbp} measurements as test method and the \ac{abp} measurements as gold standard.
For the systolic measurements, 290, 332, and 442 points are available for the horizons of one, five, and 15 minutes; for the mean measurements, 406, 430, and 542.

The trending ratios, including confidence intervals for the different horizons, are listed in Table~\ref{tab:app-mimic-trending-ratios}.
For the measurements with a horizon of one minute, the confidence intervals have lower bounds of 0.5 or slightly above.
For larger horizons, the trending ratio increases.
The difference between positive and negative trending ratios is small for all types and horizons, with overlapping confidence intervals.

Figure~\ref{fig:app-mimic-cond-prob} shows the conditional trending plots for the different horizons and the systolic and mean blood pressure measurements.
It becomes apparent that the systolic measurements have a higher trending ability than the mean measurement, except for small negative predicted changes.
This aligns with the trending ratios, but the confidence intervals overlap.

\begin{figure}
    \centering
    \includegraphics{plots/mimic/plot_4q}
    \caption{Four-quadrant plots for the different horizons and the systolic and mean blood pressure measurements. The upper row contains systolic measurements, and the lower row contains mean measurements. The columns contain the horizons one, five, and 15 minutes.}
    \label{fig:app-mimic-4q}
\end{figure}

\begin{table}
    \centering
    \begin{tabular}{l l p{0.2\textwidth} p{0.2\textwidth} p{0.2\textwidth}}
\toprule
Type & $l$ & $\mu^{l}$ & $\mu^{+, l}$ & $\mu^{-, l}$ \\
\midrule
Systolic & 1 & {0.55 (0.50, 0.60)} & {0.59 (0.52, 0.65)} & {0.58 (0.50, 0.66)} \\
Systolic & 5 & {0.63 (0.59, 0.68)} & {0.70 (0.64, 0.75)} & {0.62 (0.56, 0.69)} \\
Systolic & 15 & {0.69 (0.65, 0.73)} & {0.72 (0.66, 0.76)} & {0.74 (0.69, 0.79)} \\
Mean & 1 & {0.55 (0.51, 0.59)} & {0.62 (0.56, 0.68)} & {0.56 (0.50, 0.62)} \\
Mean & 5 & {0.59 (0.55, 0.64)} & {0.65 (0.59, 0.71)} & {0.62 (0.56, 0.68)} \\
Mean & 15 & {0.62 (0.58, 0.65)} & {0.65 (0.60, 0.70)} & {0.66 (0.61, 0.71)} \\
\bottomrule
\end{tabular}

    \caption{Trending ratios for the different horizons and the systolic and mean blood pressure measurements.}
    \label{tab:app-mimic-trending-ratios}
\end{table}

\begin{figure}
    \centering
    \begin{subfigure}[t]{.32\textwidth}
        \includegraphics{plots/mimic/cond_prob_diff_nbp_abp_lag1}
        \caption{Horizon one minute.}
    \end{subfigure}\hspace{0.01\textwidth}
    \begin{subfigure}[t]{.32\textwidth}
        \includegraphics{plots/mimic/cond_prob_diff_nbp_abp_lag5}
        \caption{Horizon five minutes.}
    \end{subfigure}\hspace{0.01\textwidth}
    \begin{subfigure}[t]{.32\textwidth}
        \includegraphics{plots/mimic/cond_prob_diff_nbp_abp_lag15}
        \caption{Horizon 15 minutes.}
    \end{subfigure}\hspace{0.01\textwidth}
    \caption{Conditional trending plot for the systolic and mean blood pressure measurements and the horizons one, five, and 15 minutes. }
    \label{fig:app-mimic-cond-prob}
\end{figure}




\subsubsection*{Discussion}

The four-quadrant plots contain a considerable number of extreme points.
Whether these points are due to measurement errors or extreme values is not distinguishable.
Some authors argue to exclude the measurements below the 10\%-quantile of the absolute differences and the points above the 90\%-quantile \citep[see][]{Critchley2010}.
We do not follow this approach here, as the extreme values are not necessarily measurement errors and could be particularly relevant.

The differences between positive and negative predicted changes are small in this example.
The positive and negative trending ratios have overlapping confidence intervals, and the conditional trending plots do not contain prominent deviations in the course.
This aligns with the four-quadrant plots, where no apparent asymmetry is visible.

The bootstrap confidence intervals are wide.
The width is around 0.1 for the trending ratio, while it gets up to 0.16 for the negative trending ratio for systolic measurement and the horizon of one minute.
Thus, more measurements would be of interest for a further trending assessment. 



\printbibliography

\appendix
\section{Data generation for Section~\ref{sec:trending}}\label{sec:app-trending-data-generation}
\section{Additional material on Section~\ref{sec:trending}}\label{sec:appendix-trending}

\subsection{Data generation for Section~\ref{sec:trending}}\label{subsec:app-trending-data-generation}

The first dataset is generated by sequentially generating $\diffx$ and $\diffy$.
First, the $\diffxt$ are sampled as a sum of a standard normal random number and a uniform random number on $(-10, 10)$:
\begin{equation*}
    \diffxt \sim N(0, 1) + U(-10, 10) \quad t = 1, \dots, T.
\end{equation*}
Subsequently, the $\diffy$ are simulated for a constant trending ratio $k$ by
\begin{equation*}
    \diffyt = \diffxt \cdot n_t * b_t,
\end{equation*}
where $n_t$ is a truncated normal distribution with mean 1 and standard deviation 0.5, truncated at 0 and a symmetric Bernoulli random variable with parameter $k$.
For a time-varying trending ratio, the parameter $k$ is modified to have a wave-shape over time, that is,
\begin{equation*}
    k_t = 0.75 + \sin(t / 365.25 \cdot 2 \pi) / 4,
\end{equation*}
where $T^\star$ is $T$ divided by the number of oscillations; in this case, 4.
For the asymmetric trending ratio, $k$ is a function of $\diffxt$,
\begin{equation*}
    k(x) = 0.5 + \min \left\{ \max \left\{ \frac{x + 5}{10}, 0  \right\} , 1 \right\} / 2.
\end{equation*}

In the second approach, $\diffyt$ and $\diffxt$ are modelled to be multivariate normal with mean 0 and covariance matrix
\begin{equation*}
    \Sigma = \begin{pmatrix} 4 & 3 \\ 3 & 4 \end{pmatrix}.
\end{equation*}
Thus, the conditional probability of trending can be calculated by a conditional normal distribution to
\begin{equation*}
    P(\diffyrv \diffxrv > 0 | \diffxrv = x) = \Phi \left( \frac{3}{4 \sqrt{7}} x \right),
\end{equation*}
where $\Phi$ is a standard normal \ac{cdf}.

The four-quadrant plots for the sample realizations of the data generation schemes are shown in Figure~\ref{fig:appendix_dgps}.

\begin{figure}
    \centering
    \begin{subfigure}{0.24\textwidth}
        \includegraphics{plots/illustrative_examples/appendix_4q_dgp1}
        \caption{Constant trending ratio.}
    \end{subfigure}\hspace{0.01\textwidth}
    \begin{subfigure}{0.24\textwidth}
        \includegraphics{plots/illustrative_examples/appendix_4q_dgp1_time}
        \caption{Time-varying trending ratio}
    \end{subfigure}\hspace{0.01\textwidth}
    \begin{subfigure}{0.24\textwidth}
        \includegraphics{plots/illustrative_examples/appendix_4q_dgp1_asym}
        \caption{Asymmetric trending ratio}
    \end{subfigure}\hspace{0.01\textwidth}
    \begin{subfigure}{0.24\textwidth}
        \includegraphics{plots/illustrative_examples/appendix_4q_dgp2}
        \caption{Second approach}
    \end{subfigure}
    \caption{Four-quadrant plots for sample realizations of the data generation schemes of Section~\ref{subsec:app-trending-data-generation}. Although the first and second plot differ over time, their difference is not discernible in the plots. The third data set's asymmetry is visible in the plot but the decrese of trending ability near 0 is not visible. }
    \label{fig:appendix_dgps}
\end{figure}

\subsection{Simulation study on bootstrapping confidence intervals}\label{subsec:app-trending-bootstrap}
We vary the number of available time steps $T$ to be a typical time-series value, such as 30 for daily data in a month, 52 for weekly data, 168, 365, 720, and 1024.
The considered datasets are outlined in Appendix~\ref{subsec:app-trending-data-generation}, the first dataset with asymmetric dependence.
In the calculations, the \verb|scipy| package's implementation of bootstrap confidence intervals is used~\parencite{Virtanen2020}.
The prescribed confidence level is 90 \%, and the number of bootstrap samples is $10,000$.
The share of confidence intervals covering the true values per method and $T$ are shown in Table~\ref{tab:trending_bootstrap}.
The true values of the accuracy are computed based on a dataset of size $10^8$, yielding 0.7501 and 0.7700 for the two datasets.
The computation times per method and dataset are shown in Figure~\ref{fig:trending_bootstrap_time}.
For the small sample sizes up to $T = 168$, only the \ac{bca} method keeps the confidence interval size and yields slightly wider confidence intervals.
The method's results do not differ for the larger sample sizes.
The computation time for the \ac{bca} method is slightly larger than for the other methods, but all methods have a moderate computation time.

\begin{table}
    \centering
    \begin{subtable}{.48\textwidth}
        \begin{tabular}{llll}
\toprule
 & percentile & basic & bca \\
\midrule
30 & 0.84 (0.249) & 0.86 (0.250) & 0.91 (nan) \\
52 & 0.89 (0.194) & 0.89 (0.193) & 0.89 (0.198) \\
168 & 0.91 (0.109) & 0.90 (0.109) & 0.90 (0.110) \\
365 & 0.90 (0.074) & 0.90 (0.074) & 0.90 (0.074) \\
720 & 0.90 (0.053) & 0.90 (0.053) & 0.90 (0.053) \\
1024 & 0.90 (0.044) & 0.90 (0.044) & 0.89 (0.044) \\
\bottomrule
\end{tabular}

        \caption{First dataset}
    \end{subtable}\hspace{0.02\textwidth}
    \begin{subtable}{.48\textwidth}
        \begin{tabular}{llll}
\toprule
 & percentile & basic & BCa \\
\midrule
30 & 0.87 (0.243) & 0.88 (0.242) & 0.92 (0.249) \\
52 & 0.87 (0.188) & 0.89 (0.188) & 0.90 (0.192) \\
168 & 0.89 (0.106) & 0.90 (0.106) & 0.90 (0.107) \\
365 & 0.90 (0.072) & 0.90 (0.072) & 0.90 (0.072) \\
720 & 0.90 (0.052) & 0.90 (0.052) & 0.90 (0.052) \\
1024 & 0.89 (0.043) & 0.90 (0.043) & 0.90 (0.043) \\
\bottomrule
\end{tabular}

        \caption{Second dataset}
    \end{subtable}
    \caption{Proportion of bootstrapping confidence intervals covering the true value of trending ratio per method and sample size $T$. The average width of the confidence interval is listed in brackets.}
    \label{tab:trending_bootstrap}
\end{table}

\begin{figure}
    \centering
    \begin{subfigure}{0.48\textwidth}
\includegraphics{plots/illustrative_examples/boxplot_comp_time_butterfly}
        \caption{First dataset}
    \end{subfigure}
    \begin{subfigure}{0.48\textwidth}
    \includegraphics{plots/illustrative_examples/boxplot_comp_time_normal}
        \caption{Second dataset}
    \end{subfigure}
    \caption{Boxplot of the computation time of the different bootstrapping method and data set sizes $T$. The computation time refers to bootstrapping one confidence interval based upon $10,000$ values. Each boxplot reflects $10,000$ samples. The \ac{bca} method takes slightly longer than the other two, but the difference is negligible.}
    \label{fig:trending_bootstrap_time}
\end{figure}



\section{Additional material on Section~\ref{sec:application-covid}}\label{sec:appendix-application-covid}


%\begin{figure}
%    \centering
%    \begin{subfigure}[t]{.48\textwidth}
%        \includegraphics{plots/covid_nowcast/20_kde_lag_1}
%        \caption{\ac{kde} for true values and nowcasts of lag 1.}
%    \end{subfigure}\hspace{0.01\textwidth}
%    \begin{subfigure}[t]{.48\textwidth}
%        \includegraphics{plots/covid_nowcast/20_kde_lag_7}
%        \caption{\ac{kde} for true values and nowcasts of lag 7.}
%    \end{subfigure}\hspace{0.01\textwidth}
%    \begin{subfigure}[t]{.48\textwidth}
%        \includegraphics{plots/covid_nowcast/20_kde_lag_14}
%        \caption{\ac{kde} for true values and nowcasts of lag 14.}
%    \end{subfigure}
%    \caption{\Ac{kde} for true values and nowcasts of lags 1, 7, and 14 to assess the distribution of values. \hl{AUSWERTUNG}. Exclusion areas based on the 10\%  quantile of absolute values are listed in Table~\ref{tab:app-covid-marginals}.}
%    \label{fig:app-covid-kde}
%\end{figure}


\begin{table}
    \centering
    \begin{tabular}{l l}
        \toprule
        Abbreviation & Nowcasting hub key \\
        \midrule
        EPI & Epiforecasts-independent \\
        ILM & ILM-prop \\
        KIT & KIT-simple\_nowcast \\
        LMU & LMU\_StaBLab-GAM\_nowcast \\
        RIVM & RIVM-KEW \\
        RKI & RKI-weekly\_report \\
        SU & SU-hier\_bayes \\
        SZ & SZ-hosp\_nowcast\\
        ENS-MEAN & NowcastHub-MeanEnsemble\\
        ENS-MED & NowcastHub-MedianEnsemble\\
        \bottomrule
    \end{tabular}
    \caption{Matching the abbreviation to the key in the nowcasting hub.
    Information on the models and references is listed in \citet[][Table 1]{Wolffram2023}.}
    \label{tab:app-covid-models}
\end{table}


\begin{table}
    \centering
    \tiny
    \begin{tabular}{llllllllll}
\toprule
 & $\widebar{x^{\Delta, 1}}$ & $\sigma_{x^{\Delta, 1}}$ & $q_{0.1} (x^{\Delta, 1})$ & $\widebar{x^{\Delta, 7}}$ & $\sigma_{x^{\Delta, 7}}$ & $q_{0.1} (x^{\Delta, 7})$ & $\widebar{x^{\Delta, 14}}$ & $\sigma_{x^{\Delta, 14}}$ & $q_{0.1} (x^{\Delta, 14})$ \\
\midrule
EPI & 72 & 520 & 45 & 37 & 1,411 & 78 & -62 & 1,976 & 145 \\
ILM & 40 & 281 & 26 & 144 & 1,457 & 103 & 147 & 2,357 & 140 \\
KIT & 24 & 355 & 50 & 112 & 1,306 & 171 & 92 & 1,965 & 265 \\
LMU & -48 & 285 & 27 & -6 & 1,180 & 124 & -109 & 1,947 & 168 \\
ENS-MEAN & 21 & 267 & 23 & 56 & 1,214 & 98 & -3 & 1,956 & 235 \\
ENS-MED & 20 & 259 & 24 & 28 & 1,207 & 101 & -52 & 1,955 & 186 \\
RIVM & -8 & 242 & 32 & -50 & 1,264 & 123 & -104 & 2,034 & 191 \\
RKI & 109 & 363 & 34 & 367 & 1,194 & 146 & 419 & 1,833 & 326 \\
SU & 43 & 376 & 47 & 23 & 1,391 & 181 & -68 & 2,127 & 264 \\
SZ & 24 & 201 & 27 & 123 & 1,155 & 185 & 105 & 1,889 & 242 \\
True & -21 & 263 & 27 & -86 & 1,238 & 127 & -109 & 2,194 & 284 \\
\bottomrule
\end{tabular}

    \caption{Analysis of the nowcast and true differences for the lags 1, 7, and 14 days.
    The column (1), $l=l$ shows the number of values greater than zero for lag $l$, $\sigma_{x^{\Delta, l}}$ the standard deviation, and $q_{0.1} (x^{\Delta, l})$ the 10\% quantile of the differences' absolute values.}
    \label{tab:app-covid-marginals}
\end{table}

\begin{table}
    \centering
    \begin{subtable}[t]{\textwidth}
        \begin{tabular}{l p{0.11\textwidth} p{0.11\textwidth} p{0.11\textwidth} p{0.11\textwidth} p{0.11\textwidth} p{0.11\textwidth}}
\toprule
 & $\mu^1$ & $\mu^{+, 1}$ & $\mu^{-, 1}$ & $\mu^1_{q_{0.1}}$ & $\mu^{+, 1}_{q_{0.1}}$ & $\mu^{-, 1}_{q_{0.1}}$ \\
\midrule
EPI & {0.68\newline(0.62, 0.74)} & {0.64\newline(0.55, 0.72)} & {0.73\newline(0.63, 0.81)} & {0.69\newline(0.63, 0.75)} & {0.64\newline(0.55, 0.73)} & {0.75\newline(0.65, 0.82)} \\
ILM & {0.73\newline(0.67, 0.79)} & {0.67\newline(0.59, 0.76)} & {0.82\newline(0.73, 0.88)} & {0.74\newline(0.68, 0.79)} & {0.68\newline(0.60, 0.77)} & {0.82\newline(0.72, 0.88)} \\
KIT & {0.62\newline(0.55, 0.68)} & {0.58\newline(0.49, 0.67)} & {0.65\newline(0.56, 0.73)} & {0.62\newline(0.56, 0.69)} & {0.59\newline(0.51, 0.67)} & {0.66\newline(0.57, 0.74)} \\
LMU & {0.66\newline(0.60, 0.72)} & {0.66\newline(0.57, 0.75)} & {0.66\newline(0.57, 0.73)} & {0.66\newline(0.59, 0.71)} & {0.66\newline(0.56, 0.75)} & {0.66\newline(0.57, 0.73)} \\
ENS-MEAN & {0.81\newline(0.75, 0.86)} & {0.76\newline(0.68, 0.84)} & {0.88\newline(0.81, 0.93)} & {0.81\newline(0.75, 0.86)} & {0.76\newline(0.68, 0.83)} & {0.88\newline(0.81, 0.94)} \\
ENS-MED & {0.75\newline(0.68, 0.80)} & {0.69\newline(0.61, 0.77)} & {0.81\newline(0.73, 0.89)} & {0.75\newline(0.69, 0.81)} & {0.69\newline(0.60, 0.77)} & {0.83\newline(0.74, 0.90)} \\
RIVM & {0.77\newline(0.72, 0.82)} & {0.75\newline(0.66, 0.83)} & {0.79\newline(0.71, 0.85)} & {0.78\newline(0.72, 0.83)} & {0.75\newline(0.66, 0.83)} & {0.81\newline(0.73, 0.87)} \\
RKI & {0.74\newline(0.68, 0.79)} & {0.67\newline(0.59, 0.74)} & {0.88\newline(0.79, 0.93)} & {0.74\newline(0.68, 0.79)} & {0.66\newline(0.58, 0.73)} & {0.87\newline(0.78, 0.93)} \\
SU & {0.71\newline(0.65, 0.77)} & {0.66\newline(0.57, 0.74)} & {0.78\newline(0.69, 0.85)} & {0.72\newline(0.66, 0.78)} & {0.67\newline(0.58, 0.75)} & {0.79\newline(0.70, 0.87)} \\
SZ & {0.74\newline(0.68, 0.80)} & {0.68\newline(0.60, 0.76)} & {0.82\newline(0.73, 0.90)} & {0.74\newline(0.68, 0.80)} & {0.68\newline(0.60, 0.76)} & {0.82\newline(0.73, 0.90)} \\
\bottomrule
\end{tabular}

    \caption{1 day.}
    \end{subtable}
    \begin{subtable}[t]{\textwidth}
        \begin{tabular}{l p{0.11\textwidth} p{0.11\textwidth} p{0.11\textwidth} p{0.11\textwidth} p{0.11\textwidth} p{0.11\textwidth}}
\toprule
 & $\mu^14$ & $\mu^{+, 14}$ & $\mu^{-, 14}$ & $\mu^14_{q_{0.1}}$ & $\mu^{+, 14}_{q_{0.1}}$ & $\mu^{-, 14}_{q_{0.1}}$ \\
\midrule
EPI & {0.83\newline(0.78, 0.87)} & {0.79\newline(0.70, 0.85)} & {0.87\newline(0.80, 0.92)} & {0.85\newline(0.80, 0.89)} & {0.81\newline(0.73, 0.87)} & {0.90\newline(0.83, 0.95)} \\
ILM & {0.86\newline(0.81, 0.90)} & {0.78\newline(0.70, 0.85)} & {0.96\newline(0.90, 0.99)} & {0.87\newline(0.82, 0.91)} & {0.80\newline(0.71, 0.86)} & {0.96\newline(0.90, 0.99)} \\
KIT & {0.81\newline(0.75, 0.86)} & {0.76\newline(0.67, 0.83)} & {0.87\newline(0.79, 0.92)} & {0.82\newline(0.76, 0.86)} & {0.76\newline(0.68, 0.84)} & {0.88\newline(0.81, 0.93)} \\
LMU & {0.88\newline(0.83, 0.92)} & {0.85\newline(0.77, 0.91)} & {0.91\newline(0.85, 0.95)} & {0.89\newline(0.85, 0.93)} & {0.87\newline(0.79, 0.92)} & {0.91\newline(0.85, 0.95)} \\
ENS-MEAN & {0.83\newline(0.77, 0.87)} & {0.77\newline(0.69, 0.84)} & {0.89\newline(0.83, 0.95)} & {0.84\newline(0.79, 0.88)} & {0.78\newline(0.70, 0.84)} & {0.91\newline(0.84, 0.95)} \\
ENS-MED & {0.84\newline(0.79, 0.89)} & {0.79\newline(0.70, 0.86)} & {0.90\newline(0.83, 0.95)} & {0.85\newline(0.80, 0.90)} & {0.80\newline(0.71, 0.86)} & {0.91\newline(0.84, 0.95)} \\
RIVM & {0.85\newline(0.80, 0.89)} & {0.82\newline(0.74, 0.88)} & {0.88\newline(0.80, 0.93)} & {0.85\newline(0.80, 0.90)} & {0.83\newline(0.75, 0.89)} & {0.88\newline(0.80, 0.93)} \\
RKI & {0.81\newline(0.75, 0.86)} & {0.71\newline(0.63, 0.77)} & {0.98\newline(0.93, 1.00)} & {0.81\newline(0.75, 0.86)} & {0.71\newline(0.63, 0.78)} & {1.00\newline(nan, nan)} \\
SU & {0.88\newline(0.83, 0.92)} & {0.84\newline(0.76, 0.90)} & {0.92\newline(0.86, 0.96)} & {0.89\newline(0.85, 0.93)} & {0.85\newline(0.77, 0.91)} & {0.94\newline(0.88, 0.97)} \\
SZ & {0.82\newline(0.77, 0.87)} & {0.76\newline(0.68, 0.83)} & {0.90\newline(0.83, 0.94)} & {0.83\newline(0.78, 0.88)} & {0.78\newline(0.69, 0.85)} & {0.90\newline(0.83, 0.94)} \\
\bottomrule
\end{tabular}

        \caption{14 days.}
    \end{subtable}
    \caption{Trending ratio $\acc$, positive trending ratio $\accp$, and negative trending ratio $\accm$ for the models with and without exclusion areas for the lag 1 and 14 days. The exclusion areas are rectangles centered on the zero points with a width and height of 10\% of the quantile of the absolute values of nowcast and true values. }
    \label{tab:app-covid-trending-ratios-lag-1-14}
\end{table}



\begin{figure}
\centering
\includegraphics{plots/covid_nowcast/30_4q_plots}
\caption{Four-quadrant plots for the models ILM, RIVM, RKI, and ENS-MEAN and the lags of one, seven, and 14 days. The spread in both directions increases with the lag. }
\label{fig:app-covid-4q}
\end{figure}


\begin{figure}
    \centering
%    \includegraphics{}
    \begin{subfigure}[t]{.48\textwidth}
    \includegraphics{plots/covid_nowcast/40_cond_prob_lag_1}
    \caption{Conditional trending plot for lag 1.}\label{fig:app-covid-cond-prob-1}
    \end{subfigure}\hfill
    \begin{subfigure}[t]{.48\textwidth}
    \includegraphics{plots/covid_nowcast/40_cond_prob_lag_14}
    \caption{Conditional trending plot for lag 1.}\label{fig:app-covid-cond-prob-14}
    \end{subfigure}
    \begin{subfigure}[t]{.48\textwidth}
    \includegraphics{plots/covid_nowcast/40_acc_eps_lag_1}
    \caption{Trending ratio over exclusion area size in $\diffx$ for lag 1.}\label{fig:app-covid-trending-ratio-1}
    \end{subfigure}\hfill
    \begin{subfigure}[t]{.48\textwidth}
    \includegraphics{plots/covid_nowcast/40_acc_eps_lag_14}
    \caption{Trending ratio over exclusion area size in $\diffx$ for lag 1.}\label{fig:app-covid-trending-ratio-14}
    \end{subfigure}
    \caption{Conditional trending plot and trending ratio over exclusion area for the nowcasts of the seven-day hospitalization rate ILM, RKI, RIVM, and ENS-MED for the lag seven days.}
    \label{fig:app-covid-cond-prob-trending-ratio-1-14}
\end{figure}


\begin{figure}
    \centering
    \begin{subfigure}[t]{.48\textwidth}
        \includegraphics{plots/covid_nowcast/60_reliability_diagram_lag_7}
        \caption{Reliability diagram for horizon seven days.} \label{fig:app-covid-reliability-7}
    \end{subfigure}\hfill
    \begin{subfigure}[t]{.48\textwidth}
        \includegraphics{plots/covid_nowcast/60_reliability_diagram_lag_14}
        \caption{Reliability diagram for horizon 14 days.} \label{fig:app-covid-reliability-14}
    \end{subfigure}
    \begin{subfigure}{\textwidth}
        \includegraphics{plots/covid_nowcast/70_prob_hist}
        \caption{Count histogram of the predicted probabilities for the horizon one, seven, and 14 days.} \label{fig:app-covid-prob-hist}
    \end{subfigure}
    \caption{The reliability diagram for the models ILM, RIVM, RKI, and ENS-MED for the horizon seven and 14 days.
    Additionally, the count of predicted probabilities for the horizons is shown.
        The reliability diagram bins are chosen according to the empirical quantiles of the predicted probabilities.
    As the models issue small or large probabilities of increase for the higher horizons, little information on the accuracy of moderate probability predictions is available.}
    \label{fig:}
\end{figure}




\end{document} 