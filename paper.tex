\documentclass{article}

% Language setting
% Replace `english' with e.g. `spanish' to change the document language
\usepackage[english]{babel}
\usepackage[utf8]{inputenc} % für Umlaute
\usepackage{fontenc}

% Set page size and margins
% Replace `letterpaper' with`a4paper' for UK/EU standard size
\usepackage[a4paper,top=2cm,bottom=2cm,left=3cm,right=3cm,marginparwidth=1.75cm]{geometry}

%%%% ---- Useful packages

%%% --- Figures and Graphics
\usepackage{graphicx}
\usepackage{subcaption}

%%% --- Colors
\usepackage{color}
\usepackage[dvpinames]{xcolor}

%%% --- Tabulars
\usepackage{booktabs}

%%% --- Mathematics
\usepackage{amsmath,amsfonts,amssymb,amsthm,mathtools}
\usepackage{bm} % bold mathematics
\usepackage{bbm} % for blackboard 1
\usepackage[linesnumbered,lined,algo2e,figure,boxed]{algorithm2e}
\usepackage[short]{optidef} % for aligning linear programs

%%% --- Other
\usepackage[colorlinks=true, allcolors=blue]{hyperref}
%\usepackage[hidelinks]{hyperref}
\usepackage[style=authoryear,backend=bibtex]{biblatex}
%\addbibresource{library.bib}
\usepackage{csquotes}
\usepackage[shortcuts]{extdash} % For Hyphens in English
%\usepackage{acro}
% \DeclareAcronym{pdf}{short=PDF, long={probability distribution function}}

%%% --- Commenting, Todonotes
\usepackage{comment, todonotes}
\newcommand{\ar}{$\rightarrow$}
\presetkeys{todonotes}{backgroundcolor=yellow!30, bordercolor=yellow!50, linecolor=yellow!50, figwidth=\textwidth}{}
\newcommand{\hl}[1]{\textcolor{Aquamarine}{#1}}

%%%% ---- Own Commands
%%% --- Theorem Settings
\theoremstyle{plain}% Theorem-like structures provided by amsthm.sty
\newtheorem{theorem}{Theorem}
\newtheorem{exa}{Example}
\newtheorem{rem}{Remark}
\newtheorem{proposition}{Proposition}
\newtheorem{lemma}{Lemma}
\newtheorem{corollary}{Corollary}

\theoremstyle{definition}
\newtheorem{definition}{Definition}
\newtheorem{remark}{Remark}
\newtheorem{example}{Example}

%%% --- Math Commands

\newcommand{\lag}[1][l]{\Delta_{#1}}
\DeclarePairedDelimiter{\abs}\lvert\rvert

%%% --- Other Commands
\renewcommand{\arraystretch}{1.2}





\title{Trending in Nowcasting}
\author{Oliver Grothe, Bolin Liu, Jonas Rieger}

\begin{document}
\maketitle

%\begin{abstract}
%Your abstract.
%\end{abstract}


\begin{itemize}
    \item Aktualität/Wichtigkeit herausstellen: 
    \begin{itemize}
        \item zwei Nowcasts mit gleichem MSE, aber unterschiedlichem Trending $\rightarrow$ Vergleich zwischen Nowcasts
    \end{itemize}
    \item Einordnung in Literatur
    \item Contributions
    \begin{itemize}
        \item Argue, why trending is important and why other measures do not \enquote{detect} it
        \item (Formalisation of Trending Ability)
        \item Evaluation and review of existing Approaches (both measures and graphical assessment)
        \item Development of new measure and graphical method to assess trending for measurements, nowcasts and forecasts
        \item Application to various data examples from practice
        \item Publish ready-to-use code
    \end{itemize}
    \item Überblick über Paper
\end{itemize}


\section{Notation}
\begin{itemize}
  \item Sei $T = \{1, 2, \dots\}$ die Zeitindexmenge, wobei für jeden Zeitpunkt $t \in T$ eine Realisierung vorliegt (die ggf. im Nachhinein veröffentlicht wird).
  \item Sei $(Y_t)_{t \in T}$ die Zeitreihe der Realisierungen, dabei steht $t$ für den Zeitpunkt, auf den sich der Wert bezieht, nicht den Veröffentlichungszeitpunkt.
  \item Sei $K = \{1, 2, \dots\}$ die Menge der Nowcaster.
	\begin{itemize}
	  \item Sei $X_{t \lvert \tau}^k (k \in K, t \in T, \tau \in T_t^k)$ der Nowcast von $k$ bezüglich des Zeitpunktes $t$, der am Zeitpunkt $\tau$ \textbf{veröffentlicht} wird (oder: berechnet / dessen Informationen sich auf Zeitpunkt $\tau$ beziehen).
	  \item Sei $g: \mathbb{R}^{\lvert T_t^k \lvert} \rightarrow \mathbb{R}$ eine Aggregationsfunktion, die alle Nowcasts bezüglich eines Zeitpunkts zu einem Nowcast zusammenfasst ($X_t^k \coloneqq g((X_{t \lvert \tau}^k)_{\tau \in T_t^k})$). Falls jeweils nur ein Nowcast veröffentlich wird, wähle $g(x) = x$.
	  \item Sei $(X_t^k)_{t \in T}$ die Zeitreihe der aggregierten Nowcasts.
	\end{itemize}
  \item Sei $\lag[l]$ der lag-$l$-Operator, der für eine Zeitreihe $(Z_t)_{t \in T}$ definiert wird durch 
		\begin{equation}
			\lag[l]Z_t = Z_{t+l} - Z_t \quad (t \in T)
		\end{equation} 
  \item Der Anteil an konkordanten Punkten für Realisierungen $y = (y_1, y_2, \dots, y_n)$ und $(x_1, x_2, \dots, x_n)$ bezüglich des Lag $l$ ergibt sich dann durch
	\begin{equation}
  		k (x, y; l, \epsilon) = \frac{\sum_{t}^{n-l} k^s (\lag y_t, \lag x_t; l, \epsilon)}{\sum_{t}^{n-l} k^\epsilon (\lag y_t, \lag x_t)}
	\end{equation}
	wobei $k^s$ die Indikatorfunktion für die Konkordanz zweier Werte außerhalb der exclusion area ist \todo{stimmt Konkordanz hier?}
	\begin{equation}
  		k^s (x, y; \epsilon) \coloneqq \begin{cases}
  			1 &, \text{falls}\ xy > 0\ \text{und}\ k^\epsilon(x, y; \epsilon) = 1\\
  			0 &, \text{sonst}
  		\end{cases}
	\end{equation}
	und $k^\epsilon$ die Indikatorfunktion für Punkte außerhalb der exclusion area ist
	\begin{equation}
  		k^\epsilon (x, y; \epsilon) \coloneqq \begin{cases}
  			1 &, \text{falls} \ \abs{x} > \epsilon \text{oder} \ \abs{y} > \epsilon \\
  			0 &, \text{sonst}
  		\end{cases}
	\end{equation}
\end{itemize}

		
\section{Statistische Modellierung}

\begin{itemize}
  \item Sei $l$ der zeitliche Abstand, mit dem Realisationen verfügbar werden
  \item $Y_{t + l} = Y_t + \lag Y_t$
  \item Die Nowcast schätzen (fehlerbehaftet) $\lag Y_t$ mithilfe von Daten, die zum Zeitpunkt $\tau \in T_t^k$ verfügbar sind, und dem neuesten, bekannten Wert $Y_t$:
	\begin{equation}
  		X_{t+l \lvert \tau}^k = Y_t + (\widehat{\lag Y_t})_{t \lvert \tau \in T_t^k}^k,
	\end{equation}
	\todo{Stimmt das so?}
	\todo{Ist festes $l$ zu starke Annahme?}
	und damit 
		\begin{equation}
  			\lag Y_t = (\widehat{\lag Y_t})_{t \lvert \tau \in T_t^k}^k + \varepsilon.
		\end{equation}


\end{itemize}

\section{Trending}

\begin{itemize}
  \item \enquote{Schwaches} Trending für lag $l$:
   \begin{equation}
  	P(\lag Y_t \widehat{(\lag Y_t)^k} > 0 \:\lvert\: \abs{\lag Y_t} > \epsilon) > P(\lag Y_t \widehat{(\lag Y_t)^k} < 0 \:\lvert\: \abs{\lag Y_t} > \epsilon)
   \end{equation}

\end{itemize}




%\printbibliography


\end{document}