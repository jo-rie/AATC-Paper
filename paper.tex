\documentclass{article}

% Language setting
% Replace `english' with e.g. `spanish' to change the document language
\usepackage[english]{babel}
\usepackage[utf8]{inputenc} % für Umlaute
\usepackage{fontenc}

% Set page size and margins
% Replace `letterpaper' with`a4paper' for UK/EU standard size
\usepackage[a4paper,top=2cm,bottom=2cm,left=3cm,right=3cm,marginparwidth=1.75cm]{geometry}

%%%% ---- Useful packages

%%% --- Figures and Graphics
\usepackage{graphicx}
\usepackage{subcaption}
\usepackage{tikz}

%%% --- Colors
\usepackage[]{xcolor}

%%% --- Tabulars
\usepackage{booktabs}

%%% --- Mathematics
\usepackage{amsmath,amsfonts,amssymb,amsthm,mathtools}
\usepackage{bm} % bold mathematics
\usepackage{bbm} % for blackboard 1
\usepackage[linesnumbered,lined,algo2e,figure,boxed]{algorithm2e}
\usepackage[short]{optidef} % for aligning linear programs

%%% --- Other
\usepackage[colorlinks=true, allcolors=blue]{hyperref}
%\usepackage[hidelinks]{hyperref}
\usepackage[style=authoryear,backend=bibtex]{biblatex}
%\addbibresource{library.bib}
\usepackage{csquotes}
\usepackage[shortcuts]{extdash} % For Hyphens in English
%\usepackage{acro}
% \DeclareAcronym{pdf}{short=PDF, long={probability distribution function}}

%%% --- Commenting, Todonotes
\usepackage{comment, todonotes}
\newcommand{\ar}{$\rightarrow$}
\presetkeys{todonotes}{backgroundcolor=yellow!30, bordercolor=yellow!50, linecolor=yellow!50, figwidth=\textwidth}{}
\newcommand{\hl}[1]{\textcolor{Aquamarine}{#1}}

%%%% ---- Own Commands
%%% --- Theorem Settings
\theoremstyle{plain}% Theorem-like structures provided by amsthm.sty
\newtheorem{theorem}{Theorem}
\newtheorem{exa}{Example}
\newtheorem{rem}{Remark}
\newtheorem{proposition}{Proposition}
\newtheorem{lemma}{Lemma}
\newtheorem{corollary}{Corollary}

\theoremstyle{definition}
\newtheorem{definition}{Definition}
\newtheorem{remark}{Remark}
\newtheorem{example}{Example}

%%% --- Math Commands

\newcommand{\lag}[1][l]{\Delta_{#1}}
\DeclarePairedDelimiter{\abs}\lvert\rvert
\DeclareMathOperator{\sign}{sign}
\newcommand{\tmax}{\bar{t}}
\newcommand{\ind}{\mathbbm{1}}

%%% --- Other Commands
\renewcommand{\arraystretch}{1.2}

%%% --- ToDo List
\usepackage{enumitem}
\newlist{todolist}{itemize}{2}
\setlist[todolist]{label=$\square$}
\usepackage{pifont}
\newcommand{\cmark}{\ding{51}}%
\newcommand{\xmark}{\ding{55}}%
\newcommand{\done}{\rlap{$\square$}{\raisebox{2pt}{\large\hspace{1pt}\cmark}}%
\hspace{-2.5pt}}
\newcommand{\wontfix}{\rlap{$\square$}{\large\hspace{1pt}\xmark}}

\title{Trending in Nowcasting}
\author{Oliver Grothe, Bolin Liu, Jonas Rieger}

\begin{document}
\maketitle

%\begin{abstract}
%Your abstract.
%\end{abstract}

\section{ToDos}


\begin{todolist}
\item Differenz gegen was? (alte Realisierung oder alter Nowcast) "Problem": Nowcaster, der schon lange far off liegt messe ich gegen das falsche? Wie lange weiß er den heutigen Wert schon? Gegeben, seinem Wissenstand: Ab wo er es kennt gegen $y$, davor gegen $x$.
\item was glaubt er heute - (was glaubt er heute von gestern?)
\item "Irrelevante Änderung" statt Rauschen
\item Berechnung $\sigma$: Theoretisch aus Daten vs. Gefilterten Daten; Aus Wendepunkten; einfach mal durchlaufen lassen
\item Rauschen-Schätzer berechnen für neue Deltas; verschiedene Sigmas
\item Sigma bestimmen: einmal aus Plot von $k$ über verschiedene $\sigma$, einmal über Gefilterte vs. reale Daten
\item $\varepsilon$ für beide Komponenten oder nur für eine? Unterschiedliche $\sigma$ für $X$ und $Y$? $\sigma$ nur auf $Y$ anwenden.
\end{todolist}



\begin{itemize}
    \item Aktualität/Wichtigkeit herausstellen: 
    \begin{itemize}
        \item zwei Nowcasts mit gleichem MSE, aber unterschiedlichem Trending $\rightarrow$ Vergleich zwischen Nowcasts
    \end{itemize}
    \item Einordnung in Literatur
    \item Contributions
    \begin{itemize}
        \item Argue, why trending is important and why other measures do not \enquote{detect} it
        \item (Formalisation of Trending Ability)
        \item Evaluation and review of existing Approaches (both measures and graphical assessment)
        \item Development of new measure and graphical method to assess trending for measurements, nowcasts and forecasts
        \item Application to various data examples from practice
        \item Publish ready-to-use code
    \end{itemize}
    \item Überblick über Paper
\end{itemize}


\section{Notation}
\begin{itemize}
  \item Sei $T = \{1, 2, \dots\}$ die Zeitindexmenge, wobei für jeden Zeitpunkt $t \in T$ eine Realisierung vorliegt (die ggf. im Nachhinein veröffentlicht wird).
  \item Sei $(Y_t)_{t \in T}$ die Zeitreihe der Realisierungen, dabei steht $t$ für den Zeitpunkt, auf den sich der Wert bezieht, nicht den Veröffentlichungszeitpunkt.
  \item Sei $K = \{1, 2, \dots\}$ die Menge der Nowcaster.
	\begin{itemize}
	  \item Sei $X_{t \lvert \tau}^k (k \in K, t \in T, \tau \in T_t^k)$ der Nowcast von $k$ bezüglich des Zeitpunktes $t$, der am Zeitpunkt $\tau$ \textbf{veröffentlicht} wird (oder: berechnet / dessen Informationen sich auf Zeitpunkt $\tau$ beziehen).
	  \item Sei $g: \mathbb{R}^{\lvert T_t^k \lvert} \rightarrow \mathbb{R}$ eine Aggregationsfunktion, die alle Nowcasts bezüglich eines Zeitpunkts zu einem Nowcast zusammenfasst ($X_t^k \coloneqq g((X_{t \lvert \tau}^k)_{\tau \in T_t^k})$). Falls jeweils nur ein Nowcast veröffentlich wird, wähle $g(x) = x$.
	  \item Sei $(X_t^k)_{t \in T}$ die Zeitreihe der aggregierten Nowcasts.
	\end{itemize}
  \item Sei $\lag[l]$ der lag-$l$-Operator, der für eine Zeitreihe $(Z_t)_{t \in T}$ definiert wird durch 
		\begin{equation}
			\lag[l]Z_t = Z_{t+l} - Z_t \quad (t \in T)
		\end{equation} 
  \end{itemize}

\begin{figure}
	\centering
\begin{tikzpicture}[scale=2]
\newcommand{\maxaxis}{1.5}
% Draw quadrants with light colors
\fill[green!30] (-\maxaxis,-\maxaxis) rectangle (0,0); % Quadrant B
\fill[red!30] (0,-\maxaxis) rectangle (\maxaxis,0); % Quadrant D
\fill[red!50] (-\maxaxis,0) rectangle (0,\maxaxis); % Quadrant C
\fill[green!50] (0,0) rectangle (\maxaxis,\maxaxis); % Quadrant A

% Quadrant borders
\draw[thick, ->] (-\maxaxis,0) -- (\maxaxis,0); % x-axis
\draw[thick, ->] (0,-\maxaxis) -- (0,\maxaxis); % y-axis
%\draw (0,0) node[below left] {0};
\draw (\maxaxis,\maxaxis) node[below left] {A};
\draw (\maxaxis,-\maxaxis) node[above left] {D};
\draw (-\maxaxis,\maxaxis) node[below right] {C};
\draw (-\maxaxis,-\maxaxis) node[above right] {B};

% Random points
%\foreach \x/\y/\label in {0.6/0.8/z_1, -0.4/0.5/z_2, 0.3/-0.7/z_3, -0.8/-0.2/z_4} {
%  \coordinate (point) at (\x, \y);
%  \fill (point) circle (1.2pt);
%  \node[above right] at (point) {$\label = (\label^x, \label^y)$};
%}

% Axis labels
\draw (\maxaxis,0) node[right] {$\lag y$};
\draw (0,\maxaxis) node[above] {$\lag x$};

\end{tikzpicture}
\caption{Die Quadranten eines 4Q-Plots. In den grünen Quadranten A und B stimmt die Richtung der Veränderung von $\lag x$ und $\lag y$ überein, in den roten C und D nicht.}
\label{fig:quadranten}
\end{figure}


\section{Trending-Problem}


Gegeben seien zwei Zeitreihen $y_1, \dots, y_{\tmax}$ und $x_1, \dots, x_{\tmax}$ aus Realisierungen beziehungsweise Nowcasts zu den Zeitpunkten $1, \dots, \tmax$. 
Hohes Trending liegt vor, wenn der Nowcast die Veränderung der Realisierungen häufig richtig angibt, d.h. der Nowcast höhere Werte angibt, wenn die Realisierung gestiegen ist und niedrigere Werte, wenn sie gefallen ist. 
Mathematisch lässt sich die Wahrscheinlichkeit für eine Entwicklung in die gleiche Richtung bezüglich eines Lags $l$ darstellen als
\begin{equation}
  P((\lag Y) (\lag X) > 0). \label{eq:concordance}
\end{equation}


Eine Möglichkeit, Gleichung \eqref{eq:concordance} zu schätzen, ist
\begin{equation}
  \frac{1}{\tmax - l} \sum_{t = 1}^{\tmax - l} \ind\{{(\lag y_t) (\lag x_t) > 0}\}
\end{equation}

\subsection{Direkte Integration der Rauschen-Betrachtung in die obige Formel}
Dabei wird nicht beachtet, ob Punkte \enquote{mittig} in den Quadranten liegen oder weit am Rand, deswegen könnte das Maß folgendermaßen variiert werden.
Zur einfacheren Notation im Folgenden lassen wir $\lag$ weg.
Wir nehmen an, dass die echten Veränderungen $X$ und $Y$ fehlerbehaftet gemessen werden und nur

\begin{align}
	X_b &= X + \epsilon_X \quad \text{beziehungsweise} \\
	Y_b &= Y + \epsilon_Y 
\end{align}

beobachtet werden können. 
Wir bezeichnen mit $Z_b = (X_b, Y_b)$, $Z = ( X, Y)$ und die Quadranten gemäß Schaubild Abbildung~\ref{fig:quadranten}. 
Wir treffen zwei Annahmen: die Fehler $\epsilon_X$ und $\epsilon_Y$ sind unabhängig und identisch normalverteilt mit Erwartungswert 0 und Varianz $\sigma^2$.
Dann ist die Wahrscheinlichkeit, dass ein Punkt tatsächlich konkordant ist, d.h. $xy > 0$, für gemessene Werte $z_b$
\begin{align*}
	k(z_b) &= P(Z \in A | X_b = x_b, Y_b = y_b) + P(Z \in B | X_b = x_b, Y_b = y_b) \\
	&= (1 - P(Z \in B | X_b = x_b, Y_b = y_b) - P(Z \in C |X_b = x_b, Y_b = y_b) \\
        &\phantom{==} - P(Z \in D |X_b = x_b, Y_b = y_b)) + P(Z \in B | X_b = x_b, Y_b = y_b) \\
	&= 1 - 2 P(Z \in C | X_b = x_b, Y_b = y_b) - 2 P(Z \in D | X_b = x_b, Y_b = y_b) \\
	&= 1 - 2 P(Y \leq 0 | X_b = x_b, Y_b = y_b) - 2 P(X \leq 0 | X_b = x_b, Y_b = y_b) + 4 P(Y \leq 0, X \leq 0| z_b) \\
	&= 1 - 2 P(y_b - \epsilon_Y \leq 0) - 2 P(x_b - \epsilon_X \leq 0) + 4 P(y_b - \epsilon_Y \leq 0, x_b -  \epsilon_X \leq 0) \\
	&= 1 - 2 P(- \epsilon_Y \leq - y_b) - 2 P(- \epsilon_X \leq -x_b) + 4 P(- \epsilon_Y \leq - y_b) P(-  \epsilon_X \leq - x_b) \\
	&= 1 - 2 \Phi(- \tfrac{y_b}{\sigma}) - 2 \Phi(- \tfrac{x_b}{\sigma}) + 4 \Phi(- \tfrac{y_b}{\sigma}) \Phi(- \tfrac{x_b}{\sigma})
\end{align*}

\todo{Die Wahrscheinlichkeit, dass ein Punkt $(x_b,y_b)$ konkordant ist, ist gleich $P(Z\in A|X_b=x_b,Y_b=y_b)+P(Z\in B|X_b=x_b,Y_b=y_b)$}

Für $n$ Realisierungen und Nowcasts $\lag z = (\lag y, \lag x)$ ergibt sich das $\bar{k}$ als arithmetisches Mittel der individuellen $k$:

\begin{equation}
  \bar{k}(\lag y, \lag x) = \tfrac{1}{n} \sum_{t=1}^n k(\lag z)
\end{equation}


Abbildung \ref{fig:weighting} zeigt die resultierende Gewichtung der Punkte für einen Fehler von $\sigma = 1$.
Die Abhängigkeit von $\sigma$ für das arithmetische Mittel ist in Abbildung \ref{fig:mean_k_over_sigma} dargestellt.

\begin{figure}
  	\centering
  	\begin{subfigure}{.48\textwidth}
	    \includegraphics[width = \textwidth]{plots/weight_by_probability.pdf}
		\caption{Gewichtung mit der Wahrscheinlichkeit eines normalverteilten Messfehlers für $\sigma = 1$}
  		\label{fig:weighting}	
  	\end{subfigure}
	\begin{subfigure}{.48\textwidth}
  		\includegraphics[width=\textwidth]{plots/mean_k_over_sigma.pdf}
  		\caption{Mittleres $k$ über $1000$ Realisierungen in Abhängigkeit von $\sigma$.}
	\end{subfigure}
	\caption{Plots zum Maß $k$.}
  	\label{fig:k}
\end{figure}

\subsection{Wie könnte man $\sigma$ bestimmen?}

Vermutung: Wenn man $\bar{k}(z_b)$ für gegebenen Vektor $z_b^1, \dots$ für verschiedene $\sigma$ berechnet, hat $\bar{k}(z_b)$ beim "echten" $\sigma$ einen Wendepunkt.

Ist im Umkehrschluss die Vorgehensweise oben eine Möglichkeit, Rauschen in den Daten zu bestimmen?

\subsection{Alternative Vorgehensweise}
Die Formel (\ref{eq:concordance}) ist eine einfache und effiziente Methode zur Messung der Konkordanz, aber die Effektivität nimmt schnell ab, wenn die Daten verrauscht sind. Daher liegt es nahe, die verwendeten Daten zunächst zu entrauschen, bevor ein Trendfähigkeitsmaß auf sie angewendet wird. 

Ob und wie stark die Schätzung eines Nowcasters verrauscht ist, ist eine Eigenschaft des Nowcasters. Starkes Rauschen bedeutet, dass der Schätzer nicht robust ist. Es ist daher nicht notwendig, die Werte des Nowcasters zu entrauschen. Die realen Daten der zu schätzenden Größe sollten jedoch mit einer geeigneten Technik entrauscht werden oder die Verteilungsinformation des Rauschens (z. B. in Form einer Normalfehlerverteilung) sollte aus den Daten gelernt werden. Die entrauschten Daten bzw. die allgemeine Verteilung des wahren Wertes sollten dann mit dem (angepassten) Maß weiterverarbeitet werden. Ein Ablaufschema ist in Abb. \ref{fig:ablauf} dargestellt.

\begin{figure}
    \centering
    \includegraphics[width=1\linewidth]{plots/ablauf.png}
    \caption{Alternativer Ablaufplan}
    \label{fig:ablauf}
\end{figure}

Hier seien ein paar Standardmethoden in der Literatur genannt,die das Entrauschen betreffen. 

\begin{itemize}
    \item Domänwissen
    \item Glättungstechniken
    \item Frequenzenanalyse
\end{itemize}

Schließlich wird noch diskutiert, inwieweit es sinnvoll ist, die Ergebnisse verschiedener Lags zu kombinieren.

Die Bedeutung des Lags hängt von der konkreten Anwendung ab. Es könnte daher sinnvoll sein, eine Methodik zu definieren, die auf der Grundlage subjektiver Einschätzungen Gewichtungen für die einzelnen Lags liefert.

\subsection{Gewichtete Konkordanz-Formel}

Motivation

\begin{itemize}
    \item Bei einer signifikanten Zunahme/Abnahme des tatsächlichen Wertes wird auch der Wert des Nowcaster-Schätzers signifikant erhöht/verringert.
    \item Signifikanz gemessen an dem aktuellen tatsächlichen Wert
    \item mind. so siginifikant heißt, mind. gleich viel relative Änderung wird vorhergesagt
    \item In solchem Fall sollte eine Vorhersage in die falsche Richtung im Bezug auf Trendability strenger bestraft werden.
    \item Betrachtung der relativen Änderung (immer im Bezug auf $\max{Y_t, Y_{t+1}}$)
    \item Gewicht der Konkordanz proportinal zu dem Betrag der relativen Änderung.
\end{itemize}

Diese Überlegungen führen zu folgedem Formelvorschlag (für Lag = 1): 
\[\lag X_t = X_{t+1}-Y_{t}\]
\[\lag Y_t = Y_{t+1}-Y_{t}\]

Berechnung der Gewichtung: 
\[w_t=\frac{\abs{\lag Y_t}}{\sum_{k=1}^{T-1}\abs{\lag Y_k}}\]

Relative Veränderungen:
\[\delta Y_t = \frac{\lag Y_t}{\min\{Y_t, Y_{t+1}\}} \]
\[\delta X_t = \frac{\lag X_t}{\min\{Y_t, X_{t+1}\}} \]

Gewichtete Formel
\begin{equation}
  \sum_{t = 1}^{T - 1} w_t \max\{\min\{\frac{\delta X_t}{\delta Y_t},1\},0\}
\end{equation}

\subsection{Weiterentwicklung der Methodik}


\section{Erste Simulationsstudie}
\begin{itemize}
    \item $\lag Y_t \sim \mathcal{N}$
    \item $X_t = X_{t-1} + \abs{Z} (2 (U < k) - 1) \sign(Y_t - Y_{t-1})$ ($Z \sim \mathcal{N}, U \sim \mathcal{U}(0, 1)$) \todo{Rückkopplung an $Y_{t-1}$}
    \item $X_t = Y_{t-1} + \abs{Z} (2 (U < k) - 1) \sign(Y_t - Y_{t-1})$ ($Z \sim \mathcal{N}, U \sim \mathcal{U}(0, 1)$)
\end{itemize}



%\subsection{Modellierung}
%\begin{itemize}
%  \item Sei $l$ der zeitliche Abstand, mit dem Realisationen verfügbar werden
%  \item $Y_{t + l} = Y_t + \lag Y_t$
%  \item Die Nowcast schätzen (fehlerbehaftet) $\lag Y_t$ mithilfe von Daten, die zum Zeitpunkt $\tau \in T_t^k$ verfügbar sind, und dem neuesten, bekannten Wert $Y_t$:
%	\begin{equation}
%  		X_{t \lvert \tau}^k = Y_{\tau - l} + (\widehat{\lag[t-\tau+l] Y_t})_{t \lvert \tau}^k,
%	\end{equation}
%	wobei
%		\begin{equation}
%  			\lag[t-\tau+l] Y_t =  (\widehat{\lag[t-\tau+l] Y_t})_{t \lvert \tau}^k + \varepsilon.
%		\end{equation}


%\end{itemize}
\section{Weitere Ideen zu Trending}

\begin{itemize}
  \item \enquote{Schwaches} Trending für lag $l$: Wahrscheinlichkeit, in die gleiche Richtung zu zeigen ist größer als Wahrscheinlichkeit in die falsche Richtung zu zeigen (für lag $l$)
  \item \enquote{Downwards}-Trending: Trends in die negative Richtung werden erkannt, in die positive Richtung nicht
  \item \enquote{Upwards}-Trending: Trends in die positive Richtung werden erkannt, in die negative nicht
  \item Nicht einfach nur 0-1-Kodierung für gleiches Vorzeichen von $\lag x$ und $\lag y$, sondern Gewichtung des $\mathbb{R}^2$, sodass zum Beispiel Punkte nahe der Achsen weniger Gewicht bekommen als solche nahe der Winkelhalbierenden
  \item PCA auf $\lag x$ und $\lag y$, Bestrafung der zweiten Komponente (Abweichung von Gerade): Bestrafung würde aber mehr umfassen, als lediglich den Trend, sondern würde Abweichung von Linearität betreffen
  \item Falls wir Exclusion Area wollen und $\lag Y_t$ heteroskedastisch: relative Werte betrachten?
\end{itemize}

%\printbibliography

\begin{comment}

Gegeben sind zwei Zeitreihen $(x_t)$ und $(y_t)$, $t\in \{t_0,...,t_N\}$. Dabei kann $(x_t)$ die zeitliche Entwicklung einer Zielgröße und $(y_t)$ die zeitlichen Schätzungen eines Nowcasters darstellen. Uns interessiert, ob die Entwicklungen der beiden Zeitreihen den gleichen Trend aufweisen bzw. ob der Nowcaster die Entwicklung der zu schätzenden Größe mit dem richtigen Trend schätzen kann. Es sollte ein Maß konstruiert werden, das beschreibt, wie gut zwei Zeitreihen den gleichen Trend aufweisen.

\subsection{Anforderungen an einem Maß/ Intuitionen}

Zwei Zeitreihen weisen den gleichen Trend auf, wenn für zwei beliebige Zeitpunkte $t$ und $\tau$ gilt: $(x_t-x_\tau)(y_t-y_\tau)>0$ oder $x_t=x_\tau \land y_t = y_\tau$.

\todo{Kann man theoretisch nicht einfach mit Rang-korrelations-koeffizient Trending überprüfen?}
\subsection{Ein einfaches Maß}
Sei $\Delta_{t,\tau}^{x}:=x_\tau-x_t$.
\begin{equation}
  		R_{t,\tau}^{x,y} \coloneqq 
        \begin{cases}
  		    \frac{\Delta_{t,\tau}^{x}\Delta_{t,\tau}^{y}}   {\vert\Delta_{t,\tau}^{x}\vert\vert\Delta_{t,\tau}^{y}\vert } &, \text{falls} \Delta_{t,\tau}^{x}\neq 0 \land \Delta_{t,\tau}^{y}\neq0\\
           1 &, \text{falls} \Delta_{t,\tau}^{x}=\Delta_{t,\tau}^{y}=0\\
            -1 &, \text{sonst} 
  	\end{cases}
	\end{equation}
Beispiel-Maß: $S(\bold{x},\bold{y})=\sum_{k=1}^{k^*}\frac{w_k}{N+1-k}\sum_{j=0}^{N-k}R_{t_j,t_{j+k}}^{x,y},$
wobei $w_k, k=1,...,k^*$ eine Gewichtung für das Trending-Verhalten in verschiedenen Zeitdistanzen darstellt. Dabei erfüllt $w_k$ die folgenden Anforderungen:

\begin{itemize}
    \item $w_k$ ist streng monoton fallend in $k$. Hintergrund: je klein der betrachtete Zeitabstand ist, desto größer ist hier der Einfluss des Rauschens
    \item $\sum_{k=1}^{k^*}w_k=1$
\end{itemize}

\subsection{Eigenschaften des vorgeschlagenen Maßes}
\begin{itemize}
    \item Seien $(x_n)$ und $(y_n)$ zwei Realisierungen eines Random Walk -Modells.\\
    Hypothese: $S((x_n),(y_n))=0$. \\
    Intuition:  Zwei Random Walks $(x_n)$ und $(y_n)$ zeigen kein gemeinsames Tendenzverhalten auf. 
\end{itemize}

\subsection{Gewichtung für das Maß $k$}

Idee: Man betrachtet für jeden Punkt $(\lag x, \lag y)$ die Wahrscheinlichkeit für diesen Punkt in jedem der vier Quadranten zu liegen, wenn Messfehler abgezogen werden.
Dann gewichten wir das Maß 
\begin{equation}
    k (x, y; l, \epsilon) = \frac{\sum_{t}^{n-l}  \omega k^s (\lag y_t, \lag x_t; l, \epsilon)}{\sum_{t}^{n-l} k^\epsilon (\lag y_t, \lag x_t)},
\end{equation}
mit $\omega = P(Q_{ru}) + P(Q_{ll}) - P(Q_{lu}) - P(Q_{rl})$ für jeden Punkt. \todo{$\omega$ ist Funktion von jedem Punkt (siehe Blatt)}


Gewichtung über verschiedene lags entweder optimieren oder über Plots verschiedene Lags darstellen
 

% \section{Statistische Modellierung}
% \subsection{Modellierung}
% \begin{itemize}
%   \item Sei $l$ der zeitliche Abstand, mit dem Realisationen verfügbar werden
%   \item $Y_{t + l} = Y_t + \lag Y_t$
%   \item Die Nowcast schätzen (fehlerbehaftet) $\lag Y_t$ mithilfe von Daten, die zum Zeitpunkt $\tau \in T_t^k$ verfügbar sind, und dem neuesten, bekannten Wert $Y_t$:
% 	\begin{equation}
%   		X_{t \lvert \tau}^k = Y_{\tau - l} + (\widehat{\lag[t-\tau+l] Y_t})_{t \lvert \tau}^k,
% 	\end{equation}
% 	wobei
% 		\begin{equation}
%   			\lag[t-\tau+l] Y_t =  (\widehat{\lag[t-\tau+l] Y_t})_{t \lvert \tau}^k + \varepsilon.
% 		\end{equation}


<<<<<<< Updated upstream
% \end{itemize}
\subsection{Weitere Ideen zu Trending}

\begin{itemize}
  \item \enquote{Schwaches} Trending für lag $l$: Wahrscheinlichkeit, in die gleiche Richtung zu zeigen ist größer als Wahrscheinlichkeit in die falsche Richtung zu zeigen (für lag $l$)
  \item \enquote{Downwards}-Trending: Trends in die negative Richtung werden erkannt, in die positive Richtung nicht
  \item \enquote{Upwards}-Trending: Trends in die positive Richtung werden erkannt, in die negative nicht
  \item \enquote{Complete}-Trending: Trends in beiden Richtungen werden erkannt
  \item Nicht einfach nur 0-1-Kodierung für gleiches Vorzeichen von $\lag x$ und $\lag y$, sondern Gewichtung des $\mathbb{R}^2$, sodass zum Beispiel Punkte nahe der Achsen weniger Gewicht bekommen als solche nahe der Winkelhalbierenden
  \item PCA auf $\lag x$ und $\lag y$, Bestrafung der zweiten Komponente (Abweichung von Gerade): Bestrafung würde aber mehr umfassen, als lediglich den Trend, sondern würde Abweichung von Linearität betreffen
  \item Falls wir Exclusion Area wollen und $\lag Y_t$ heteroskedastisch: relative Werte betrachten?
\end{itemize}

\section{Simulationsstudie}

\begin{itemize}
    \item $\lag Y_t \sim \mathcal{N}$
    \item $X_t = X_{t-1} + \abs{Z} (2 (U < k) - 1) \sign(Y_t - Y_{t-1})$ ($Z \sim \mathcal{N}, U \sim \mathcal{U}(0, 1)$) \todo{Rückkopplung an $Y_{t-1}$}
    \item $X_t = Y_{t-1} + \abs{Z} (2 (U < k) - 1) \sign(Y_t - Y_{t-1})$ ($Z \sim \mathcal{N}, U \sim \mathcal{U}(0, 1)$)
\end{itemize}

%\printbibliography


||||||| Stash base

\section{Statistische Modellierung}
\subsection{Modellierung}
\begin{itemize}
  \item Sei $l$ der zeitliche Abstand, mit dem Realisationen verfügbar werden
  \item $Y_{t + l} = Y_t + \lag Y_t$
  \item Die Nowcast schätzen (fehlerbehaftet) $\lag Y_t$ mithilfe von Daten, die zum Zeitpunkt $\tau \in T_t^k$ verfügbar sind, und dem neuesten, bekannten Wert $Y_t$:
	\begin{equation}
  		X_{t \lvert \tau}^k = Y_{\tau - l} + (\widehat{\lag[t-\tau+l] Y_t})_{t \lvert \tau}^k,
	\end{equation}
	wobei
		\begin{equation}
  			\lag[t-\tau+l] Y_t =  (\widehat{\lag[t-\tau+l] Y_t})_{t \lvert \tau}^k + \varepsilon.
		\end{equation}


\end{itemize}
\subsection{Weitere Ideen zu Trending}

\begin{itemize}
  \item \enquote{Schwaches} Trending für lag $l$: Wahrscheinlichkeit, in die gleiche Richtung zu zeigen ist größer als Wahrscheinlichkeit in die falsche Richtung zu zeigen (für lag $l$)
  \item \enquote{Downwards}-Trending: Trends in die negative Richtung werden erkannt, in die positive Richtung nicht
  \item \enquote{Upwards}-Trending: Trends in die positive Richtung werden erkannt, in die negative nicht
  \item Nicht einfach nur 0-1-Kodierung für gleiches Vorzeichen von $\lag x$ und $\lag y$, sondern Gewichtung des $\mathbb{R}^2$, sodass zum Beispiel Punkte nahe der Achsen weniger Gewicht bekommen als solche nahe der Winkelhalbierenden
  \item PCA auf $\lag x$ und $\lag y$, Bestrafung der zweiten Komponente (Abweichung von Gerade): Bestrafung würde aber mehr umfassen, als lediglich den Trend, sondern würde Abweichung von Linearität betreffen
  \item Falls wir Exclusion Area wollen und $\lag Y_t$ heteroskedastisch: relative Werte betrachten?
\end{itemize}

%\printbibliography


\end{comment}

\end{document} 