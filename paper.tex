\documentclass[oneside]{article}

% Language setting
% Replace `english' with e.g. `spanish' to change the document language
\usepackage[english]{babel}
\usepackage[utf8]{inputenc} % für Umlaute
\usepackage{fontenc}

% Set page size and margins
% Replace `letterpaper' with`a4paper' for UK/EU standard size
\usepackage[a4paper,top=2cm,bottom=2cm,left=3cm,right=3cm,marginparwidth=1.75cm]{geometry}

%%%% ---- Useful packages

%%% --- Figures and Graphics
\usepackage{graphicx}
\usepackage{subcaption}
\usepackage{tikz}

%%% --- Colors
\usepackage[]{xcolor}

%%% --- Tabulars
\usepackage{booktabs}
\usepackage{multirow} % cells over several rows
\usepackage{tabularx}

%%% --- Mathematics
\usepackage{amsmath,amsfonts,amssymb,amsthm,mathtools, mathabx}
\usepackage{bm} % bold mathematics
\usepackage{bbm} % for blackboard 1
\usepackage[linesnumbered,lined,algo2e,figure,boxed]{algorithm2e}
\usepackage[short]{optidef} % for aligning linear programs

%%% --- Other
\usepackage[colorlinks=true, allcolors=blue]{hyperref}
%\usepackage[hidelinks]{hyperref}
\usepackage[style=authoryear,backend=bibtex]{biblatex}
\addbibresource{library.bib}
\usepackage{csquotes}
\usepackage[shortcuts]{extdash} % For Hyphens in English
%\usepackage{acro}
% \DeclareAcronym{pdf}{short=PDF, long={probability distribution function}}

%%% --- Commenting, Todonotes
\usepackage[colorinlistoftodos,prependcaption]{todonotes}
\usepackage{comment, soul}
\newcommand{\ar}{$\rightarrow$}
\presetkeys{todonotes}{backgroundcolor=yellow!30, bordercolor=yellow!50, linecolor=yellow!50, figwidth=\textwidth}{}
%\newcommand{\hl}[1]{\textcolor{Aquamarine}{#1}}

\usepackage{xargs} % Use optional arguements in commands
\newcommandx{\unsure}[2][1=]{\todo[linecolor=blue,backgroundcolor=blue!25,bordercolor=blue,#1]{#2}}
\newcommandx{\missing}[2][1=]{\todo[linecolor=red,backgroundcolor=red!25,bordercolor=red,#1]{#2}}
\newcommandx{\change}[2][1=]{\todo[linecolor=red,backgroundcolor=red!25,bordercolor=red,#1]{#2}}
\newcommandx{\info}[2][1=]{\todo[linecolor=OliveGreen,backgroundcolor=OliveGreen!25,bordercolor=OliveGreen,#1]{#2}}
\newcommandx{\improvement}[2][1=]{\todo[linecolor=Plum,backgroundcolor=Plum!25,bordercolor=Plum,#1]{#2}}
\newcommandx{\thiswillnotshow}[2][1=]{\todo[disable,#1]{#2}}
\usepackage{marginnote}
\let\marginpar\marginnote

%%%% ---- Own Commands
%%% --- Theorem Settings
\theoremstyle{plain}% Theorem-like structures provided by amsthm.sty
\newtheorem{theorem}{Theorem}
\newtheorem{exa}{Example}
\newtheorem{rem}{Remark}
\newtheorem{proposition}{Proposition}
\newtheorem{lemma}{Lemma}
\newtheorem{corollary}{Corollary}

\theoremstyle{definition}
\newtheorem{definition}{Definition}
\newtheorem{remark}{Remark}
\newtheorem{example}{Example}

%%% --- Math Commands

\newcommand{\lag}[1][l]{\Delta_{#1}}
\DeclarePairedDelimiter{\abs}\lvert\rvert
\DeclareMathOperator{\sign}{sign}
\newcommand{\tmax}{\bar{t}}
\newcommand{\ind}[1]{\mathbbm{1}\{#1\}}
\newcommand{\ydiff}{D y}
\newcommand{\ydifft}{Dy^\star}
\newcommand{\xdiff}{Dx}
\newcommand{\xdifft}{Dx^\star}
\newcommand{\Ydiff}{DY}
\newcommand{\Xdiff}{DX}
\newcommand{\Prob}[1]{P(#1)}
\newcommand{\mprob}{\tilde{m}}
\newcommand{\Ber}{\text{Ber}}
\newcommand{\SBer}{\text{SBer}}

%%% --- Other Commands
\renewcommand{\arraystretch}{1.2}

%%% --- ToDo List
\usepackage{enumitem}
\newlist{todolist}{itemize}{2}
\setlist[todolist]{label=$\square$}
\usepackage{pifont}
\newcommand{\cmark}{\ding{51}}%
\newcommand{\xmark}{\ding{55}}%
\newcommand{\done}{\rlap{$\square$}{\raisebox{2pt}{\large\hspace{1pt}\cmark}}%
\hspace{-2.5pt}}
\newcommand{\wontfix}{\rlap{$\square$}{\large\hspace{1pt}\xmark}}

\title{Assessment and Optimization of Nowcast Measures for Trend Detection}
\author{Oliver Grothe, Bolin Liu, Jonas Rieger}

\begin{document}
\maketitle

\begin{abstract}
Existing and new measures for the capability of capturing the trend of nowcasts are presented, evaluated, and compared on synthetic and real-world data.
\end{abstract}

\subsection*{Offene Punkte}

\begin{itemize}
    \item was ist mit dem Trendinghorizont, also welche Zeitinkremente sind zu nehmen. Da würde sich irgendwie ein Plot Maß gegen Inkrement anbieten. Dann würden sich für unterschiedliche Inkremente unterschiedliche Trendingcapabilities ergeben (gabe es nicht so einen schonmal bei irgendeinem Treffen?)
    \item Wahl des Rauschens; haben wir da irgendeine Methode aus den Daten das Sigma zu bekommen?
    \item Was ist mit anderen Verteilungen?
    \item Nicht nur Nowcast, sondern auch Measurement und Forecasts; in Introduction Unterschiede zwischen den Feldern; mehr Literatur; Einleitung ausbauen
    \item Messen die Maße das richtige? Sollte man andere Maße nutzen? Braucht man Korrekturterm?
    \item Indikatoren oder Wahrscheinlichkeiten?
    \item Wahrscheinlichkeit von der Pike herleiten
\end{itemize}

Für nächstes Treffen:
\begin{itemize}
    \item Formale Fragestellung
    \item Wie wird $\sigma$ geschätzt? Filter vs. Empirie
    \item Kondidenzinterval     
\end{itemize}


\listoftodos


\section{Introduction}

\begin{itemize}
	\item Trending is an important feature of nowcasts to be identify changes in the situation before the actual quantity can be measured
	\item What scores are currently used?
	\item Examples: Covid, BIP, Medicine, ... 
 \item nowcasting schön aufschreiben, dass auch für Nicht-Nowcaster klar wird, warum Nowcasting speziell ist
\end{itemize}


\section{Trend measures for nowcasts}

Notation
\begin{itemize}
  \item Let $y_1, \ldots, y_T$ be the values of the nowcasted quantity.
  \item Let $x_t^k$ be the $k$-th nowcast for time $t = 1, \ldots, T$ (possibly after some aggregation if several raw nowcasts apply).
  \item Let $\ydiff_t = y_t - y_{t-1}$ be the change that occurred between the time steps $t-1$ and $t$ and $\xdiff_t^k = x_t - y_{t-1}$ the change nowcaster $k$ issued between the time steps. If $y_{t-1}$ is not available at time $t$, it can be reasonable to use an updated nowcast for time $t-1$. (Move this point to a place after error model, as the error model can account for the error made at this point) \todo{Kaligraphisches $x$, um vorzubereiten, dass es anderes sein könnte}
  \item Then, $\xdiff_t^k$ can be seen as point forecast of nowcaster $k$ for $\ydiff_t$. 
\end{itemize}

\subsection{What is the trend of a nowcast?}

Trend: Issuing the right \enquote{direction} of change between two time steps; statistical consistency of $\sign(\ydiff_t)$ and $\sign(\xdiff_t)$ (Problem: mit kleinen Fehlern; Hinweis Exclusion area)

\begin{itemize}
  \item Translate into mathematical notation
  \item Illustrative Example: Two nowcasts of quantity of interest, where one jumps up and down and one points in right direction with the same rmse; small subset of Simulation~\ref{sec:simulation_rmse_mae}; Plot time series \& time series of differences
\end{itemize}

\subsection{Graphical methods}


\begin{itemize}
  \item 4Q-Plot
  \item Andere Medizin-Plots mit Verweis auf Grothe-Paper
\end{itemize}


\subsection{Measures}

%\begin{table}
%	\centering
%	\begin{tabular}{l l c c}
%	\toprule
%	& & \multicolumn{2}{c}{Observed trend} \\
%	& & $\ydiff_t > 0$ & $\ydiff_t < 0$ \\	\cline{3-4}
%	\multirow{2}{*}{Forecasted trend} & $\xdiff_t > 0$ & hit & false alarm\\
%	& $\xdiff_t < 0$ & miss & correct negative \\
%	\bottomrule
%\end{tabular}
%\end{table}

\begin{itemize}
  \item Accuracy: \begin{equation}
  	m_{\text{acc}} = \frac{\sum_{t=T-w}^T \ind{\ydiff_t > 0, \xdiff_t > 0} + \sum_{t=T-w}^T \ind{\ydiff_t < 0, \xdiff_t < 0}}{T-w}
\end{equation}
	\item Capability of detecting a positive trend (probability of detection)
	\begin{equation}
  		m_{\text{py}} = \frac{\sum_{t=T-w}^T \ind{\ydiff_t > 0, \xdiff_t > 0}}{\sum_{t=T-w}^T \ind{\ydiff_t > 0}}
	\end{equation}
	\item False alarm rate \hl{Name does not fit} \begin{equation}
  m_{\text{far}} = \frac{\sum_{t=T-w}^T \ind{\ydiff_t < 0, \xdiff_t > 0}}{\sum_{t=T-w}^T \ind{\xdiff_t > 0}}
\end{equation}
maybe use \enquote{positive} measure instead, i.e.,
\begin{equation}
  m_{\text{px}} = \frac{\sum_{t=T-w}^T \ind{\ydiff_t > 0, \xdiff_t > 0}}{\sum_{t=T-w}^T \ind{\xdiff_t > 0}}  
\end{equation}
	\item same measures for negative direction:
	\begin{align}
		m_{\text{ny}} &=  \frac{\sum_{t=T-w}^T \ind{\ydiff_t < 0, \xdiff_t < 0}}{\sum_{t=T-w}^T \ind{\ydiff_t < 0}} \\
		m_{\text{nx}} &=  \frac{\sum_{t=T-w}^T \ind{\ydiff_t < 0, \xdiff_t < 0}}{\sum_{t=T-w}^T \ind{\xdiff_t < 0}}
	\end{align}
 \item Bezug auf Confusionsmatrix 
\end{itemize}


Resulting measures:
\begin{itemize}
  % \item Scoring rules for probabilities

% \begin{itemize}
%   \item Quadratic/Brier Score for $k$ possible outcomes \hl{Erst mal rauslassen; wäre anderes setting: probabilistischer Nowcast}: \begin{equation}
%   S((p_1, \dots, p_k), i) = 2 p_i - \sum_{l=1}^k p_i^2 - 1;
% \end{equation}
% for dichotomous outcomes ($p_1 = \hat{P}(\ydiff_t > 0)$: probability for $\ydiff_t > 0$):
% \begin{equation}
%   S(p_1, \ydiff_t) = p_1 \ind{\ydiff_t > 0} + (1 - p_1) (1 - \ind{\ydiff_t > 0}) + p_1^2 + (1-p_1)^2 - 1
% \end{equation}
% \end{itemize}
\item Probabilistic versions of non-probabilistic measures; small sample characteristic of $m_{acc}$ can be driven by random effects near the axis; therefore probabilistic \enquote{correction}:
\begin{itemize}
  \item Assume an additive error decomposition for both observation and nowcast to account for non-systematic and short-term (i.e., intra-day) effects or for effects on inaccurate representation of $y_{t-1}$ in the computation of $Dx_t$ (\hl{this, no additional complexity}):
  	\begin{align}\label{additive error decomposition}
  		\ydiff_t &= \ydifft + \varepsilon_t^y \\
  		\xdiff_t &= \xdifft + \varepsilon_t^x
	\end{align}
 \item \hl{Wouldn't it be more natural to model the original time series with a noise term?} (Qualitatively, it would yield the same results)
 \begin{align}
     y_t &= y_t^\star + \epsilon_t^y \\
     x_t &= x_t^\star + \epsilon_t^x \\
     \ydiff_t &= y_t^\star + \epsilon_t^y - (y_{t-1}^\star + \epsilon_{t-1}^y) = y_t^\star - y_{t-1}^\star + \underbrace{\epsilon_t^y - \epsilon_{t-1}^y}_{\sim N(0, \sqrt{2} \sigma_y)} \\
     \xdiff_t &= x_t^\star + \epsilon_t^x - (x_{t-1}^\star + \epsilon_{t-1}^x) = x_t^\star - x_{t-1}^\star + \underbrace{\epsilon_t^x - \epsilon_{t-1}^x}_{\sim N(0, \sqrt{2} \sigma_x)}
 \end{align}
 Advantage of the approach: For different lags, no new standard deviation has to be estimated.\\
 Disadvantage: Theoretically, there is serial correlation of the errors. (Wie behandle ich die Autokorrelation; wie rausrechnen)
\item  Replace indicators in the above formulations by the corresponding probabilities:
\begin{align}
		\mprob_{\text{acc}} &= \frac{\sum_{t=T-w}^T \Prob{ \ydifft_t > 0, \xdifft_t > 0} + \sum_{t=T-w}^T \Prob{\ydifft_t < 0, \xdifft_t < 0}}{T-w}  \\
   \mprob_{\text{py}} &= \frac{\sum_{t=T-w}^T \Prob{\ydifft_t > 0, \xdifft_t > 0}}{\sum_{t=T-w}^T \Prob{\ydifft_t > 0}} \\
    \mprob_{\text{px}} &= \frac{\sum_{t=T-w}^T \Prob{\ydifft_t > 0, \xdifft_t > 0}}{\sum_{t=T-w}^T \Prob{\xdifft_t > 0}} \\
    \mprob_{\text{ny}} &= \frac{\sum_{t=T-w}^T \Prob{\ydifft_t < 0, \xdifft_t < 0}}{\sum_{t=T-w}^T \Prob{\ydifft_t < 0}} \\
    \mprob_{\text{px}} &= \frac{\sum_{t=T-w}^T \Prob{\ydifft_t < 0, \xdifft_t < 0}}{\sum_{t=T-w}^T \Prob{\xdifft_t < 0}} 
\end{align} 
\item Simple error model:
  \begin{align}
	  \varepsilon_Y \sim N(0, \sigma_Y) \\
	  \varepsilon_X \sim N(0, \sigma_X)
  \end{align}
  yields, e.g.,
  	\begin{equation}
  		\mprob_{\text{acc}} = \frac{1}{T-w} \sum_{t=T-w}^T  \big( 1 - \Phi_{\ydiff_t, \sigma_y}(0) - \Phi_{\xdiff_t, \sigma_x} (0) + 2 \Phi_{\xdiff_t,\sigma_x}( 0)\Phi_{\ydiff_t,\sigma_y}( 0) \big), 
	\end{equation}
	where $\Phi$ denotes a (possibly multivariate) normal distribution
\end{itemize}
\item For $\sigma_x \sigma_y \rightarrow 0$, the probabilistic measures approach the count-based ones
\item For $\sigma_x \sigma_y \rightarrow \infty$, the probabilistic accuracy approaches   0.5 as all probabilities are approximately 0.5
\item Lag-$l$-measures can easily be obtained by using the $D^l$-difference, i.e., $D^l y_t = y_t - y_{t-l}$
\end{itemize}



\subsection{Estimating the standard deviation of white noise}

First, sometimes variance can be determined heuristically directly from domain knowledge. 
However, to determine the standard deviation of white noise using only the time series data, one can basically follow two approaches, time domain approach or frequency domain approach. For the first approach, a model assumption (e.g. ARIMA models) is specified and the residual analysis technique is applied [\cite{chitturi1974distribution}]. 
In the frequency domain approach, the time series are first decomposed using spectral methods such as (discrete) Fourier transformation [\cite{sauer1992noise}] or wavelet transformation [\cite{heil1989continuous,donoho1994ideal,jansen2012noise}]. After reconstructing the time series, the residuals can be formed in order to determine the standard deviation of white noise. For the use of wavelet, it is particularly convenient that the standard deviation of the white noise can be estimated by the standard deviation of the detail coefficients of the last level, e.g., if the wavelet family db1 is used.

For the noisy signal in the form of \ref{eq: simulation},  there are the following results (cf. pic \ref{fig:estimation std})



%\begin{figure}
%  \centering
%    \begin{subfigure}{0.24\textwidth}
%        \includegraphics{plots/std estimation/std_estimation_boxplot_25.pdf}
%    \end{subfigure}
%    \begin{subfigure}{0.24\textwidth}
%        \includegraphics{plots/std estimation/std_estimation_boxplot_50.pdf}
%    \end{subfigure}
%    \begin{subfigure}{0.24\textwidth}
%        \includegraphics{plots/std estimation/std_estimation_boxplot_100.pdf}
%    \end{subfigure}
%    \begin{subfigure}{0.24\textwidth}
%        \includegraphics{plots/std estimation/std_estimation_boxplot_1000.pdf}
%    \end{subfigure}
%  \caption{Estimation of std with different methods.}
%  \label{fig:estimation std}
%\end{figure}%

\begin{figure}
  \centering
    \begin{subfigure}{0.48\textwidth}
        \includegraphics[width=\textwidth]{plots/std estimation/std_estimation_boxplot_25.pdf}
    \end{subfigure}
    \begin{subfigure}{0.48\textwidth}
        \includegraphics[width=\textwidth]{plots/std estimation/std_estimation_boxplot_50.pdf}
    \end{subfigure}

    \begin{subfigure}{0.48\textwidth}
        \includegraphics[width=\textwidth]{plots/std estimation/std_estimation_boxplot_100.pdf}
    \end{subfigure}
    \begin{subfigure}{0.48\textwidth}
        \includegraphics[width=\textwidth]{plots/std estimation/std_estimation_boxplot_1000.pdf}
    \end{subfigure}

  \caption{Estimation of std with different methods.}
  \label{fig:estimation std}
\end{figure}


\subsection{Confidence intervals and missing data}


\begin{itemize}
  \item Bootstrapping: Compare percentile and BCa approach?
  \item Bootstrapping: What is the \enquote{true} value?
  \item Missing data
\end{itemize}


\section{Simulation studies}

\subsection{Different trending abilities for same rmse and mae} \label{sec:simulation_rmse_mae}

Let the quantity of interest 
\begin{equation}\label{eq: simulation}
  y_t = a \sin(2 \pi t / T) + \varepsilon_t^y, \ t = 1, \dots, T
\end{equation}
with $\varepsilon_t^y \stackrel{\text{iid}}{\sim} N(0, \sigma_y)$ and 
\begin{equation}
  \ydiff_t = y_t - y_{t-1}.
\end{equation}

\todo[inline]{Anpassung der Simulationssettings an Gl. \ref{additive error decomposition} (?)}
The nowcasts are characterised by 
\begin{align}
	(\xdifft_t)^1 &= e^x_t \ydifft_t \\
	(\xdifft_t)^2 &= \begin{cases}
		e^x_t \ydifft_t &, \text{if}\ q^2_t = 0 \lor e^x_t \geq 0\\
		(| e^x_t | + 2) \ydifft_t &, \text{if}\ q^2_t = 1 \land e^x_t < 0
	\end{cases} \\
	(\xdifft_t)^3 &= \begin{cases}
		e^x_t \ydifft_t &, \text{if}\ (q^{3, \text{down}}_t = 0 \lor e^x_t > 0) \land \ydifft_t \geq 0\\
		(| e^x_t | + 2) \ydifft_t &, \text{if}\ (q^{3, \text{down}}_t = 1 \land e^x_t < 0) \land \ydifft_t \geq 0 \\
		e^x_t \ydifft_t &, \text{if}\ (q^{3, \text{up}}_t = 0 \lor e^x_t > 0) \land \ydifft_t < 0\\
		(| e^x_t | + 2) \ydifft_t &, \text{if}\ (q^{3, \text{up}}_t = 1 \land e^x_t < 0) \land \ydifft_t < 0
	\end{cases}
\end{align}
where $q^2_t \sim \text{Ber}(2p_2 - 1)$ and $p_2 > 0.5$ denotes the probability that the direction of nowcast 2 is correct.
For nowcast 3 with $q^{3, \text{up}}_t \sim \text{Ber}(2p_3^{\text{up}}-1)$, $q^{3, \text{down}}_t \sim \text{Ber}(2p_3^{\text{down}}-1)$ the probabilities of direction correction depend on the direction of change. Additionally, let $e^x_t \sim N(0, 1)$.

\begin{figure}
  \centering
  \includegraphics{plots/simulation_same_rmse_mae/time_series.pdf}
  \caption{Realisation of Simulation~\ref{sec:simulation_rmse_mae} with $T = 100$, $\sigma_y=0.1$, $a = 10$, $p_2 = 0.8$, and $p_3 = (0.9, 0.5)$}
  \label{fig:simulation_rmse_mae_ts}
\end{figure}

\begin{figure}
  \centering
  \begin{subfigure}{.32\textwidth}
  	\includegraphics{plots/simulation_same_rmse_mae/4q_plot_1}
  \end{subfigure}
  \begin{subfigure}{.32\textwidth}
  	\includegraphics{plots/simulation_same_rmse_mae/4q_plot_2}
  \end{subfigure}
    \begin{subfigure}{.32\textwidth}
  	\includegraphics{plots/simulation_same_rmse_mae/4q_plot_3}
  \end{subfigure}
  \caption{4Q-Plot for Simulation~\ref{sec:simulation_rmse_mae} with $T = 100$, $\sigma_y=0.1$, $a = 10$, $p_2 = 0.8$, and $p_3 = (0.9, 0.5)$.}
  \label{fig:simulation_rmse_mae_4q}
\end{figure}

\todo[inline]{Weiteres Beispiel für viele richtigen Punkte sehr nah an den Entscheidungsgrenzen.}
\subsection{Simulation: Bootstrap}

Simulation setup:

\begin{itemize}
  \item $T = 10,000$
  \item $\ydiff_t \sim N(0, 1)$ 
  \item $\xdiff_t^{1, k} = b_t  \ydiff_t, b_t \sim \SBer(k)$ (symmetric Bernoulli)
\end{itemize}
\unsure[inline]{(or both + random (non-systematic) noise?)}

Analysis:
\begin{itemize}
\item Do for $l = 1, \dots, 100$:
\begin{itemize}
  \item Sample $\ydiff, \xdiff$
  \item Compute bootstrap confidence intervals $(b_{low}, b_{up})_l$ using $B = 10,000$
\end{itemize}
  \item Visualize all intervals with real value (confidence interval should cover real value in $\alpha$ cases)
\end{itemize}



\subsection{Effect of increasing sample size?}

\section{Application on disease and economic data}

\subsection{Nowcasts for the COVID infections in Germany}

\begin{itemize}
  \item Describe QoI and Nowcasts; which nowcasts do we consider? Which data is available when?
  \item Plot data
\end{itemize}

\subsection{Nowcasting of GDP}

\section{Conclusion}
\medskip
\printbibliography

\appendix
\section{Illustrative examples for probabilistic measures}

\begin{figure}
    \centering
    \includegraphics{plots/illustrative_examples/plots_4q_and_mtilde.pdf}
    \caption{Various examplary courses of the measures $\tilde{m}$ for different configurations of $\ydiff$ and $\xdiff$}
    \label{fig:example_mtilde}
\end{figure}

\end{document} 