\documentclass{article}

% Language setting
% Replace `english' with e.g. `spanish' to change the document language
\usepackage[english]{babel}
\usepackage[utf8]{inputenc} % für Umlaute
\usepackage{fontenc}

% Set page size and margins
% Replace `letterpaper' with`a4paper' for UK/EU standard size
\usepackage[a4paper,top=2cm,bottom=2cm,left=3cm,right=3cm,marginparwidth=1.75cm]{geometry}

%%%% ---- Useful packages

%%% --- Figures and Graphics
\usepackage{graphicx}
\usepackage{subcaption}

%%% --- Colors
\usepackage{color}
\usepackage[dvpinames]{xcolor}

%%% --- Tabulars
\usepackage{booktabs}

%%% --- Mathematics
\usepackage{amsmath,amsfonts,amssymb,amsthm,mathtools}
\usepackage{bm} % bold mathematics
\usepackage{bbm} % for blackboard 1
\usepackage[linesnumbered,lined,algo2e,figure,boxed]{algorithm2e}
\usepackage[short]{optidef} % for aligning linear programs

%%% --- Other
\usepackage[colorlinks=true, allcolors=blue]{hyperref}
%\usepackage[hidelinks]{hyperref}
\usepackage[style=authoryear,backend=bibtex]{biblatex}
%\addbibresource{library.bib}
\usepackage{csquotes}
\usepackage[shortcuts]{extdash} % For Hyphens in English
%\usepackage{acro}
% \DeclareAcronym{pdf}{short=PDF, long={probability distribution function}}

%%% --- Commenting, Todonotes
\usepackage{comment, todonotes}
\newcommand{\ar}{$\rightarrow$}
\presetkeys{todonotes}{backgroundcolor=yellow!30, bordercolor=yellow!50, linecolor=yellow!50, figwidth=\textwidth}{}
\newcommand{\hl}[1]{\textcolor{Aquamarine}{#1}}

%%%% ---- Own Commands
%%% --- Theorem Settings
\theoremstyle{plain}% Theorem-like structures provided by amsthm.sty
\newtheorem{theorem}{Theorem}
\newtheorem{exa}{Example}
\newtheorem{rem}{Remark}
\newtheorem{proposition}{Proposition}
\newtheorem{lemma}{Lemma}
\newtheorem{corollary}{Corollary}

\theoremstyle{definition}
\newtheorem{definition}{Definition}
\newtheorem{remark}{Remark}
\newtheorem{example}{Example}

%%% --- Math Commands

\newcommand{\lag}[1][l]{\Delta_{#1}}
\DeclarePairedDelimiter{\abs}\lvert\rvert

%%% --- Other Commands
\renewcommand{\arraystretch}{1.2}

%%% --- ToDo List
\usepackage{enumitem}
\newlist{todolist}{itemize}{2}
\setlist[todolist]{label=$\square$}
\usepackage{pifont}
\newcommand{\cmark}{\ding{51}}%
\newcommand{\xmark}{\ding{55}}%
\newcommand{\done}{\rlap{$\square$}{\raisebox{2pt}{\large\hspace{1pt}\cmark}}%
\hspace{-2.5pt}}
\newcommand{\wontfix}{\rlap{$\square$}{\large\hspace{1pt}\xmark}}

\title{Trending in Nowcasting}
\author{Oliver Grothe, Bolin Liu, Jonas Rieger}

\begin{document}
\maketitle

%\begin{abstract}
%Your abstract.
%\end{abstract}

\section{ToDos}


\begin{todolist}
\item[\done] Vorlage erstellen (Jonas)
\item[\done] Schätzer einarbeiten (Jonas)
\item[\done] Alternativvorschlag Notation (Bolin)
\item[\done] Aufschreiben der Diskussionsergebnisse (Bolin): Grundproblems, Anforderungen und Beispielmaß
\item Eigenschaften des Beispiel-Maßes untersuchen/ angeben (Jonas, Bolin ) 
\end{todolist}


The evaluation of measurement, prediction, and forecasting methods becomes increasingly important and sophisticated as technologies and data availability enable their application in more and more fields. 
Conventionally, methods are evaluated using distance measures of local differences between predictions (or measurements) and target values. 
These measures do not consider whether the right direction of change is predicted or measured.
The information about whether an increase or decrease is predicted correctly is crucial when making decisions based on the estimator's prediction results. 
In the following, we provide a more specific overview of the characteristics and current evaluation schemes of the three application fields, forecasting, nowcasting, and measurement, and highlight why trending evaluation is highly relevant in the respective fields.

The above-described trending idea is of fundamental interest for evaluating and comparing forecasting methods. 
Forecasting methods predict the future based on historical data, patterns, and exogenous factors. 
The forecast is computed based on the current value of the quantity of interest and an estimate of the development until the target time.
A forecast's trending is perfectly consistent with the actual development of the target value if the actual change in the target value over this period matches the forecast change. 
In the current practice, a forecasting method is usually evaluated in terms of its prediction accuracy, measuring the deviation of the prediction from the actual outcome of the target variable. 
Popular measures are scale-dependent measures such as the \ac{rmse}, measures based on percentage errors such as the root mean squared percentage error, or probabilistic scoring rules \textcite[see the review in][]{hyndman2006another}. 
These typical measures are based on the absolute difference between the prediction and the true values, locally or globally. 
They are not capable of assessing the trending ability discussed above.

Methodologically evolved from forecasting, nowcasting methods focus on predictions for the present, the immediate future, and the recent past \parencite{banbura2013now} and are now widely used in fields such as economics and medicine \parencite{bok2018macroeconomic, Wolffram2023}.
Nowcasting has its origins in meteorology, and the methods were initially developed to describe the current state of the weather in detail and to predict the expected change on a time scale of a few hours \parencite{browning1989nowcasting,schmid2019nowcasting}. 
In economics, nowcasting is used to predict statistics on the current economic situation, for example, the gross domestic product, which is collected with low frequency and is available with a considerable time delay \parencite{banbura2013now}.
In medicine, epidemic nowcasting assesses the current situation during an ongoing epidemic, considering the main pathogenic, epidemiological, clinical, and socio-behavioral factors \parencite{wu2021nowcasting}. 
Nowcasting methods use high-frequency indicators related to the target variable and estimate the value of a target variable for a specific time based on current preliminary measurements, which are finalized with a considerable time delay. 
Thus, the nowcasts can produce early and ongoing estimates for the target variable during the relevant period \parencite{castle2017forecasting}. 
For example, the nowcasting method can correct the daily COVID case numbers for events that have occurred but have not yet been reported \parencite{gunther2021nowcasting}. 
Like forecasts, nowcasts are often evaluated and compared in the literature based on performance measures such as the \ac{rmse} \parencite{gunther2021nowcasting} or probabilistic scoring rules \parencite{Wolffram2023}. 
The aspect of trending ability is not considered in the literature to the best of our knowledge. 
However, trending evaluation adds valuable information on the methods' capability of predicting the development of the target variable, for example, the number of cases of an epidemic.
The epidemic's development, in turn, can be the foundation of decisions on introducing or canceling policy measures. 

Measurement aims to obtain accurate and reliable data about the current state of a system. 

A parameter or variable can be measured regularly over a certain period to evaluate the system's development.
When introducing new measurement methods, they are evaluated against current measurement techniques, called the gold standard, by measuring the same quantity in different settings, such as individuals or environments.
Various indices, such as the interclass correlation coefficient or Lin's concordance correlation, have been proposed to check the reproducibility of the measurement or to compare different measurement methods \parencite{lawrence1989concordance,koo2016guideline} in addition to paired t-tests to determine systematic differences between two measurement series \parencite{watson2010method}. 
Furthermore, graphical tools were developed, such as the Bland-Altman diagram, visualizing the differences between the methods in relation to their mean value \parencite{bland1986statistical}. 
Trending considerations, that is, the consistency of the test methods development with the gold standard, is a field of active research in the last years \parencite{Saugel2015,saugel2018error,hiraishi2021concordance}. 
In this paper, we build upon the existing methods and extend them with new functionality.

As outlined above, trending evaluation is crucial for measurements, predictions, and forecasts. 
However, regardless of the application area, most conventional method comparisons and assessments consider local absolute deviations without assessing whether the correct direction of change is being predicted or measured. 
In this paper, we develop trending evaluation methods that complement current evaluation measures with trending measurements.

The main contributions of this paper are manifold.
\begin{itemize}
\item We formalize trending and present different variants of trending measures that consider either noiseless data or data with noise and small non-informative changes.
\item We introduce the conditional trending plot, a new graphical method for assessing local trending behavior, and review bootstrap methods for calculating confidence intervals.
\item We extend the concept of trending to probabilistic predictions.
\item We apply trending evaluation to three applications: Measurement, nowcasting, and forecasting data. 
\item We provide a ready-to-use code for trending evaluation. The code and the source code to replicate the results of our study are available at ... .
\end{itemize}

The remainder of this paper is structured as follows. 
Section \ref{sec:trending} formalizes trending and investigates several extensions, such as noise-aware methods, confidence intervals, and probabilistic forecasts and nowcasts. 
In particular, we introduce a new graphical method for evaluating trending, the conditional trending plot, which can represent the simultaneous evaluation of different intervals and asymmetries. 
In Section \ref{sec:application}, we show the results of applying our method to several practical examples from measurement, forecasting, and nowcasting. 
We conclude the paper in Section~\ref{sec:conclusion}.


\section{Notation}
\begin{itemize}
  \item Sei $T = \{1, 2, \dots\}$ die Zeitindexmenge, wobei für jeden Zeitpunkt $t \in T$ eine Realisierung vorliegt (die ggf. im Nachhinein veröffentlicht wird).
  \item Sei $(Y_t)_{t \in T}$ die Zeitreihe der Realisierungen, dabei steht $t$ für den Zeitpunkt, auf den sich der Wert bezieht, nicht den Veröffentlichungszeitpunkt.
  \item Sei $K = \{1, 2, \dots\}$ die Menge der Nowcaster.
	\begin{itemize}
	  \item Sei $X_{t \lvert \tau}^k (k \in K, t \in T, \tau \in T_t^k)$ der Nowcast von $k$ bezüglich des Zeitpunktes $t$, der am Zeitpunkt $\tau$ \textbf{veröffentlicht} wird (oder: berechnet / dessen Informationen sich auf Zeitpunkt $\tau$ beziehen).
	  \item Sei $g: \mathbb{R}^{\lvert T_t^k \lvert} \rightarrow \mathbb{R}$ eine Aggregationsfunktion, die alle Nowcasts bezüglich eines Zeitpunkts zu einem Nowcast zusammenfasst ($X_t^k \coloneqq g((X_{t \lvert \tau}^k)_{\tau \in T_t^k})$). Falls jeweils nur ein Nowcast veröffentlich wird, wähle $g(x) = x$.
	  \item Sei $(X_t^k)_{t \in T}$ die Zeitreihe der aggregierten Nowcasts.
	\end{itemize}
  \item Sei $\lag[l]$ der lag-$l$-Operator, der für eine Zeitreihe $(Z_t)_{t \in T}$ definiert wird durch 
		\begin{equation}
			\lag[l]Z_t = Z_{t+l} - Z_t \quad (t \in T)
		\end{equation} 
  \item Der Anteil an konkordanten Punkten für Realisierungen $y = (y_1, y_2, \dots, y_n)$ und $(x_1, x_2, \dots, x_n)$ bezüglich des Lag $l$ ergibt sich dann durch
	\begin{equation}
  		k (x, y; l, \epsilon) = \frac{\sum_{t}^{n-l} k^s (\lag y_t, \lag x_t; l, \epsilon)}{\sum_{t}^{n-l} k^\epsilon (\lag y_t, \lag x_t)}
	\end{equation}
	wobei $k^s$ die Indikatorfunktion für die Konkordanz zweier Werte außerhalb der exclusion area ist
	\begin{equation}
  		k^s (x, y; \epsilon) \coloneqq \begin{cases}
  			1 &, \text{falls}\ xy > 0\ \text{und}\ k^\epsilon(x, y; \epsilon) = 1\\
  			0 &, \text{sonst}
  		\end{cases}
	\end{equation}
	und $k^\epsilon$ die Indikatorfunktion für Punkte außerhalb der exclusion area ist
	\begin{equation}
  		k^\epsilon (x, y; \epsilon) \coloneqq \begin{cases}
  			1 &, \text{falls} \ \abs{x} > \epsilon \text{oder} \ \abs{y} > \epsilon \\
  			0 &, \text{sonst}
  		\end{cases}
	\end{equation}
\end{itemize}

\section{Trending-Problem}
\subsection{Grundproblem}
Gegeben sind zwei Zeitreihen $(x_t)$ und $(y_t)$, $t\in \{t_0,...,t_N\}$. Dabei kann $(x_t)$ die zeitliche Entwicklung einer Zielgröße und $(y_t)$ die zeitlichen Schätzungen eines Nowcasters darstellen. Uns interessiert, ob die Entwicklungen der beiden Zeitreihen den gleichen Trend aufweisen bzw. ob der Nowcaster die Entwicklung der zu schätzenden Größe mit dem richtigen Trend schätzen kann. Es sollte ein Maß konstruiert werden, das beschreibt, wie gut zwei Zeitreihen den gleichen Trend aufweisen.
\subsection{Anforderungen an einem Maß/ Intuitionen}

Zwei Zeitreihen weisen den gleichen Trend auf, wenn für zwei beliebige Zeitpunkte $t$ und $\tau$ gilt: $(x_t-x_\tau)(y_t-y_\tau)>0$ oder $x_t=x_\tau \land y_t = y_\tau$.

\todo{Kann man theoretisch nicht einfach mit Rang-korrelations-koeffizient Trending überprüfen?}
\subsection{Ein einfaches Maß}
Sei $\Delta_{t,\tau}^{x}:=x_\tau-x_t$.
\begin{equation}
  		R_{t,\tau}^{x,y} \coloneqq 
        \begin{cases}
  		    \frac{\Delta_{t,\tau}^{x}\Delta_{t,\tau}^{y}}   {\vert\Delta_{t,\tau}^{x}\vert\vert\Delta_{t,\tau}^{y}\vert } &, \text{falls} \Delta_{t,\tau}^{x}\neq 0 \land \Delta_{t,\tau}^{y}\neq0\\
           1 &, \text{falls} \Delta_{t,\tau}^{x}=\Delta_{t,\tau}^{y}=0\\
            -1 &, \text{sonst} 
  	\end{cases}
	\end{equation}
Beispiel-Maß: $S(\bold{x},\bold{y})=\sum_{k=1}^{k^*}\frac{w_k}{N+1-k}\sum_{j=0}^{N-k}R_{t_j,t_{j+k}}^{x,y},$
wobei $w_k, k=1,...,k^*$ eine Gewichtung für das Trending-Verhalten in verschiedenen Zeitdistanzen darstellt. Dabei erfüllt $w_k$ die folgenden Anforderungen:

\begin{itemize}
    \item $w_k$ ist streng monoton fallend in $k$. Hintergrund: je klein der betrachtete Zeitabstand ist, desto größer ist hier der Einfluss des Rauschens
    \item $\sum_{k=1}^{k^*}w_k=1$
\end{itemize}

\subsection{Eigenschaften des vorgeschlagenen Maßes}
\begin{itemize}
    \item Seien $(x_n)$ und $(y_n)$ zwei Realisierungen eines Random Walk -Modells.\\
    Hypothese: $S((x_n),(y_n))=0$. \\
    Intuition:  Zwei Random Walks $(x_n)$ und $(y_n)$ zeigen kein gemeinsames Tendenzverhalten auf. 
\end{itemize}
 
 \subsection{Alternative Notation}

\begin{itemize}
	\item Seien $x_1, x_2, \dots, x_N$ die (aggregierten) Nowcasts; $y_1, x_2, \dots, y_N$ die realisierten Werte
	\item Gewichteter Anteil an konkordanten Differenzen: 
	\begin{equation}
		\sum_{l = 1}^{l^*} \omega_l \sum_{t=1}^{N - l} \frac{\lag x_i \lag y_i}{\abs{\lag x_i} \abs{\lag{y_i}}}, 
	\end{equation}
	wobei $\infty / \infty \coloneqq 1$ und $\infty / - \infty \coloneqq -1$.
	\item Evtl. geschicktere Lokalisierung (0: \enquote{perfekte} Gegenläufigkeit, 1: \enquote{Perfektes} Trending bis Lag $l^*$):
	\begin{equation}
		\sum_{l = 1}^{l^*} \omega_l \sum_{t=1}^{N - l} \frac{1}{2 (N-l)} \left( \frac{\lag x_i \lag y_i}{\abs{\lag x_i} \abs{\lag{y_i}}} + (N-l) \right)
	\end{equation}
\end{itemize}



\section{Statistische Modellierung}
\subsection{Modellierung}
\begin{itemize}
  \item Sei $l$ der zeitliche Abstand, mit dem Realisationen verfügbar werden
  \item $Y_{t + l} = Y_t + \lag Y_t$
  \item Die Nowcast schätzen (fehlerbehaftet) $\lag Y_t$ mithilfe von Daten, die zum Zeitpunkt $\tau \in T_t^k$ verfügbar sind, und dem neuesten, bekannten Wert $Y_t$:
	\begin{equation}
  		X_{t \lvert \tau}^k = Y_{\tau - l} + (\widehat{\lag[t-\tau+l] Y_t})_{t \lvert \tau}^k,
	\end{equation}
	wobei
		\begin{equation}
  			\lag[t-\tau+l] Y_t =  (\widehat{\lag[t-\tau+l] Y_t})_{t \lvert \tau}^k + \varepsilon.
		\end{equation}


\end{itemize}
\subsection{Weitere Ideen zu Trending}

\begin{itemize}
  \item \enquote{Schwaches} Trending für lag $l$: Wahrscheinlichkeit, in die gleiche Richtung zu zeigen ist größer als Wahrscheinlichkeit in die falsche Richtung zu zeigen (für lag $l$)
  \item \enquote{Downwards}-Trending: Trends in die negative Richtung werden erkannt, in die positive Richtung nicht
  \item \enquote{Upwards}-Trending: Trends in die positive Richtung werden erkannt, in die negative nicht
  \item Nicht einfach nur 0-1-Kodierung für gleiches Vorzeichen von $\lag x$ und $\lag y$, sondern Gewichtung des $\mathbb{R}^2$, sodass zum Beispiel Punkte nahe der Achsen weniger Gewicht bekommen als solche nahe der Winkelhalbierenden
  \item PCA auf $\lag x$ und $\lag y$, Bestrafung der zweiten Komponente (Abweichung von Gerade): Bestrafung würde aber mehr umfassen, als lediglich den Trend, sondern würde Abweichung von Linearität betreffen
  \item Falls wir Exclusion Area wollen und $\lag Y_t$ heteroskedastisch: relative Werte betrachten?
\end{itemize}

%\printbibliography


\end{document} 